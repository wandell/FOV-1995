\subsection*{Digital frame-buffer}


There are two basic frame-buffer organizations.  The most common type
of display monitor uses an {\em indexed color map} to represent the
image that will be displayed.  In this form, the frame-buffer entries
can be expressed as a single matrix of numbers.  In a perfect system,
each value of the frame-buffer matrix controls the light signal
emitted from a single point on the display screen.  In a conventional
indexed color map scheme, the meaning of the numbers in the
frame-buffer can be manipulated by the programmer.


The digital values in the frame-buffer are converted by a 
{\em digital-to-analog converter} into a continuous
electrical signal.
This electrical signal controls the intensity of the three different
electron beams in the CRT that ultimately determine the intensity of
the light in three types of phosphors emitted from the CRT.


By storing the digital values in a frame-buffer,
one can use the computer to alter an acquired signal,
or one can simply create a signal and write it into the frame-buffer.

Reading and writing the frame-buffer memory is very fast
to permit data to be transferred from the camera to the buffer,
or from the buffer to the display.
Typically, a monitor image is refreshed
$60$ times a second, that is, once every 16.7msec.
If the monitor displays 1000 lines, we must
be able to read out the data
at an average rate of one line per 16.7 microseconds.
If there are a thousand points in a single line,
we must be able to read out the data
at an average rate of one point per 0.0167 microseconds.

Behavioral and physiological experiments that
require only the display of visual stimuli
must create data and write it into the frame-buffer memory.
In this case, calibration mainly involves
understanding the relationship between the
data in the frame-buffer and the display output.
Computer vision applications often involve the
acquisition of image data and interpretation
of the values in the frame-buffer.
In these studies,
calibration means understanding the
relationship between the light incident at the camera
and the digital frame-buffer values.

Cameras and displays are built
to satisfy the demands of the commercial market;
so, image acquistion and display devices must work together smoothly.
One of the most important criterion
for a commercial system
when we drive the monitor with the camera signal
directly, the monitor display
must be a reasonable facsimile of the image at the camera.
We will see the impact of this requirement
over the next few paragraphs.

