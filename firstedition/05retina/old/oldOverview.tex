\begin{figure}
\centerline {
\psfig{figure=../04retina/fig/overview.ps,clip=,height=5.0in}
}
\caption[Overview of Peripheral Pathways]{
Overview of the peripheral visual pathways.
\comment{
The retina rests at the back of the eyeball and contains
several layers of neurons that encode the visual signal.
The axons of the retinal ganglion cells make up the
optic nerve which provides the visual input to more
central neural units.
Signals on the optic nerve from each retina follow
two different paths depending on the location
of the retinal ganglion cell.
Optic nerve fibers originating
in retinal ganglion cells in the temporal portion
of the retina stay on the same side of the brain,
while optic nerve fibers originating
from retinal ganglion cells in the nasal visual field cross
to the opposite side of the brain.
The segregation of the neurons takes place at the optic chiasm.
Some neurons send their outputs to the superior colliculus,
but the majority of neurons send their outputs to the lateral
geniculate nucleus and then on to the visual cortex.
%	Parts
% left side mine
% right side retina near the fovea
% scanned from Rodieck, p. 365, XII-23 but original is from Brown, Watanabe
% and Murakami (1965) The early and late receptor potentials of monkey
% cones and rods.   Cold Spring Harb. Symp. Quanta. Biol. v. 30 p. 457-482.
% right side LGN is a Hubel and Wiesel FIgure 4 from their Ferrier
% Lecture  in 1977 Functional architecture of macaque monkey visual
% cortex, Proc. R. Soc. Lond. B. v. 198 p. 1-59.
% right side cortex is from Lund, 1973 Organization of neurons in the
% visual cortex, area 17, of the monkey (Macaca mulatta), J. Comp.
% Neurol., v. 147 p. 455-496.
}
}
\label{f4:overview}
\end{figure}

In this and the next chapter we will study the
representation of visual information from the photoreceptors, 
through the retina, to the first visual areas in the cortex.
Figure \ref{f4:overview}
illustrates a few of the major
anatomical components of the visual pathways.

