
\subparagraph{Cylic Convolution and Exponentials}

Now consider the predicted response of a shift-invariant linear
system when the input function is an exponential,
say $e^{i / a}$, with parameter $a$.
Using Equation \ref{eq1:c2} and some simple
algebra we find 

\begin{eqnarray}
\label{eq1:eigen}
\resvechati{i}  & = & 
   \sumzeroNk \hat{h}_ k e^{ (i - k)/(a) } \nonumber \\
      & = & \left ( \sumzeroNk \hat{h}_k e ^ { - k / a } \right ) e ^ {i / a } \\
      & = & s e ^ { i / a }
\end{eqnarray}

When the input function is an exponential with a parameter $a$, 
the output
function is also an exponential with the
same parameter, $a$.
The only difference between the input and output is
a scale factor, $s = \sumzeroNk h _k e ^ { - k / a} $.

\begin{figure}
\centerline {
\psfig{figure=../02col/sundraw/expEigen.ps,clip=,height=3.5in}
}
\caption[]{
Exponentials are eigenfunctions of shift-invariant
linear systems.
When we input a function $e^{i/a}$, the output is a scaled
copy of the input, $s e^{i/a}$.
We must meausre only one parameter of the output response
to specify its form.
}
\label{fig1:expEigen}
\end{figure}

A function that passes through a linear system
unchanged except for a scale factor
is called an {\em eigenfunction} of the system.
Equation \ref{eq1:eigen}
shows us that exponentials
are eigenfunctions of shift-invariant linear systems.
That exponentials are e
and convolution is illustrated in 
Figure \ref{fig1:exp.Eigen}
shows the predicted outputs of Campbell and Gubish's
experiment when we use various exponentials as inputs.
Eigenfunctions are particularly simple experimental stimuli 
to use as inputs because we know precisely what to
expect as outputs.
We can focus all of our 
our attention on estimating the one unknown value of the output,
the scale factor.




\subsection{Estimating the System Matrix with Cosinusoids}

Up to this point, we have seen two ways of
estimating a system matrix.
Figure \ref{fig1:RegResp} shows that in general,
for any linear system,
we can use $N$ independent input stimuli, 
measure $N$ response vectors, and use linear regression
to solve for the system matrix.
For the special case of a shift-invariant system, we need
to measure only a single column of the system matrix
since all the columns are the same except for the shift.
This procedure involves only one input stimulus, but we
must measure the response to this input stimulus at
$N$ different output locations.

Sometimes, measurements with a single line input
offer difficult technical challenges.
Consider, for example, the responss
to line inputs measured by Campbell and Gubish
and plotted in Figure \ref{fig1:cg.data}.
The main differences in the experimental linespread measurements
are present in the tails of the functions
which are hard to measure compared to the noise.
Measuring with other types of stimuli can sometimes
reduce some of the problems due to noise.

\begin{figure}
\centerline {
\psfig{figure=../02col/sundraw/sinResp.ps,clip=,height=3.5in}
}
\caption[]{}
\label{fig1:sinResp}
\end{figure}
The simple relationship between
cosinusoids and shift-invariant systems offers us
a nice way to estimate the system matrix without doing
too much additional work.
The idea is illustrated in Figure \ref{fig1:sinResp}.
We use a set of cosinusoids at increasing frequency as 
input stimuli.
Place the cosinusoids, $\CfNi$, 
from $f = 0$ to $f = {N / 2}$,
in the ${ N / 2 } + 1$ columns of
a matrix that stores all of the input stimuli.
(The cosinusoid at $f = 0$ is really just the constant function).
As we have just seen,
the outputs to these
stimuli will be shifted and scaled copies of the input cosinusoids
at the corresponding frequencies.
Our measurement procedure is relatively simple
because we only have to measure two parameters of the
output:  the scale factor and the phase position.
Knowing what output to expect can help
us to eliminate in the  measurements.
We place the measured outputs to the constant
function and the cosinusoids in the first 
${N / 2 } + 1$ columns of an output matrix.

Finally, we can fill up the rest of the input and output
matrices.
We use sinusoidal functions, $\SfNi$,
ranging from $f = 1$ to $f = {N / 2} - 1$ as inputs.
(The sinusoidal functions at $f = 0$ and $f = {N / 2 }$ are the
zero-functions, so we don't need to include them.)
Because the system is shift-invariant and sinusoids are shifted
copies of cosinusoids, we can fill in the corresponding output columns
without making any additional measurements.
Just fill in the final columns of the output matrix with shifted copies
of the first set of columns.



\subsection{Discussion}

Do be careful about two aspects of the scope of these results.
First, these calculations are based on the idealized, infinite
shift-invariant linear system which
gives rise to convolution equation \ref{eq1:cc}.
Our physical measurements
never include an infinite stimulus, 
such as a sinusoid or an exponential.
The real system may approximate
the idealized behavior;
but it cannot match it precisely.
Second,
these beautiful relations are true only for shift-invariant linear
systems, not for all linear systems.
As we progress,
we will see many examples of important linear systems in the
visual pathways.
Not all of these systems will be shift-invariant.
Exponential and sinusoidal functions do not have the
same significance when we analyze linear
but not shift-invariant components of the
visual pathways.



