
\part{Image Encoding}

\section*{Introduction}
The first section of this book
describes the initial encoding of light by the eye.
Chapter~\ref{chapter:imageformation} reviews the
image formation process, that is, the process
by which light incident at the eye is focused
onto the retina.
Chapters~\ref{chapter:mosaic} and \ref{chapter:wavelength}
review some of the basic properties of the conversion
of light into a neural signal by the light-sensitive
elements of the eye, the {\em photoreceptors}.
These early stages of image encoding
establish essential limits on vision;
the consequences of the image formation process
can found in many parts of the 
visual neural representation.

In addition to the properties of image formation,
the early chapters introduce and make
use of the principles of linear systems.
Linear methods are fundamental to vision science,
as they are to much of science\footnote{
I describe the principles in the text,
and place most of
the mathematical notation and derivations in the
first appendix.}.
Since the methods apply well to image formation,
it seems natural to introduce them as a solution
to the problem of measuring the properties of eye.

The well-designed experiment provides a
method of extrapolating beyond
the experimental measurements,
and linear systems methods provide such a method.
A notion of how to extrapolate
beyond the experimental measurements is necessary because
we can rarely measure the response to every important stimulus.
The significance of linear systems methods
is that they permit us to evaluate
whether we can use the
response to a few stimuli to predict the responses to other stimuli.

\subsection*{Image formation}
The quality and general properties of the
image formed at the retina establish the basic
image parameters that the rest of the nervous system
must use to make inferences about objects.
Because the image formation process is linear,
we can characterize its properties fairly thoroughly.
Measurements from the eye
show that even when optical focus is at its best,
the image of a point is spread across 
eight or more photoreceptors.
It follows that the image formation process
attenuates the contrast of patterns that
vary rapidly across space.
This leaves the nervous system with only
a small contrast range available
in the fine spatial detail of an image,
while there is a substantial
contrast range present in slowly varying spatial patterns.
Finally, the precise meaning of high and low frequency
varies with the wavelength of the incident light
because the quality of the retinal image varies strongly
with wavelength.
Under ordinary viewing conditions
the short wavelength light (blue portion of the spectrum)
is blurred strongly so that very little pattern information
is available in this part of the spectrum compared to
longer wavelengths of light (green, yellow and red parts of the spectrum).

\subsection*{The Spatial Mosaic of Photoreceptors}
Chapter~\ref{chapter:mosaic} reviews the
the spatial arrangement of the light-sensitive elements of
the photoreceptors.
There are two fundamentally different types of receptors,
the rods and cones.
The spatial organization of the rod and cone photoreceptor
mosaics differ; each mosaic reflects
the main goal of the visual stream it initiates.

The rod visual stream initiates vision under low illumination
conditions when relative few quanta are available.
The rods are present in high density
to capture more quanta, not to achieve
high spatial resolution.
Indeed, the spatial resolution of the rod pathway is fairly coarse
since many rod photoreceptors converge their outputs onto
single cells in the retina.

The visual streams initiated by
the cone mosaic ordinarily operate at high light levels
where there are plenty of quanta.
The organization of the cone mosaic
can be understood 
in terms of the goal of representing
fine spatial detail rather than capturing more quanta.
This goal is reflected separately in the spatial arrangement
of the separate mosaics of the three different types of cones,
the $\Red$, $\Green$ and $\Blue$ cones.
The density of the short-wavelength sensitive $\Blue$ cones
is lowest, matching the poor resolution of the optics
in the short-wavelength region.
Only the $\Red$ and $\Green$ cones 
are present in the very central fovea,
where they have a very high
sampling density and form a regular sampling grid.
The sampling density of the $\Red$ and
$\Green$ cones is also a good match to the quality of the
image passed by the optics of the eye in the portion
of the wavelength spectrum where they have their peak sensitivity.
Signals from individual cones in the fovea do not converge
onto retinal neurons, but instead these signals are communicated
along private neural channels to the cortex.

\subsection*{Wavelength Encoding}
Chapter~\ref{chapter:wavelength} reviews
how the visual pathways encode the wavelength of light,
an encoding that has much to do with color appearance.
The behavioral predictions that the eye contains
three types of cones,
as well as behavioral predictions of the way
these cones encode wavelength,
have been confirmed in a stunning set of experiments
that represent an intellectual collaboration between
very different disciplines.
The nexus of results from physics,
psychology, and biology concerning wavelength encoding
form one of most beautiful and satisfying stories in science.
The successful interactions between these disciplines
is a remarkable intellectual achievement.
The facts concerning how the visual pathways encode wavelength
has been important for all color imaging technologies.
The scientific methods that link the
color matching experiment to the cone photocurrents
are important for all of us who wish
to relate behavior and brain.

