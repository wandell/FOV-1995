\comment{
Taken from cvis.4.tex interferometery...
The discussion in Jenkins and White is a good
and simple description of how to calculate the
intensity pattern of the interference fringe.
See pages 265 et seq.
A form of their figure 13F might make it into the problems section.

They derive from the assumptions that
that the amplitude waveform is $4a^2 \cos^2( \delta / 2 )$ where
$\delta$ is the phase difference between the two signals.
When
(a) the signals are far from the screen compared to
one another, and 
(b) the wavelength of the light is small
compared to the distance from the screen, 
then $\delta$ is
approximately proportional to the distance along the screen image.
In that case we can derive the intensity pattern
as a function of distance along the screen
using (CRC Handbook, page 137, double-angle relations)

\begin{eqnarray}
2 \cos^2( x ) = 2cos^2( x ) - ( cos^2(x) + sin^2(x) ) + 1 \nonumber \\
& = & cos^2(x) - sin^2(x) + 1 \nonumber \\
& = & cos( { x } + { x } ) + 1 \nonumber \\
& = & cos ( 2 x ) + 1 
\end{eqnarray}

Substituting this formula
the amplitude distribution is equal to $2 a^2 cos ( \delta ) + 1 $,
which is a cosinusoid modulated around a constant term.
This derivation should probably be in the problems section.
}
