\chapter{Multiresolution Image Representations}
\label{chapter:multiresolution}
\section{Introduction}

Our review of the organization of neural and behavioral data have led
us to several specific hypotheses about how the visual system
represents pattern information.  Neural evidence suggests that visual
information is segregated into a number of different visual streams
that are specialized for different tasks.  Behavioral evidence
suggests that within the streams that are specialized for pattern
sensitivity, information is further organized by local orientation and
spatial scale and color.  All of the evidence we have reviewed
to this point suggest that image contrast, rather than image
intensity, is the key variable represented by the visual pathways.

We will spend this chapter mainly just thinking about how these and
other organizational principles might be relevant to solving various
visual tasks.  In addition to the intrinsic and practical interest of
solving visual problems, finding principled solutions for visual tasks
can also be helpful in understanding and interpreting the organization
of the human visual pathways.

But, which tasks should we consider?  There are many types of visual
problems; only some of these have to do with tasks that are essential
for human vision.  One approach to thinking about visual algorithms,
then, is to adopt a general approach to vision, sometimes called {\em
computational vision}, in which we do not restrict our thinking to
those problems that are important for human vision.  Instead, we open
our minds to visual algorithms that may have no biological
counterpart.

In this book, however, we will restrict our analysis to the subset of
algorithms that is related to human vision.  While this is not the
broadest possible class, it is a very important one because there are
many potential applications for visual algorithms that emulate human
performance.  For example, suppose a person who wants to search
through a database of images to find images with ``red vehicles.''  To
assist the person, the computer program must have some algorithmic
representation related to human vision; after all, the words ``red''
and ``vehicle'' are defined by human perception.  This is but one
example from the set of computer vision algorithms that can serve to
augment the human ability to manipulate, analyze, search and create
images.  These algorithms will be of tremendous importance over the
next few decades.

There is a second reason for paying special attention to visual
problems related to human vision.  Many investigators have argued that
studying the visual pathways and visual behavior is an efficient
method for discovering novel algorithms for computational vision.  The
idea is that by studying the specific properties of a successful visual
system, we will be led to an understanding of the general design
principles of computational vision.  This process is analogous to the
idea of {\em reverse-engineering } that is often used to improve
instrument design in manufacturing.  This view has been suggested by
many authors, but Marr (1982) has argued particularly forcefully that
biology is a good source of ideas for engineering design.  I have
never been persuaded by this argument; it seems to me that
reverse-engineering methods are most successful when one understands
the fundamental principles and only wishes to improve the
implementation.  It is very difficult to analyze how a system works
from the implementation unless one already has a set of general
principles as a guide.  I think engineering algorithms have done more
for understanding the neuroscience than neuroscience has done for
engineering algorithms.

Whichever way the benefits flow, from neuroscience to engineering or
the other way around, just the presence of a flow is a good reason for
the vision scientist and imaging engineer to be familiar with
biological, behavioral, and computational issues.  In the remainder of
this book, we will spend more time engaged in thinking about
computational issues related to human vision.  In this chapter I will
describe ideas and algorithms related to multiresolution image
representations.  In the following chapters I will describe work on
color appearance, motion and objects.  Algorithms for all of these
topics continue be an important part of vision science and
engineering.

\section{Efficient Image Representations}
\label{sec6:efficient} In this chapter we will consider several
different multiresolution representations.  Multiresolution
representations have been used as part of a variety of visual
algorithms ranging from image segmentation to stereo depth and motion
(e.g., Burt, 1988; Vetterli, 1992).  To unify the introduction of
these various representations, however, I will introduce them all by
considering how they solve a single engineering problem: efficient
storage of image image information.

Efficient image representations are important for systems with finite
resources, which is to say all systems.  No matter how much computer
memory or how many visual neurons we have, we can always perform
better computations, transmit more information, or store higher
quality images if we use efficient storage algorithms.  If we fail to
consider efficiency, then we waste resource that could improve
performance.

Image compression algorithms transform image data from one
representation to a new one that requires less storage space.  To
evaluate the efficiency of a compression algorithm, we need some way
to describe the amount of space required to store an image.  The most
common way to measure the amount of storage space necessary to encode
an image is to count the total number of bytes used to represent the
image\footnote{ A bit is a single {\bf b}inary dig{\bf it}, that is, 0
or 1.  A byte is 8 bits and represents 256 levels ($2^8)$.  A megabyte
is $10^6$ bytes, while a gigabyte is $10^9$ bytes.}.  Color images
acquired from cameras or scanners, or color images that are about to
be displayed on a monitor, are represented in terms of the intensities
at a set of picture locations, called {\em pixels}.  The color data
are represented in three color bands, usually called the red, green
and blue bands.  We can compute the number of bytes data represented
in a single image fairly easily.  Suppose we have a modest size image
of $512$ rows and $512$ columns, and that each color band represents
intensity using one byte.  The image representation within a single
color band requires $512 \times 512 \times 3$ bytes of data, or
approximately $0.75$ Megabytes of data.  If we have an image
comprising $1024$ rows and columns, we will require $3.0$ Megabytes to
represent the image.  In a movie clip, in which we update the image
sixty times a second, the numbers grow at an alarming rate; one minute
requires 10 Gigabytes of information, and one hour requires 600
Gigabytes.

Notice that color image encoding already uses a significant amount of
image compression that is made possible by the special characteristics
of human vision.  The physical signal consists of light with energy at
many wavelengths, i.e., a complete spectral power distribution.  The
image data, however, does not encode the complete spectral power
distribution of the displayed or acquired color signal.  The data
represent only three color bands, a very compressed representation of
the image.  The results of the color-matching experiment justifies the
compression of information (see Chapter~\ref{chapter:wavelength}).
This part of compression is so well understood, it is rarely mentioned
explicitly in discussions of image compression.

In addition to color trichromacy, two main factors permit us to
compress images with little loss of quality.  First, adjacent pixels
in natural images tend to have similar intensity levels.  We say that
there is considerable {\em spatial redundancy} in these images.  This
redundancy is part of the signal, and it may be removed without any
loss of information in order to obtain more efficient representations.
Second, we know that human spatial resolution to certain spatial
patterns is very poor (see Chapters~\ref{chapter:imageformation} and
\ref{chapter:space}).  People have very poor spatial resolution to
short-wavelength light, and only limited spatial resolution for
colored patterns in general.  Representing this information in the
stored image is unnecessary because the receiver, that is the visual
system, cannot detect it.  By eliminating this information, we improve
the efficiency of the image representation.

In this chapter we will consider efficient encoding algorithms of
monochrome images, spending most of our time on issues of intensity
and spatial redundancy.  In the next Chapter \ref{chapter:Color},
which is devoted to color broadly, we will again touch on some of the
issues of color image representation.

\subsection*{Intensity Redundancy in Image Data} Suppose that we have
an image we wish to encode efficiently, such as the image in
Figure~\ref{f7:grayResolution}a.  The camera I used to acquire this
image codes up to $256$ different intensity levels ($8$ bits).  You
might imagine, therefore, that this is an $8$ bit image.  To see why
that is not the case, let's look at the distribution of pixel levels
in the image.

\begin{figure}
\centerline{ \psfig{figure=../07imgrep/fig/grayResolution.ps,clip=
 ,width=5.5in} }
\caption[Gray Scale Resolution]{ {\em The distribution of image
intensities} is an important factor in obtaining an efficient image
representation.  (a) This image was acquired by a device capable of
reproducing $256$ gray-levels.  But, the image data consists of only
16 different gray levels.  (b) A histogram of the gray levels used to
code the image shown in (a).  Device properties limit the gray-level
resolution; they do not enforce a resolution.  }
\label{f7:grayResolution}
\end{figure}

In Figure~\ref{f7:grayResolution}b I have plotted the number of points
in the image at each of the 256 intensity levels the device can
represent.  This graph is called the image's {\em intensity
histogram}.  The histogram shows that intensity level $128$ occurs
quite frequently and only a few other levels occur at all.

Although the device used to acquire this image could potentially
represent $256$ different intensity levels, the image itself does not
contain this many levels.  We do not need to represent the data at the
device resolution, but only at the intrinsic resolution of the image,
which is considerably smaller.  Since the image in
Figure~\ref{f7:grayResolution}a contains $16$ levels, not $256$, we
can represent it using $4$ bits per image point rather than the $8$
bit resolution the device can manage.  This saves us a factor of two
in storage space.

The first savings in efficiency is easy to understand; we must not
allocate storage space to intensity levels that do not occur in the
image.  We can refine this idea by taking advantage of the fact that
the different intensity levels do not occur with equal frequency.
Consider one method to take advantage of the fact that some intensity
levels are more likely than others.  Assign the one-bit sequence, $1$,
to code level $128$.  Encode the other levels using a five bit
sequence that starts with a zero, say $0xxxx$, where $xxxx$ is the
original four bit code.  For example, the level $5$ is coded by the
five bit sequence $00101$.  We can unambiguously decode an input
stream as follows.  When the first bit is a one, then the current
value is $128$.  When the first bit is a zero, read four more bits to
define the intensity level at that pixel.

In this image, the intensity level $128$ occupies sixty percent of the
pixels.  Our encoding scheme reduces the space devoted to coding these
pixels from $4$ bits to $1$ bit, saving $3$ bits at $60$ percent of
the locations.  The encoding method costs us 1 bit of storage for 40
percent of the pixels.  The average savings across the entire image is
$0.6 \times 3 - 0.4 \times 1 = 1.2$ bits per pixel.  Using these very
simple rules, we have reduced our storage requirements to $2.8$ bits
per pixel.

The example I have provided here is very simple; many more elaborate
and efficient algorithms exist for taking advantage of the
redundancies in a data set.  In general, the more we know about the
input distribution the better we can do at designing efficient codes.
A great deal of thought has been to the question of designing
efficient coding strategies for single images and also for various
classes of images such as business documents and natural images.  I
will not review them here, but you will find references to books on
this topic in the bibliography.

\subsection*{Spatial Redundancy in Image Data} Normally, intensity
histograms of natural images are not as coarsely discretized
as the example in Figure~\ref{f7:grayResolution}.  In natural images, intensity
distributions range across many intensity levels and
strategies that rely only on intensity redundancy do not save much
storage space.

\begin{figure}
\centerline{ \psfig{figure=../07imgrep/fig/Barlow.ps,clip=
  ,width=5.5in} }
\caption[Pixel Redundancy: Barlow]{ {\em Experimental measurement of
spatial redundancy in an image.}  The image shows Professor Horace
Barlow; random noise is added to the picture.  Subjects were asked to
adjust the intensity of the noisy pixels to the level they thought
must have been present in the original image.  Subjects are very
accurate at this task, using the information present in nearby pixels.
This is an experimental demonstration that people can take advantage
of the spatial redundancy in image data (Source: Kersten, 1987).
\comment{Figure 1(a) from Kersten, JOSA, V4 N 12 1987: Predictability
and redundancy of natural images. } }
\label{f7:kersten}
\end{figure} 
But there are {\em spatial redundancies} in natural images, and we can
use the same general encoding principles we have been discussing to
take advantage of these redundancies as well.  Specifically, certain
spatial patterns of pixel intensities are much more likely than
others.  There are various formal and informal ways to convince
oneself of the existence of these spatial redundancies.  First,
consider the image in Figure~\ref{f7:kersten}.  This figure contains a
picture of Professor Horace Barlow, an eminent visual scientist.  A
few of the pixel intensities have been set randomly to a new intensity
value.  Kersten (1987) has shown that naive observers are quite good at
adjusting the intensity of these pixels back to their original
intensity.  With one percent of the pixels deleted, observers correct
the pixel intensity to its original value nearly 80 percent of the
time.  Even with forty percent of the pixels deleted observers set the
proper intensity level more than half the time.

\begin{figure}
\centerline{ \psfig{figure=../07imgrep/fig/correlogram.ps,clip=
  ,width=5.5in} }
\caption[Pixel redundancy: Correlograms]{ {\em Computational
measurement of spatial redundancies in a natural image.}  The natural
image used for these computations is shown in the middle.  (a) The
image intensity histogram shows the distribution of image intensities.
(b) A correlogram of the intensity at a pixel located at position
$(x,y)$ and the intensity of a pixel located at position (x,y+1).  (c)
A correlogram of the intensity at a pixel located at position $(x,y)$
and the intensity difference between it and the adjacent pixel at
$(x,y+1)$.  (d) A histogram of the intensity differences showing that
they are concentrated near the zero level.  }
\label{f7:pixel.correlogram}
\end{figure}

Second, we can measure the spatial redundancy in natural images by
comparing intensities at neighboring pixels.
Figure~\ref{f7:pixel.correlogram}a shows the pixel intensities from
the image shown in the center of the figure.  Measured one at a time,
the pixel intensities are distributed across many values and do not
contain a great deal of redundancy.  Figure
\ref{f7:pixel.correlogram}b shows an image {\em cross-correlogram}
that measures the intensity of a pixel, $p(x,y)$, on the horizontal
axis and the intensity of its neighboring pixel, $p(x,y+1)$, on the
vertical axis.  Because adjacent pixels tend to have the same
intensity level, the points in the cross-correlogram cluster near the
identity line.  Because the intensity of one pixel tells us a great
deal about the probable intensity level of an adjacent pixel, we know
that the pixel intensity levels are redundant.

We can improve the efficiency of the image representation by removing
this spatial redundancy.  One way of removing the redundancy is to
transform the image representation.  For example, instead of coding
the intensities at the two pixels at adjacent locations independently,
we can code one pixel level, $p(x,y)$ and the difference between the
adjacent pixel values, $p(x,y+1) - p(x,y)$.  This pair of values
preserves the image information since we can recover the original from
$p(x,y)$ and $p(x,y+1)-p(x,y)$ by a simple subtraction.

After transforming the data, the number of bits needed to code
$p(x,y)$ is unchanged.  But the difference, $p(x,y+1)- p(x,y)$, can
fall in a larger range, anywhere between $255$ and $-255$ so that we may
need as many as $9$ bits to store this value.  In principle, requiring
an additional bit is worse, but in practice the difference between
most adjacent pixels is quite small.  This point is illustrated by the
cross-correlogram of the transformed values shown in
Figure~\ref{f7:pixel.correlogram}c.  The horizontal axis measures the
pixel intensity $p(x,y)$, and the vertical axis measures the
difference value, $p(x,y+1) - p(x,y)$.  First, notice that most of the
values of the intensity difference cluster near zero.  Second, notice
that there is virtually no correlation between the transformed values;
knowing the value of $p(x,y)$ does not help us know the value of the
difference.

To build an efficient representation, we can use the same strategy I
outlined in the previous section.  We use a short code (say, 5 bits)
to encode the small difference values that occur frequently.  We use a
longer code (say, 10 bits) to encode the rarely occurring large
values.  Because most of the pixel differences are small, the
representation will more efficient.\footnote{ We could improve even
this coding strategy in many different ways.  For example, after the
first pair of pixels we never need to encode an absolute pixel level,
we can always encode only differences between adjacent pixels.  This
is called Differential Pulse Code Modulation, or {\em DPCM}.  Or, we
could consider the pair of pixels as a vector, calculate the frequency
distribution of all possible vectors, and build an efficient code for
sending communicating the values of these vectors.  This is called
Vector Quantization, or {\em VQ}.  All of these methods trade on the
fact that natural images are more likely to contain some spatial
patterns than others.  }

\subsection*{Decorrelating Transformations} We can divide the image
compression strategies I have discussed into two parts.  First, we
linearly transformed the image intensities to a new representation by
a linear transformation.  The linear transformation computes $p(x,y)$
and $p(x,y)-p(x,y+1)$ from $p(x,y)$ and $p(x,y+1)$.  The
matrix form of this transformation is simply
\begin{equation} \label{e7:decorrelate}
 \left ( \begin{array}{c} p(x,y) \\ p(x,y+1) - p(x,y)
 \\ \end{array} \right ) = \left ( \begin{array}{cc} 1 & 0 \\ -1 & 1
 \\ \end{array} \right ) \left ( \begin{array}{c} p(x,y) \\ p(x,y+1)
 \\ \end{array} \right )
\end{equation} We apply the linear transformation because the
correlation of the transformed values is much smaller than the
correlation in the original representation.

Second, we find a more efficient representation of the transformed
representation.  Because we have removed the correlation, in natural
images the variation of the transformed values will be smaller than
the variation of the original pixel intensities.  Hence we will be
able to encode the transformed data more efficiently than the original
data.

From our example, we can identify
a key property of the linear transformation
that is essential for achieving efficient coding.
The new transformation should convert the data to {\em decorrelated}
values.  Values are decorrelated when we gain no advantage in
predicting one value from knowing the other.  It should seem intuitive
that decorrelation is important part of efficiency: if we can predict
one value from the another, there is no reason to encode both.
Generalizing this idea, if we can predict approximately predict one
value from another, we can achieve some efficiencies in our
representation.  In our example we found that the value $p(x,y)$ is a
good predictor of the value $p(x,y+1)$.  Hence, it is efficient to
predict that $p(x,y+1)$ is equal to $p(x,y)$ and to encode only the
error in our prediction.  If we have a good predictor (i.e., high
correlation) the prediction error will span a smaller range than the
data value.  Hence, the error can be encoded using fewer bits and we
can save storage space.

The transformation in Equation~\ref{e7:decorrelate} does yield a pair
of approximately decorrelated values.  To make the example simple, I
chose a simple linear transformation.  We might ask how we might find
a decorrelating linear transformation in general.  When the set of
images we will have to encode is known precisely, then the best linear
transformation for lossless image compression can be found using a
matrix decomposition called the {\em singular value decomposition}.
The singular value decomposition defines a linear transformation from
the data to a new representation with statistically independent values
that are concentrated over smaller and smaller ranges.  This
representation is just what we seek for efficient image encoding.  The
singular value decomposition is at the heart of principal components
analysis and goes by many other names including the Karhunen-Loeve
transform and the Hoteling transform.  The singular value
decomposition may be the most important technique in linear algebra.

In practice, however, the image population is not known precisely.
Nor are the image statistics for the set of natural images are known
precisely.  As a result, the singular value decomposition has no ready
application to image compression.  As a practical matter, then,
selecting a good initial linear transformation remains an
engineering skill acquired by experience with algorithm design.

\subsection*{Lossy Compression} To this point, we have reviewed
compression methods that transform the original data with no loss of
information.  Since we can recover the original image data perfectly
from the compressed data, the methods are called {\em lossless} image
compression.  Ordinarily, we can achieve savings of a factor of two or
three based on lossless compression methods, though this number is
strongly image dependent.

If we are willing to tolerate some difference between the original
image and the stored copy, then we can develop schemes that save
considerably more space.  Transformations that lose information are
called {\em lossy} image compression methods.  Using only three sensor
responses to represent color information is the most successful
example of a perceptually lossless encoding.  We can not recover the
original wavelength representation from the encoded signal.  Still, we
use this lossy representation because we know from the color-matching
experiment that when done perfectly there should be no difference
between the perceived image and the original image (see
Chapter~\ref{chapter:wavelength}).

Lossy compression is inappropriate for many types of applications,
such as storing bank records. But, some amount of image distortion is
acceptable for many applications.  It is possible to build lossy image
compression algorithms for which the difference between the original
and stored image is barely perceptible, and yet the savings in storage
space can be as large as a factor of five or ten.  Users often judge the
efficiency to be worth the image distortion.  In cases when the image
distortion is not visible, some authors refer to the compression as
{\em perceptually lossless}.

As we reviewed in Chapters~\ref{chapter:imageformation} and
\ref{chapter:space}, the human visual system is very sensitive to some
patterns and wavelengths but far less sensitive to others.
Perceptually lossless encoding methods are designed to account for
these differences in human visual sensitivity.  These schemes allocate
more storage space to represent highly visible patterns and less
storage space to represent poorly visible patterns (Watson and
Ahumada, 1989;  Watson, 1990).

Perceptually lossless image encoding algorithms follow a logic that
has much in common with the lossless encoding algorithms.  First, the
image data are transformed to a new set of values, using a linear
transformation.  The transformed values are intended to represent {\em
perceptually decorrelated features}.  Second, the algorithm allocates
different amounts of precision to these transformed values.  In this
case, the precision allocated to each transformed value depends on the
visual salience of the feature the value represents; hence, salient
features are allocated more storage space than barely visible
features.  It is at this point in the process where lossy algorithms
differ from lossless algorithms.  Lossless algorithms allocate enough
storage so that the transformed values are represented perfectly, yet
due to the decorrelation they still achieve some savings.  Lossy
algorithms do not allocate enough storage to perfectly represent the
initial information; the image cannot be reconstructed perfectly from
the compressed representation.  The lossy algorithm is designed,
however, so that the lost information would not have been visible
anyway.  Thus, the new picture will require less storage and still
look like the original image\footnote{ In practice, lossy and lossless
compression are concatenated to compress image data. First a lossy
compression algorithm is applied, followed by a lossless algorithm.}.


\subsection*{Perceptually Decorrelated Features}
\begin{figure}
\centerline {
  \psfig{figure=../07imgrep/fig/features.ps,clip=,width=5.5in} }
\caption[Image features: Matrix Tableau]{ {\em An operational
definition of perceptual features.}  (a) Image compression usually
begins with a linear transformation that maps image intensities into a
set of transformed coefficients.  (b) An operational definition of an
image feature, associated with that linear transformation, is to find
the image whose transformation results in a representation that is
zero at all values except for one transformation coefficient.  }
\label{f7:features}
\end{figure} In my overview of perceptually lossless compression
algorithms, I used -- but did not define -- the phrase ``perceptually
decorrelated features.''  The notion of a ``perceptual feature'' is
widely used in a very loose way to describe the image properties that
are essential for object perception.  There is no widely agreed on the
specific image features that comprise the perceptual features.  In the
context of image compression, however, we can use a very useful
operational definition for perceptual feature, defined in terms of the
linear transformation used to decorrelate the image data.  The idea is
illustrated in the matrix tableau drawn in Figure~\ref{f7:features}.

Suppose we represent the original image as a list of intensities, one
intensity for each pixel in the image.  We then apply a linear
transformation to the image data, as shown in
Figure~\ref{f7:features}a, to yield a new vector of {\em transform
coefficients}.  This is the same procedure we applied in our simple
example defined by Equation~\ref{e7:decorrelate}.

In the transformed representation, each value represents something
about the contents of the input image.  One way to represent the
visual significance of each transformed value is to identify an input
image that is represented by that transform coefficient alone.  This
idea is illustrated in Figure~\ref{f7:features}b, in which a feature is defined
as the input image that is represented by a set of transform
coefficients that are zero everywhere except at one location.  We call
this image the {\em feature} represented by this transform coefficient.
Using this operational definition, features are defined with respect
to a particular linear transformation.

Next, we must define what we mean by ``perceptually decorrelated'' image
features.  We can use Kersten's (1987) experiment to provide an
operational definition.  In that experiment subjects adjusted the
intensity of certain pixels to estimate the intensity level in the
original image.  Kersten found that observers inferred the intensity
levels of individual pixels quite successfully and that observers
perceived a great deal of correlation when comparing individual
pixels.  We can conclude that pixels are a poor choice to serve as
decorrelated features.

Now, suppose we perform a variant of Kersten's experiment.  Instead of
randomly perturbing pixel values in the image, suppose that we perturb
the values of the transform coefficients.  And, suppose we ask
subjects to adjust the transform coefficient levels to reproduce the
original image.  This experiment is the same as Kersten's task except
except we use the transform coefficients, rather than individual
pixels, to control image features.

We concluded that individual pixels do {\em not} represent
perceptually decorrelated features because subjects performed very
well.  We will conclude that a set of transform coefficients represent
decorrelated features only if subjects perform badly.  When knowing
all the transform coefficients but one does not help the subject set
the level of an unknown coefficient, we will say the features
represented by the transformation are perceptually independent.  I am
unaware of perceptual studies analogous to Kersten's that test for the
perceptual independence of image features; but, in principle, these
experiments offer a means of evaluating the independence of features
implicit in different compression algorithms.

The important compression step in perceptually lossless algorithms
occurs when we use different numbers of bits to represent the
transform coefficients.  To decide on the number of bits allocated to
a transform coefficient, we consider the visual sensitivity of the
image feature represented by that coefficient.  Because visual
sensitivity to some image features is very poor, we can use very few
bits to represent these features with very little degradation in the
image appearance.  This permits us to achieve very compact
representations of image data.  By saving information at the level of
image features, the perceptual distortion of the image can be quite
small while the efficiencies are quite large.

This compression strategy depends on the perceptual independence of
the image features.  If the features are not independent, then the
distortions we introduce into one feature may have unwanted side
effects on a second feature.  If the observer is sensitive to the
second feature, we will introduce unwanted distortions.  Hence,
discovering a set of image features that are perceptually independent
is an important part of the design of a perceptually lossless image
representation.  If distortions of some features have unwanted effects
on the appearance of other features, that is if the representation of
a pair of features is perceptually correlated, then the linear
transformation is not doing its job.
\section{A Block Transformation: The JPEG-DCT}
\label{sec6:dct}
%
%	Get references to JPEG
%
The Joint Photographic Experts Group (JPEG) committee of the
International Standards Organization has defined an image compression
algorithm based on a linear transformation called the {\em Discrete
Cosine Transformation} (DCT).  Because of the widespread acceptance of
this standard, and the existence of hardware to implement the
JPEG-DCT compression algorithm is likely to appear on
your desk and in your home within the next few years.  The JPEG-DCT
compression algorithm has a multiresolution character and bears an
imprint from work in visual perception\footnote{The DCT is similar to
the Fourier Series computation reviewed in
Chapters~\ref{chapter:imageformation} and \ref{chapter:Appendix}.}.

\begin{figure}
\centerline{
  \psfig{figure=../07imgrep/fig/dctBasis.ps,clip= ,width=5.5in}
}
\caption[DCT Basis Functions]{
{\em The perceptual features of the DCT.}  The DCT features are
products of harmonic functions, $\cos(2 \pi f_1 j)cos(2 \pi f_2 k)$,
where $j$ and $k$ refer to position along the horizontal and vertical
directions.  These functions have both positive and negative values,
and they are shown as contrast patterns varying about a constant gray
background.  }
\label{f7:dctBasis}
\end{figure}
	
The JPEG-DCT algorithm uses the DCT to transform the data into
a set of perceptually independent features.
The image features associated with the DCT are shown in Figure
\ref{f7:dctBasis}.  The image features are all products of cosinusoids
at different spatial frequencies and two orientations.  Hence, the
independent features implicit in the DCT are loosely analogous to a
collection of oriented spatial frequency channels.  The features are
not the same as the features used to model human vision since the DCT
image features are comprised of high and low frequencies, while others
contain signals with perpendicular orientations.  Still, there is a
rough similarity between these features and the oriented spatial
frequency organization of models of human multiresolution
representations; this is particularly so for the features pictured
along the edges and along the diagonal in Figure~\ref{f7:dctBasis},
where the image features are organized along lines of increasing
spatial frequency and within a single orientation.

\begin{figure}
\centerline{
  \psfig{figure=../07imgrep/fig/dctAlgorithm.ps,clip= ,width=5.5in}
}
\caption[DCT encoding]{
{\em An outline of the JPEG compression algorithm} based on the DCT.
The original image is divided into a set of nonoverlapping square
blocks, usually $8 \times 8$ pixels.  The image data are transformed
using the DCT to a new set of coefficients.  The transform
coefficients are quantized using using a simple multiply-round-divide
operation.  The quantized coefficients are zeroed by this operation,
making the image well-suited for efficient lossless compression
applied prior to storage or transmission.  To reconstruct the image,
the quantized coefficients are converted by the inverse DCT, yielding
a new image that approximates the original.  The error in the
reconstruction, i.e., the difference between the original and the
reconstruction, consists of mainly high frequency texture.  The error
is shown as an image on the right.}
\label{f7:dctAlgorithm}
\end{figure}
The main steps of the JPEG-DCT algorithm are illustrated in
Figure~\ref{f7:dctAlgorithm}.  First, the data in the original
image are separated into blocks.  The computational
steps of the algorithm are applied separately to
each block of image data, making the algorithm well-suited to parallel
implementation.  The image block size is usually $8 \times 8$
pixels, though it can be larger.  Because the algorithm begins by
subdividing the image into blocks, it is one of a group of algorithms
called {\em block coding} algorithms.

Next, the data in each image block are transformed using the linear
DCT.  The transform coefficients for the image block are shown as an
image in Figure~\ref{f7:dctAlgorithm} labeled ``Transform
coefficients.''  In this image white means a large absolute value and
black means a low absolute value.  The coefficients are represented in
the same order as the image features in Figure~\ref{f7:dctBasis}; the
low spatial frequencies coefficients are in the upper left of the
image and the high spatial frequency coefficients are in the lower
right.

In the next step, the transform coefficients are quantized.  This is
one stage of the algorithm where compression is achieved.  The
quantization is implemented by multiplying each transform coefficient by a
scale factor, rounding the result to the nearest integer, and then
dividing the result by the scale factor.  If the scale factor is
small, then the rounding operation has a strong effect and the number
of coefficient quantization levels is small.  The scalar
values for each coefficient are shown in the image marked
``Quantization scale factor.''  For this example,
I chose large scalar values corresponding to the low spatial
frequency terms (upper left) and small values for the high
spatial frequency terms (lower right).

The quantized coefficients are shown in the next image.  Notice that
many of the quantized values are zero (black).  Because there are so
many zero coefficients, the quantized coefficients are very suitable
for lossless compression.  JPEG-DCT algorithm includes a lossless
compression algorithm applied to the quantized coefficients.  This
representation is used to store or transmit the image.

To reconstruct an approximation of the original image, we only need
to apply the inverse of the DCT to the quantized coefficients.  This
yields an approximation to the original image.  Because of the
quantization, the reconstruction will differ from the original
somewhat.  Since we have removed information mainly about the high
spatial frequency components of the image, the difference between the
original and the reconstruction is an image comprised of mainly fine
texture.  The difference image for this example is labeled ``Error''
in Figure~\ref{f7:dctAlgorithm}.

One of the most important limiting factors in compressing images
arises from the separation of the original image into distinct blocks
for independent processing.  Pixels located at the edge of these blocks
are reconstructed without any information concerning the intensity level of the
pixels that are adjacent, in the next block.  One of the most important
visual artifacts of the reconstruction, then, is the appearance of
distortions at the edges of these blocks, which are commonly called
{\em block artifacts}.  These artifacts are visible in the reconstructed image
shown in Figure~\ref{f7:dctAlgorithm}.

There are two aspects of the JPEG-DCT algorithm that connect it with human
vision.  First the algorithm uses a roughly multiresolution
representation of the image data.  One way to see the multiresolution
character of the algorithm is to imagine grouping together the
coefficients obtained from the separate image blocks.  Within each
block, there are 64 DCT coefficients corresponding to the 64 image
features.  By collecting the corresponding transform coefficients from
each block, we obtain a measure of the amount of each image feature
within the image blocks.  Implicitly, then, the DCT coefficients
define sixty four images, each describing the contribution of the
sixty four image features of the DCT.  These implicit images are
analogous to the collection of neural images that make up a
multiresolution model of spatial vision
(Chapter~\ref{chapter:space})\footnote{ There is something that may
strike you as odd when you think about the JPEG representation in this
way.  Notice that each block contributes the same number of
coefficients to represent low frequency information as high frequency
information.  Yet, from the Nyquist sampling theorem (see
Chapter~\ref{chapter:mosaic}), we know that we can represent the low
frequency information using many fewer samples than are needed to
represent the high frequency information.  Why isn't this differential
sampling rate is not part of the JPEG representation?  The reason is
in part due to the block coding, and in part due to the properties of
the image features.}.

Second, the JPEG-DCT algorithm relies on the assumption
that quantization in the high spatial frequency coefficients does not
alter the quality of the image features coded by low spatial frequency
coefficients.  If reduced resolution of the high spatial frequencies
influences very visible features in the image, then the algorithm will
fail.  Hence, the assumption that the transform yields perceptually
{\em independent} features is very important to the success of the
algorithm.

The independent features in the JPEG-DCT algorithm do not conform perfectly
to the multiresolution organization in models of human spatial vision.
High and low frequency components are mixed in some of the features,
components at very different orientations are also combined in a
single feature.  These features are desirable for efficient
computation and implementation.  In the next section, we will consider
multiresolution computations that reflect the character of human
vision a bit more closely.

\section{Image Pyramids}
\label{sec6:pyramid}

Image pyramids are multiresolution image representations.  Their
format differs from the JPEG-DCT in several ways, perhaps the two most
important being that (a) pyramid algorithms do not segment the image into
blocks for processing, and (b) the pyramid multiresolution representation
is more similar to the human visual representation than that of the
JPEG-DCT.  In fact, much of the interest in pyramid methods in image coding
is born of the belief that the image pyramid structure is well-matched
to the human visual encoding.  This sentiment is described nicely in
Pavlidis and Tanimoto's paper, one of the first on the topic.

\begin{quote}
It is our contention that the key to efficient picture analysis lies
in a system's ability, first, to find the relevant parts of the
picture quickly, and second, to ignore (not waste time with)
irrelevant detail.  The retina of the human eye is ... structured so
as to see a wide angle in a low-resolution (``high-level'') way using
peripheral vision, while simultaneously allowing high-resolution,
detailed perception by the fovea.  [Tanimoto and Pavlidis, page 104].
\end{quote}

The linear transformations used by pyramid algorithms have image
features comprising periodic patterns at a variety of spatial
orientations, much like human multiresolution models.  Because the
coefficients in the image pyramid represent data that fall mainly in
separate spatial frequency bands, it is possible to use different
numbers of transform coefficients to represent the different spatial
frequency bands.  Image pyramids use a small number of transform
coefficients to represent the low spatial frequency features and many
coefficients to represent the high spatial frequency features.  It is
this feature, namely that decreasing number of coefficients are used
to represent high to low spatial frequency features, that invokes the
name pyramid.

\subsection*{The Pyramid Operations:  General Theory}
Image pyramid construction relies on two fundamental operations that
are approximately inverses of one another.  The first operation blurs
and samples the input.  The second operation interpolates the blurred
and sampled image to estimate the original.  Both operations are
linear. I will describe the pyramid operations on one-dimensional
signals to simplify notation;  none of the principles change when we apply these methods to
two-dimensional images.  At the
end of this section, I will illustrate how to extend the
one-dimensional analysis to two-dimensional images.

\begin{figure}
\centerline{
  \psfig{figure=../07imgrep/fig/pyramid1Tableau.ps,clip= ,width=5.5in}
}
\caption[Matrix tableau of one-dimensional pyramid]{
{\em A matrix tableau representation of the one-dimensional pyramid
operations. }  (a)  The basic pyramid operation consists of blurring
and then sampling the signal.  The blurring operation is a convolution
that can be represented by a square matrix whose rows are a
convolution kernel.  The sampling operation can be represented by a
rectangular matrix, consisting of zeros and ones, that pulls out the
sample values from the blurred result.  (b)  To create a series of
images at decreasing resolution, we apply the blurring and sampling
operation recursively.
}
\label{f7:pyramid1Tableau}
\end{figure}
Suppose we begin with a one-dimensional input vector, $g_0$,
containing $n$ entries.  The first basic pyramid operation consists of
convolving the input with a smoothing kernel and then sampling the
result.  The blurring and sampling go together, intuitively, because
the result of blurring is to create a smoother version of the
original, containing fewer high frequency components.  Since
blurring removes high frequency information, according to
the sampling theorem we can represent the blurred data using fewer
samples than the are needed for the original.  We do this by sampling
the blurred image at every other value.  

As we have seen in Chapter~\ref{chapter:mosaic}, both convolution
and sampling are linear operations.  Therefore, we can represent each
by a matrix multiplication.  We represent convolution by the matrix
multiplication $B_0 g_0$, where the rows of $B_0$ contain the
convolution kernel.  We represent sampling by a rectangular matrix,
$S_0$, whose entries are all zeroes and ones.  The combined operation
of blurring and sampling is summarized by the basic pyramid matrix
$P_0 = S_0 B_0$.  Multiplication of the input by $P_0$ yields a
reduced version of the original, $g_1 = P_0 g_0$, containing only half
as many entries; a matrix tableau representing the blurring and
sampling operator, $P_0$, is shown in Figure~\ref{f7:pyramid1Tableau}a.

To create the image pyramid, we repeat the convolution and sampling on
each resulting image.  The first operation creates a reduced image
from the original, $g_1$.  To create the next level of the pyramid, we
blur and sample $g_1$ to create $g_2$; then, we blur and sample $g_2$
to create $g_3$, and so forth.  When the input is a one-dimensional
signal, each successive level contains half as many sample values as
the previous level.  When the image is two-dimensional, sampling is
applied to both the rows and the columns so that the next level of
resolution contains only one-quarter as many sample values as the
original.  This repeated blurring and sampling is shown in matrix
tableau in Figure~\ref{f7:pyramid1Tableau}b.

The second basic pyramid operation, interpolation, serves as an
inverse to the blurring and sampling operation.  Blurring and sampling
transforms a vector with $n$ entries to a vector with only $n/2$
entries.  While this operation does not have an exact inverse, still,
we can use $g_1$ to make an informed guess about $g_0$.  If there is a
lot of spatial redundancy in the input signals, our guess about the original
image may not be too far off the mark.  Interpolation is the process
of making an informed guess about the original image from the reduced
image. We interpolate by selecting a matrix, call it $E_0$, to
estimate the input.  We choose the {\em interpolating} matrix $E_0$ so
that in general $E_0 g_1 \approx g_0$.

We can now put together the two basic pyramid operations into a
constructive sequence will use several times in this chapter.  First,
we transform the input by convolution and sampling, $g_1 = P_0 g_0$.
We then form our best guess about the original using the interpolation
matrix, $\gest{0} = E_0 g_{1} = E_0 P_0 g_{0}$.  The estimate $\gest{0}$
has the same size as the original image.  Finally, to preserve
all of the information, we create one final image to save the error.
The {\em error} is the difference between the true signal and the
interpolated signal, $e_{0} = g_0 - \gest{0}$.  This completes
construction of the first level of the pyramid.

To complete the construction of all levels of the pyramid, we apply
the same sequence of operations, but now beginning with first level of
the pyramid, $g_1$.  We build a new convolution matrix, $B_1$; we
sample using the matrix, $S_1$; we build $g_2 = S_1 B_1 g_1$; we
interpolate $g_2$ using a matrix $E_1 $; finally, we form the new
error image $g_1 - \gest{1}$, where $\gest{1} = E_1 g_2$.  To
construct the entire pyramid we repeat the process, reducing the
number of elements at each step.  We stop when we decide that the
reduced image, $g_n$, is small enough so that no further blurring and
sampling would be useful.

The pyramid construction procedure defines three sequences of signals;
the series of blurred and sampled signals whose size is continually
being reduced, the interpolated signals, and the error signals.  We
can summarize their relationship to one another in a few simple
equations.  First, the reduced image at the $i^{th}$ level is created
by applying the basic pyramid operation to the previous level.
\begin{equation}
\label{e7:reduced}
g_i = P_{i - 1} g_{i-1}
\end{equation}

The estimate of the image $gest{i}$ is created from the lower resolution
representation by the calculation
\begin{equation}
\label{e7:interpolated}
\gest{i} = E_{i+1} g_{i+1}
\end{equation}

Finally, the difference between the original and the
estimate is the error image,
\begin{equation}
\label{e7:error}
e_i = g_i -  \gest{i} = g_i - E_i {P_i} g_i 
\end{equation}

Two different sets of these signals preserve the information in
the original.  One sequence consists of the original input and the
sequence of {\em reduced signals}, $g_0$, $g_1$, $g_2$, \ldots ,
$g_n$.  This sequence provides a description of the original signal at
lower and lower resolution.  It contains all of the data in the
original image trivially since the original image is part of the
sequence.  This image sequence is of interest when we display low
resolution versions of the image.

The second sequence consists of the {\em error signals}, $e_0, e_1, \ldots
e_{n-1}, g_n$ (note that $g_n$ is part of this sequence, too).
Perhaps surprisingly, this sequence also contains all of the
information in the original image.  To prove this to yourself, notice
that we can build the sequence of images, $g_i$, from the error
signals.  The terms $g_n$ and $e_{n-1}$ are sufficient to permit us to
construct $g_{n-1}$; $g_{n-1}$ and $e_{n-2}$ can recover $g_{n-2}$,
and so forth.  Ultimately, we use $e_0$ and $g_1$ to reconstruct the
original, $g_0$.  This image sequence is of interest for image
compression (Mallat, 1989).

\subsection*{Pyramids:  An Example}
Figure~\ref{f7:pyramid1d} illustrates the process of constructing a
pyramid.  The specific calculations used to create this example were
suggested by Burt and Adelson (1983), who were perhaps the first to
introduce the general notion of an image pyramid to image coding.

The example in Figure~\ref{f7:pyramid1d} begins with a one-dimensional
squarewave input, $g_0$.  This signal is blurred using a Gaussian
convolution kernel and then sampled at every other location; the
reduced signal, $g_1$, is shown below.  
This process is then repeated to form a sequence of reduced signals.
When the convolution kernel is
a Gaussian function, the sequence of reduced signals is called the
{\em Gaussian pyramid}.

To interpolate the reduced signal to a higher resolution, Burt and
Adelson proposed the following ad hoc procedure.  Place the data in
$g_1$ into every other entry of a vector with the same number of
entries as $g_0$.  The procedure is called {\em up-sampling}; it is
equivalent to multiplying the vector $g_i$ by the transpose of the
sampling matrix, ${S_0}^t$.  Then, convolve the up-sampled vector with
(nearly) the same Gaussian that was used to reduce the image.  The
Gaussian used for interpolation differs from the Gaussian used to blur
the signals only in that is is multiplied by a factor of $2$ to
compensate for the fact that the up-sampled vector only has non-zero
values at one out of every two locations.  In this important example,
then, the interpolation matrix is equal to two times the transpose of
the convolution-sampling matrix,
\begin{equation}
E_0 = 2 {B_0}^{t} {S_0}^{t} = 2 {P_0}^{t} .
\end{equation}
The interpolated signal, that is, the estimate of the higher
resolution signal, is shown in the middle column of
Figure~\ref{f7:pyramid1d}.

\begin{figure}
\centerline{
 \psfig{figure=../07imgrep/fig/pyramid1d.ps,clip= ,width=5.5in}
}
\caption[The One-dimensional Pyramid]{
{\em One-dimensional pyramid construction.}  The input signal
(upper left) is convolved with a Gaussian kernel and the result is
sampled.  This creates a blurred copy of the signal at lower
resolution.  An estimate of the original is created by interpolating
the low resolution signal, and the difference between the original and
the estimate is saved in the error pyramid.  The process is
repeated, beginning with the blurred copy, thus creating series of
copies of the original at decreasing resolution (on the left) and a
series error images (on the right).  The signal at the lowest
resolution level is stored as the final element in the error pyramid.
}
\label{f7:pyramid1d}
\end{figure}
Next, we calculate the error signal, the difference between the
estimate and the original.  The error signals are shown on the right
of Figure~\ref{f7:pyramid1d}.  The sequence of error signals forms the
error pyramid.  As I described above, we can reconstruct the original
$g_0$ without error from the signals $e_0$ and $g_1$. Burt and Adelson
called the error signals created by the combination of Gaussian
blurring and interpolation functions the {\em Laplacian pyramid}.

\begin{figure}
\centerline{
  \psfig{figure=../07imgrep/fig/pyramid2d.ps,clip= ,width=5.5in}
}
\caption[Image pyramid construction]{
{\em The Gaussian and Laplacian image pyramids.}  (a) The series of
reduced images that form the Gaussian image pyramid begins with the
original image, on the left.  This image is blurred by a Gaussian
convolution kernel and then sampled to form the image at a lower
spatial resolution and size.  (b) Each reduced image in the Gaussian
pyramid can be used to estimate the image at a higher spatial
resolution and size.  The difference between the estimate the higher
resolution image forms an error image, which in the case of Gaussian
filtering is called the Laplacian pyramid.  These error images can
have positive or negative values, so I have shown them as contrast
images in which gray represents zero error, while white and black
represent positive and negative error respectively.  (After Burt and
Adelson, 1983).  }
\label{f7:pyramid2d}
\end{figure}
Figure~\ref{f7:pyramid2d}a shows the result of applying the pyramid
process to a two-dimensional signal, in this case an image.  The
sequence of reduced images forming the Gaussian pyramid is shown on
the top, with the original image on the left.  These images were
created by blurring the original and then representing the new data at
one half the sampling rate for both the rows and the columns.  Thus,
in the two-dimensional case each reduced image contains only
one-quarter the number of coefficients as its
predecessor\footnote{Therefore, in the estimation phase we multiply
the interpolation matrix by a factor of 4, not 2, i.e., $E_0 = 4
{P_0}^t$.}.

The sequence of error images forming the Laplacian pyramid is shown in
Figure~\ref{f7:pyramid2d}b.  Because the interpolation routine uses a
smooth Gaussian function to interpolate the lower resolution images,
the large errors tend to occur near the edges in the image.  And,
because the images are mainly smooth (adjacent pixel intensities are
correlated) most of the errors are small\footnote{In order to display
the error images, which negative coefficients, the image intensities
are scaled so that black is a negative value, medium gray is zero, and
white is positive.}.

\subsection*{Image Compression Using the Error Pyramid}
From the point of view of image compression, the sequence of images in
the Gaussian pyramid is not very interesting because that sequence
contains the original.  Rather than the use the entire sequence, we
might as well just code the original.  The sequence of images in the
Laplacian pyramid, however, is interesting for two reasons.

First, the information represented in the Laplacian pyramid varies
systematically as we descend in resolution.  At the highest levels,
containing the most transform coefficients, the Laplacian pyramid
represents the fine spatial detail in the image. At the lowest levels,
containing the fewest transform coefficients, the Laplacian pyramid
represents low spatial resolution information.  Intuitively, this is
so because the error image is the difference between the original,
which contains all of the fine detail, and an estimate of the original
based on a slightly blurred copy.  The difference between the original
and an estimate from a blurred copy represents image information in
the resolution band between the two levels.  Thus, the Laplacian
pyramid is a multiresolution representation of the original image.

Second, the values of the transform coefficients in the error images
are distributed over a much smaller range than the pixel intensities
in the original image.  Figure~\ref{f7:pyramidEntropy}a shows
intensity histograms of pixels in the first three elements of the
Gaussian pyramid.  These intensity histograms are broad and not
well-suited to the compression methods we reviewed earlier in this
chapter.  Figure~\ref{f7:pyramidEntropy}b shows histograms of the
pixel intensities in the Laplacian pyramid.  The transform
coefficients tend to cluster near zero and thus they can be
represented very efficiently.  The reduced range of transform
coefficient values in the Laplacian pyramid arises because of the
spatial correlation in natural images.  The spatial correlation
permits us to do fairly well in approximating images using smooth
interpolation.  When the approximations are close, the errors are
small, and they can be coded efficiently.
\begin{figure}
\centerline{
  \psfig{figure=../07imgrep/fig/entropy.ps,clip= ,width=5.5in}
}
\caption[Pixel Entropy and Coefficient Entropy]{
{\em Histograms of the Gaussian and Laplacian pyramids.}  (a) The
separate panels show the intensity histograms at each level of the
Gaussian pyramid.  The intensities are distributed across a wide range
of values, making the intensities difficult to code efficiently.  (b)
The Laplacian pyramid coefficients are distributed over a modest range
near zero and can be coded efficiently.}
\label{f7:pyramidEntropy}
\end{figure}

There is one obvious problem with using the images in the Laplacian
pyramid as an efficient image representation: there are more
coefficients in the error pyramid than pixels in the original image.
When building an error pyramid from two-dimensional images, for
example, we sample every other row and every other column.  This forms
a sequence of error images equal to $1$, $1/4$, $1/{16}$ the size of
the original; hence, the error pyramid contain 1.33 times as many
coefficients as the original (see Figure~\ref{f7:pyramid2d}).  Because
of the excess of coefficients, the error image representation is
called {\em overcomplete}.  If one is interested in image compression,
overcomplete representations seem to be a step in the wrong direction.

Burt and Adelson (1983) point out, however, that there is an important
fact pertaining to human vision that reduces the significance of the
overcompleteness: The vast majority of the the transform coefficients
represent information in the highest spatial frequency bands where
people have poor visual resolution.  Therefore, we can quantize these
elements very severely without much loss in image quality.
Quantization is the key step in image compression, so that having most
of the transform coefficients represent information that can be
heavily quantized is an advantage.

The ability to quantize severely many of the transform coefficients
with little perceptual loss, coupled with the reduced variance of the
transform coefficients, make the Laplacian pyramid representation
practical for image compression.  Computing the pyramid can be more
complex than the DCT, depending on the block size, but special purpose
hardware has been created for doing the computation efficiently.  The
pyramid representation performs about as well or slightly better the
JPEG computation based on the DCT.  It is also applicable to other
visual applications, as we will discuss later (Burt, 1988).

\section{QMFs and Orthogonal Wavelets}
\label{sec6:qmfs}

Pyramid representations using a Gaussian convolution kernel have many
useful features; but, they also have several imperfections.  By
examining the problematic features of Gaussian and Laplacian pyramids,
we will see the rationale for using a different convolution kernel,
{\em quadrature mirror filters} (QMFs), in creating image pyramids.

The first inelegant feature of the Gaussian and Laplacian pyramids is an
inconsistency in the blurring and sampling operation.  Suppose we had
begun our analysis with the estimated image, $\gest{0}$, rather
than $g_0$.  From the pyramid construction point of view, the estimate
should be equivalent to the original image.  It seems reasonable to
expect, therefore, that the reduced image derived from $\gest{0}$
should be the same as the reduced image derived from ${g_0}$.  We can
express this condition as an equation,

\begin{equation}
\label{e7:ortho}
g_1 = P_0 (2 {P_0}^t) g_1 = P_0 \gest{0} .
\end{equation}

Equation \ref{e7:ortho} implies that the square matrix ${P_0}( 2
{P_0}^t)$ must be the identity matrix.  This implies that the columns of
the matrix, $P_0$ should be {\em orthogonal} to one
another\footnote{Orthogonality is defined in
Chapter~\ref{chapter:pattern} and the Appendix. Two vectors are
orthogonal when $a^t b = 0$.}.  This is not a property of the Gaussian
and Laplacian pyramid.

A second inelegant feature of the Gaussian and Laplacian pyramid is that the
representation is overcomplete, i.e., there are more transform
coefficients than there are pixels in the original image.  The
increase in the transform coefficients can be traced to the fact that
we represent an image $g_i$ with $N_i$ pixels by a reduced signal and
an error signal that contain more than $N_i$ coefficients.  For
example, we represent the information in
the $i^{th}$ level of the pyramid using the reduced image $g_{i+1}$ and
the error image $e_i$.

\begin{equation}
g_i = (2 {P_i}^t) g_{i+1} + e_i
\end{equation}

In the one-dimensional case, the error image, $e_i$, contains $N_{i}$
transform coefficients.  The reduced signal, $g_{i+1}$, contains
${N_i} / 2 $ coefficients.  To create an efficient representation, we
must represent $g_i$ using $N_i$ transform coefficients, not $1.5 N_i$
coefficients as in the Gaussian pyramid.

The error signal and the interpolated signal are intended to code
different components of the original input; the interpolated vector
$\gest{i} = (2 {P_i}^t) g_{i+1}$ codes a low resolution version of the
original, and $e_i$ codes the higher frequency terms left out by the
low resolution version.  To improve the efficiency of the
representation, we might require that the two terms code completely
different types of information about the input.  One way to interpret
the phrase ``completely different'' is to require that the two vectors
be orthogonal, that is,

\begin{equation}
\label{e7:split}
0 = {e_i}^t \gest{i} .
\end{equation}

If we require that $\gest{i}$ and $e_i$ to be orthogonal, we can
obtain significant efficiencies in our representation.  By definition,
we know that the interpolated image $\gest{i}$ is the weighted sum
of the columns of ${P_i}^t$.  If we the error $e_i$ image is
orthogonal to the interpolated image, then the error image must be the
weighted sum of a set of column vectors that are all orthogonal to the
columns of ${P_i}^t$.  In the (one-dimensional) Gaussian pyramid
construction, ${P_i}^t$ has $N_i / 2$ columns.  From basic linear
algebra, we know that there are $ (1 / 2) N_i$ vectors perpendicular
to the columns of ${P_i}^t$.  Hence, if $\gest{i}$ is orthogonal to
$e_i$, we can describe both of these images using only $N_i$ transform
coefficients, and the representation will no longer overcomplete.

But, what conditions must be met to insure that $e_i$ and $\gest{i}$
are orthogonal?  By substituting Equations~\ref{e7:reduced},
\ref{e7:interpolated} and \ref{e7:error} into Equation~\ref{e7:split} we have
\begin{eqnarray}
\label{e7:pyrOrth}
0 = {e_i}^t \gest{i}
  & = & ( g_i^t (2 {P_i}^t) P_i ) 
        ( (2 {P_i}^t) P_i g_i -   g_i) \nonumber \\ 
  & = & [ g_i^t (2 {P_i}^t) ( P_i (2 {P_i}^t) ) P_i g_i ] 
    - [ g_i^t (2 {P_i}^t) P_i g_i ] .
\end{eqnarray}
If the rows of $P_i$ are an orthogonal set, then by appropriate
scaling  we can arrange it so that
${P_i} (2 {P_i}^t)$ is equal to the identity matrix.
In that case, the final
term in Equation~\ref{e7:pyrOrth} simplifies and we have
\begin{equation}
{e_i}^t \gest{i} = 
   [ {g_i}^t (2 {P_i}^t) P_i g_i ] - [ {g_i}^t (2 {P_i}^t) P_i g_i ] = 0 ,
\end{equation}
thus guaranteeing that the error signal and the interpolated estimate
will be orthogonal to one another.  For the second time, then, we find
that the orthogonality of the rows of the pyramid matrix is a useful
property.

We can summarize where we stand as follows.  The basic pyramid
operation has several desirable features.  The rows within each level
of the pyramid matrices are shifted copies of one another, simplifying
the calculation to nearly a convolution; the pyramid operation
represents information at different resolutions, paralleling human
multiresolution representations; the rows of the pyramid matrices are
localized in space, as are receptive fields, yet they are not sharply
localized as the blocks used in the JPEG-DCT algorithm.  Finally, from our
criticisms of the error pyramid, we have added a new property we would
like to have: The rows of each pyramid matrix should be an orthogonal set.

We have accumulated an extensive set of properties we would like the
pyramid matrices, $P_i$, to satisfy.  Now, one can have a wish list,
but there is no guarantee that there exist any functions that satisfy
all our requirements.  The most difficult pair of constraints to
satisfy is the combination of orthogonality and localization.  For
example, if we look at convolution operators alone, there are no
convolutions that are simultaneously orthogonal and localized in
space.

Interestingly there exists a class of discrete-valued functions,
called {\em quadrature mirror filters}, that satisfy all of the
properties on our wish list (Esteban and Galand, 1977; Simoncelli and
Adelson, 1990; Vetterli; 1988).  The quadrature mirror filter pair
splits the input signal into two orthogonal components.  One of the
filters defines a convolution kernel that we use to blur the original
image and obtain the reduced image.  The second filter is orthogonal
to the first and can be used to calculate an efficient representation
of the error signal.  Hence, the quadrature mirror filter pair splits
the original signal into coefficients that define of the two
orthogonal terms, $\gest{i}$ and $e_i$; Each set of coefficients has
only $n/2$ terms, so the new representation is an efficient pyramid
representation.  Figure~\ref{f7:qmf} shows an example of a pair of
quadrature mirror filters.  The function shown in Figure~\ref{f7:qmf}a
is the convolution kernel that is used to create the reduced images,
$g_i$.  The function in Figure~\ref{f7:qmf}b is the convolution kernel
needed to calculate the transform coefficients in the error pyramid
directly.  When the theory of these filters is developed for
continuous, rather than discrete, functions the convolution kernels
are called {\em orthogonal wavelets} (Debauchie, 1992).
\begin{figure}
\centerline{
  \psfig{figure=../07imgrep/fig/Qmf.ps,clip= ,width=5.5in}
}
\caption[Quadrature Mirror Filters]{
{\em A quadrature mirror filter pair.}  One can use these two
functions as convolution kernels to construct a pyramid.  Convolution
with the kernel in (a) followed by sampling produces the transform
coefficients in the set of reduced signals.  Transformation by the
kernel in (b) followed by sampling yields the transform coefficients
of the error pyramid (Source: Simoncelli, 1988).  }
\label{f7:qmf}
\end{figure}

The discovery of quadrature mirror filters and wavelets was a bit of a
surprise.  It is known that there are no nontrivial convolution
kernels that are orthogonal; i.e., no convolution matrix, $B_i$,
satisfies the property that $B_i B_i^t = I$.  Hence it was surprising
to discover that convolution kernels do exist for the pyramid
operation, which relies so heavily on convolution, can satisfy $P_i
{P_i}^t = I$.

The quadrature mirror filter and orthogonal wavelet representations
have many fascinating properties and are an interesting area of
mathematical study.  They may have significant implications for
compression because they remove the problem of having an overcomplete
representation.  But, it is not obvious that once quantization and
correlation are accounted for that the savings in the number of
coefficients will prove to be significant.  For now, the design and
evaluation of quadrature mirror filters remains an active area of
research in pyramid coding of image data.






\section{Applications of multiresolution representations}

The statistical properties of natural images
make multiresolution representations efficient.
Were efficiency a primary
concern, the visual pathways
might well have evolved to use the multiresolution format.
But, there is no compelling reason to think that
the human visual system, with hundreds of
millions of cortical neurons 
available to code the outputs of tens of thousands
of cone photoreceptors,
was subject to very strong evolutionary pressure to
achieve efficient image representations.
Understanding the neural multiresolution representation
may be helpful when we design image compression algorithms;
but, it is unlikely that neural multiresolution representations
arose to serve the goal of image compression alone.

If multiresolution representations are present in the
visual pathways, what other purpose might they serve?
In this section, 
I will speculate about how multiresolution representations
may be a helpful component of
several visual algorithms.
\comment{
It is possible to create
a multiresolution version of almost any algorithm,
but there are few examples in which the
multiresolution representation is essential.
}

\subsection*{Image Blending}

Imagine blending refers to methods for smoothly
connecting several adjacent or overlapping
images of a scene
into a larger photomosaic (Milgram, 1975;  Carlbom, 1994).
There are several different reasons why we
might study the problem of joining together
several pieces of an image.
For example, in practical imaging applications
we may find that a camera's field of view may be too small
to capture the entire region of interest.
In this case we would like to blend several
overlapping pictures to form a complete image.

The human visual system also needs to blend images.  As we saw in the
early chapters of this volume, spatial acuity is very uneven across
the retina.  Our best visual acuity is in the fovea, and primate
visual systems rely heavily on eye-movements to obtain multiple images
of the scene.  To form a good high acuity representation of more than
the central few degrees, we must gather images from a sequence of
overlapping eye fixations.  How can the overlapping images acquired
through a series of eye movements be joined together into the single,
high resolution representation that we perceive?

Burt and Adelson (1983b) showed that multiresolution image
representations offer a useful framework for blending images together.
They describe some fun examples based on the pyramid representation.

We can see some of the advantages of a multiresolution image blending
by comparing the method with a single resolution blend.  So, let's
begin by defining a simple method of joining the two pictures, based
on a single resolution representation.  Suppose we decide to join a
picture on the left $L(x,y)$ and a picture on the right $R(x,y)$.  We
will blend the images by mixing their intensity values near the border
where the join.  A formal rule for to blend the image data must
specify how to combine the data from the two images.  We do this using
a blending function, call it $b(x,y)$, whose values vary between $0$
and $1.0$.  To construct our single-resolution blending algorithm we
form a mixture image from the weighted average

\begin{equation}
M(x,y) = b(x,y) L(x,y) + ( 1 - b(x,y)) R(x,y) .
\end{equation}

Consider the performance of this type of single resolution blend on an
a pair of simulated astronomical images in Figure \ref{f7:stars}.
Each of these images contain $512$ rows and columns.  The two images
were built to simulate the appearance of a starry sky.  The images
contain three distortions to illustrate some of the advantages of
multiresolution methods for blending images.

First, the images contain two kinds of objects (stars and clouds)
whose spatial structure places them in rather different spatial
frequency bands.  Second, the images have different mean levels (the
image on the top right being dimmer than the one on the top left).
Third, the images are slightly shifted in the vertical direction as if
there was some small jitter in the camera position at the time of
acquiring the pictures.

\begin{figure}
\centerline{
  \psfig{figure=../07imgrep/fig/Stars.ps,clip= ,width=5.5in}
}
\caption[Image Blending]{
{\em A comparison of single-resolution and multiresolution image
blending methods.}  The images in (a) and (b) have a slightly
different mean and are translated vertically.  Abutting the right and
left halves of the images is shown in (c).  Spatial averaging over a
small distance across the image boundary is shown in (d).  Spatial
averaging over a large distance across the image boundary is shown in
(e).  The multiresolution blend from Burt and Adelson is shown in (f).
(Source: Burt and Adelson, 1983).
%Figure 3 from Burt and Adelson Multiresolution Spline with Application.
}
\label{f7:stars}
\end{figure}

Because the images are divided along a vertical line, we need to
concern ourselves only with the variation with $x$ and join the images
the same way across each row.

The most trivial, and rather ineffective, way of joining the two
images is shown in panel (c).  In this case the two images are simply
divided in half and joined at the dividing line.  Simply abutting the
two images is equivalent to choosing a function $s(x,y)$ equal to
\begin{equation}
b(x,y) = 
  \left \{ 
    \begin{array}{ll}
     1 & \mbox{if $x < m$} \\
     0 & \mbox{otherwise}
    \end{array}
  \right.
\end{equation}
where $m$ is the midpoint of the image, $256$ in this case.  This
smoothing function leads to a strong artifact at the midpoint because
of the difference in mean gray level.

We might use a less drastic blending function for $b(x,y)$.  For
example, we might choose as function that varied as a linear ramp over
some central width of the image.
\begin{equation}
\label{e6:ramp}
b(x,y) = 
  \left \{ 
    \begin{array}{ll}
     {1 } & \mbox{if $x < m - w $} \\
     {1 - \frac{x - m - w }{2w}} & \mbox{if $m - w \leq x \leq m + w$} \\
     0 & \mbox{otherwise}
    \end{array}
  \right.
\end{equation}
Using a ramp to join the images blurs the image at the edge, as
illustrated in panels (d) and (e) of figure \ref{f7:stars}.  In panel
(d) the width parameter of the linear ramp, $w$, is fairly small.
When the width is small the edge artifact remains visible.  As the
width is broadened, the edge artifact is removed (panel (e)) and
elements from both images contribute to the image in the central
region.  At this point the vertical shift between the two images
becomes apparent.  If you look carefully in the central region, you
will see double stars shifted vertically one above the other.  Image
details that are much smaller than the width of the ramp appear in the
blended image and they appear at their shifted locations.  The stars
are small compared to the width of the linear ramp, so the blended
image contains the an artifact due to the shift in the image details.

Multiresolution representations provide a natural, way for combining
the two images that avoid some of these artifacts.  We can state the
multiresolution blending method as an algorithm.

\begin{enumerate}
\item Form the pyramid of error images for $L$ and $R$.
\item Within each level of the pyramid, average
the error images with a blend function $b(x,y)$.
A simple ramp function, as in Equation \ref{e6:ramp}
with $w = 1$, will do as the blend function.
\item Compute the new image by reconstructing the
image from the blended pyramid of error images.
\end{enumerate}

The image in panel (f) of Figure \ref{f7:stars} contains the results
of applying the multiresolution blend to the images.  The
multiresolution algorithm avoids the previous artifacts because by
averaging the two error pyramids, two images combine over different
spatial regions in each of the resolution bands.  Data from the low
resolution level is combined over a wide spatial region of the image,
while data from the high resolution levels are combined over a narrow
spatial region of the image.

By combining low frequency information over large spatial regions, we
remove the edge artifact.  By combining high frequency information
over narrow spatial regions, we reduce the artifactual doubling of the
star images to a much narrower spatial region.

Burt and Adelson (1983b) also describe a method of blending images
with different shapes.  Figure \ref{f7:HandEye} illustrates one of
their amusing images.  They combined the woman's eye taken from panel
(a) and the hand, taken from panel (b) into a single image shown in
panel (d).  The method for combining images with different shapes is
quite similar to the algorithm I described above.  Again, we begin by
forming the error images $e_i$ for each of the two images.  For the
complex region, however, we must a method to define a blend function
$s_i (x,y)$, appropriate for combining the data at each resolution of
the pyramid over these different shapes.  Burt and Adelson have a
nifty solution to this problem.  Build an overlay image that defines
the location where second image is to be placed over the first, as in
panel (c) of figure \ref{f7:HandEye}.  Build the sequence of pyramid
of reduced images, $g_i$, corresponding to the overlay image.  Use the
elements of the image sequence $g_i$ to define the blend functions for
combining the images at resolution $e_i$.

\begin{figure}
\centerline{
  \psfig{figure=../07imgrep/fig/HandEye.ps,clip= ,width=5.5in}
}
\caption[Hand Eye Coordination]{
{\em Image blending of regions with arbitrary shape.}
To create a multiresolution
blend of the images of the eye and hand,
we must define a blending function for each level of the pyramid.
The blending function can be created by building the Gaussian pyramid
representation of the region where the image of the eye will be
inserted.  The different levels of the Gaussian pyramid can be used as
the blending functions to combine the error pyramids of the two images
(Source:  Burt and Adelson, 1983).
\comment{Figure 8 from Burt and Adelson Spline paper.}
}
\label{f7:HandEye}
\end{figure}

\subsection*{Progressive Image Transmission}
For many devices, transmitting an image from its stored representation
to the viewer can take a noticeable amount of time.  And, in some of
these cases, transmission delays may hamper our ability to perform the
task.  Suppose we are scanning through a database for suitable
pictures to use in a drawing, or we are checking a directory to find
the name of the person who recently waved hello.  We may have to look
through many pictures before finding a suitable one.  If there is a
considerable delay before we see each picture, the tasks become
onerous; people just won't do them.

Multiresolution image representations are natural candidates to
improve the rate of image transmission and display.  The
reconstruction of an image from its multi-resolution image proceeds
through several stages.  The representation stores the error images
$e_i$ and the lowest reduced image $g_n$.  We reconstruct the original
by computing a set of reduced images, $g_i$.  These reduced images are
rough approximations of the original, at reduced resolution.  They are
represented by fewer bits than the original image, so they can be
transmitted and displayed much more quickly.  We can make these low
resolution images available for the observer to see during the
reconstruction process.  If the observer is convinced that this image
is not worth any more time, then he or she can abort the
reconstruction and go on to the next image.  This offers the observer
a way to save considerable time.

We can expand on this use of multiresolution representations by
allowing the observer to request a low resolution reconstruction, say
at level $g_i$, rather than a full representation at level $g_0$.  The
observer can choose a few of the low resolutions for viewing at high
resolution.  Multiresolution representations are efficient because
there is little wasted computation.  The pyramid reconstruction method
permits us to use the work invested in reconstructing the low
resolution image as we to reconstruct the original at full resolution.

Th engineering issues that arise in progressive image transmission may
be relevant to the internal workings of the human visual system.  When
we call up an image from memory, or sort through a list of recalled
images, we may wish to image low resolution images rather than
reconstruct each image in great detail.  If the images are stored
using a multi-resolution format, our ability to search efficiently
through our memory for images may be enhanced.

\subsection*{Threshold and Recognition}
Image compression methods link {\em visual sensitivity} measurements
to an engineering application.  This makes sense because threshold
sensitivity plays a role in image compression; perceptually lossless
compression methods, by definition, tolerate threshold level
differences between the reconstructed image and the original.

For the applications apart from compression, however, sensitivity is
not the key psychological measure.  Since low resolution images do not
look the same as the high resolution images, sensitivity to
differences is not the key behavioral measure.  To understand when
progressive image transmissions methods work well, or which low
resolution version is the best approximation to a high resolution
version of an image, we need to be informed about which
multiresolution representations permit people to {\em recognize}
quickly or {\em search} for an item in a large collection of low
resolution images quickly.  Just as the design of multiresolution
image compression methods requires knowing visual sensitivity to
different spatial frequency bands, so too multiresolution methods for
progressive image transmission requires knowing how important
different resolution bands will be for expressing the information in
an image.

As we study these applications, we will learn about new properties of
human vision.  To emphasize some of the interesting properties that
can arise, I will end this section by reviewing a perceptual study by
Bruner and Potter (1964).  This study illustrates some of the
counter-intuitive properties we may discover as we move from threshold
to recognition studies.

Bruner and Potter (1964) studied subjects ability to recognize common
objects from low resolution images.  Their subjects were shown objects
using slides projected onto a screen.  In 1964 low resolution images
were created much more quickly and easily than today; rather than
requiring expensive computers and digital framebuffers low resolution
images were created by blurring the focus knob on the projector.

Bruner and Potter compared subjects' ability to recognize images in a
few ways.  I want to abstract from their results two key
observations\footnote{ There are a number of important methodological
features of the study I will not repeat here, and I encourage the
reader to return to the primary sources to understand more about the
design of these experiments.  }.

Figure~\ref{f7:bruner.potter} illustrates three different measurement
conditions.  Observers in one group saw the image develop from very
blurry to only fairly blurry over a two minute period.  At the end of
this period the subjects were asked to identify the object in the
image.  They were correct on about a quarter of the images.  Observers
in a second group only began viewing the image after after 87 seconds.
They first saw the image at a somewhat higher resolution, but then
they could watch the image develop for only about a half minute.  The
difference between the second group and the first group, therefore,
was whether they saw the image in a very blurry state, during the
first 90 seconds.  The second group of observers performed
substantially better, recognizing the object $44$ percent of the time
rather than $25$ percent.  Surprisingly, the initial 90 seconds of
viewing the image come into focus made the recognition task more
difficult.  A third group was also run.  This group only saw the image
come into focus during the last 13 seconds.  The third group did not
see the first 107 seconds as the image came into focus.  This group
also recognized the images correctly about $43$ percent of the time.

\begin{figure}
\centerline{
 \psfig{figure=../07imgrep/fig/bruner.potter.ps,clip= ,width=5.5in}
}
\caption[Bruner and Potter]{
{\em The experimental viewing conditions used by Bruner and Potter in
their recognition experiment.}  One group saw the picture come into
slowly and continuously over a period of 122 seconds.  A second group
saw nothing for 87 seconds and then watched the remainder of the image
come into focus.  The final group only saw the image 109 seconds.
Surprisingly, the group that watched the image come into focus for the
full 122 seconds had the lowest recognition rate (Source: Bruner and
Potter, 1964).
\comment{J.S. Bruner and M. C. Pooter, Interference in VIsual
Recognition,
Science, 1964, v. 144, pp. 424-425.}
}
\label{f7:bruner.potter}
\end{figure}

Seeing these images come into focus slowly made it harder for the
observers to recognize the image contents.  Observers who saw the
images come into focus over a long period of time formulated
hypotheses as to the image contents.  These hypotheses were often
wrong and ultimately interfered with their recognition judgments.

Bruner and Potter illustrated the same phenomenon a different way.
They showed one group of observers the image sequence coming into
focus and a second group the same image sequence going out of focus.
These are the same set of images, shown for the same amount of time.
The difference between the stimuli is the time-reversal.  Subjects who
saw the images come into focus recognized the object correctly $44$
percent of the time.  Subjects who saw the image going out of focus
recognized the object correctly $76$ percent of the time.  Seeing a
low resolution version of an image can interfere with our subsequent
ability to recognize the contents of an image.

Now, don't draw too strong a conclusion from this study about the
problems progressive image enhancement will create.  There are a
number of features of this particular experiment which make the data
quite unlike applications we might plan for progressive image
transmission.  Most importantly, in this study subjects never saw very
clear images.  At the best focus, only half of the subjects recognized
the pictures at all.  Also, the durations over which the images
developed were quite slow, lasting minutes.  These conditions are
sufficiently unlike most planned applications of progressive image
transmission that we cannot be certain the results will apply.  I
mention the result here to emphasize that even after the algorithms
are in place, human testing will remain an important element of the
system design.


%
%	References go here
%
% Get Chuck Stein and other references comparing image quality
% at different resolution levels
% Also, Vetterli, on QMFs
% Mallat, Grossman and Mayer on wavelets
% Bruner and Potter
\nocite{Adelson1987,Burt1983a,Burt1983b}
\nocite{Kersten1987}
\nocite{Mallat1989}
\nocite{Vetterli1984,Vetterli1986}
%
%       Homework Exercises
%
%
\newpage
\section*{Exercises}

\be

\item Answer these questions concerning the relations
between image compression algorithms and visual perception.

 \be

 \item What factors make it possible to achieve large compression
ratios with natural images?

 \item What properties of the human optics are important for
compression?  Do these have any implications for compression of
colored images?

 \item The {\em neuron doctrine} states that a stimulus that generates
a strong neural response defines what the neuron codes, usually called
the neuron's {\em trigger feature}.  In this chapter we have defined
the image feature corresponding to a transform coefficient.  How does
the definition we have used in this chapter compare with the
definition of a feature used in the neuron doctrine?  Explain how the
definition in this chapter could be stated in terms of a physiological
hypothesis.

 \item There are certain properties, such as shadows, surface
curvature, that are common across many different images.  How might
one take advantage of the properties of the image formation process to
design more efficient image compression algorithms?

 \ee

\item  There are two ways in which compression can be useful.
One is for representing the physical image efficiently.  A second
reason to compress image data is to obtain a form that is efficient
for subsequent visual calculations of motion, color and form.  Answer
the following questions relating these two aspects of image
compression.

 \be

 \item Make a hypothesis about where in the visual pathways you would
make a master representation of the input data, and how other cortical
areas might retrieve data from this site.

 \item What properties would be computed by the separate cortical areas?

 \item What properties would define an efficient central
representation of visual information?

 \item How would you test your hypotheses with physiological or
behavioral experiments?

 \ee

\item Answer these questions about the JPEG-DCT image compression
algorithm and image pyramids.

 \be

 \item Describe the JPEG-DCT computation in terms of convolution and
decimation.

 \item What is the main difference between the JPEG-DCT convolution/decimation
process and the image pyramid convolution/decimation process?

 \item  At what stage is information lost in compression algorithms? 

 \item Why is it better to quantize the DCT coefficients rather than
the quantize the individual pixels?

 \ee

\item Here are some specific formulae concerning the DCT.
Suppose that the data in an image block is $p(j,k)$.  The DCT formula
is

\begin{equation}
P(u,v) =  
  \frac {4 c(u) c(v) }{ n ^ 2}
   \sum_{j = 0}^{n-1} 
     \sum_{k=0}^{n-1} p(j,k) 
       \cos ( \frac{(2j + 1) u \pi}{2n} ) \cos ( \frac{ (2k+1)v \pi}{2n} )
\label{e6:dct}
\end{equation}
where $P(u,v)$ are the transform coefficients, and $c$ is a
normalizing function defined as
\begin{equation}
c ( w  ) =  \left 
   \{ \begin{array}{ll} 
     { 1 / \sqrt{2} } & \mbox{if w = 0 } \\
     { 1 } & {\mbox{otherwise}}
    \end{array}
   \right.
\end{equation}

Answer the following questions about the DCT.

 \be

 \item Prove that the inverse to the DCT is
\begin{equation}
p(j,k) = 
   \sum_{u = 0}^{n-1} \sum_{v=0}^{n-1}  c(u) c(v) P(u,v) 
      \cos ( \frac{(2j + 1) u \pi}{2n} ) \cos ( \frac{ (2k+1)v \pi}{2n} )
\label{e6:dct.inv}
\end{equation}
(Notice that, except for a scale factor, this formula is nearly the
same as the forward DCT formula.  Because of this we say the DCT is
{\em self-inverting}.)

 \item The DCT is separable with respect to the rows and columns of
the input image block.  This means that we can express the DCT
calculation as a matrix product $\it {D_r P D_c}$ where the matrix
$\it P$ represents the image data, $p(j,k)$ and the matrices $\it D_r$
and $\it D_c$ represent the DCT transformation.  Create the matrices
$\it D_r$ and $\it D_c$.  How are they related?

 \ee

\item Answer these questions about wavelets and QMFs.

 \be

  \item What advantage(s) does orthogonality confer on the wavelet/QMF
representation?

  \item Separability is a good property since it makes computation
easier.  Circular symmetry is another good feature since we can
represent the value of a function from a single number.  There is only
one function that is both separable and circularly symmetric.  What is
it?  Prove the result.

%(The Gaussian is the only probability density is both
%separable and circularly symmetric.
%Try proving that using functional equations.)
%
%f(a,b) = f(a)f(b) and f(a^2 , b^2) = f(a^2 + b^2 , 0 )
%f(a^2)f(b^2) = f(a^2 + b^2) => f = e^a (Cauchy eqn).
%
 \ee

 \item Wavelets have been in the news in recent years.  Search recent
newspaper articles in the library to see what claims have been made.
Also, look for conference proceedings and companies started based on
the technology.

\ee

\comment{
History lesson:

Working at Princeton, Tanimoto and Pavlidis described the first image
pyramid representation.  The work by Tanimoto and his colleagues (e.g.
Sloan and Tanimoto) extended this work and remained focussed on
discretely valued intensities levels and raster operations.

Peter Burt, working at Maryland and then RCA, developed a slightly
more general notion of image pyramids and at RCA laboratories.  Along
with Adelson, Burt applied pyramid methods to image compression.  A.
B. Watson, working with collaborators at NASA-Ames, have also made
large use of pyramid representations.  Other references are listed at
the end of this chapter.

Faces in particular for progressive image transmission.  A.J. Bilson.
}
