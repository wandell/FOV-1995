% Documentstyle included in preamble.tex
%
\input{/u/brian/lib/tex/standardBook}
\begin{document}

\def\sfrac#1#2{{\textstyle \frac{#1}{#2}}}   
\def\mat#1{{\bf #1}}

%
%	This resides in /usr/share/src/Tex/lib/inputs
%
% Do not make ANY changes to this file. It is read by Latex when
% putting epsi figures into a document.
% 
% Psfig/TeX Release 1.3
% dvips version
%
% All software, documentation, and related files in this distribution of
% psfig/tex are Copyright 1987, 1988 Trevor J. Darrell
%
% Permission is granted for use and non-profit distribution of psfig/tex 
% providing that this notice be clearly maintained, but the right to
% distribute any portion of psfig/tex for profit or as part of any commercial
% product is specifically reserved for the author.
%
% $Header: psfig.tex,v 1.9 88/01/08 17:42:01 trevor Exp $
% $Source: $
%
% Thanks to Greg Hager (GDH) and Ned Batchelder for their contributions
% to this project.
%
% Modified by J. Daniel Smith on 9 October 1990 to accept the
% %%BoundingBox: comment with or without a space after the colon.  Stole
% file reading code from Tom Rokicki's EPSF.TEX file (see below).

%
% from a suggestion by eijkhout@csrd.uiuc.edu to allow
% loading as a style file:
\edef\psfigRestoreAt{\catcode`@=\number\catcode`@\relax}
\catcode`\@=11\relax
\newwrite\@unused
\def\typeout#1{{\let\protect\string\immediate\write\@unused{#1}}}
\typeout{psfig/tex 1.4-dvips}


%% Here's how you define your figure path.  Should be set up with null
%% default and a user useable definition.

\def\figurepath{./}
\def\psfigurepath#1{\edef\figurepath{#1}}

%
% @psdo control structure -- similar to Latex @for.
% I redefined these with different names so that psfig can
% be used with TeX as well as LaTeX, and so that it will not 
% be vunerable to future changes in LaTeX's internal
% control structure,
%
\def\@nnil{\@nil}
\def\@empty{}
\def\@psdonoop#1\@@#2#3{}
\def\@psdo#1:=#2\do#3{\edef\@psdotmp{#2}\ifx\@psdotmp\@empty \else
    \expandafter\@psdoloop#2,\@nil,\@nil\@@#1{#3}\fi}
\def\@psdoloop#1,#2,#3\@@#4#5{\def#4{#1}\ifx #4\@nnil \else
       #5\def#4{#2}\ifx #4\@nnil \else#5\@ipsdoloop #3\@@#4{#5}\fi\fi}
\def\@ipsdoloop#1,#2\@@#3#4{\def#3{#1}\ifx #3\@nnil 
       \let\@nextwhile=\@psdonoop \else
      #4\relax\let\@nextwhile=\@ipsdoloop\fi\@nextwhile#2\@@#3{#4}}
\def\@tpsdo#1:=#2\do#3{\xdef\@psdotmp{#2}\ifx\@psdotmp\@empty \else
    \@tpsdoloop#2\@nil\@nil\@@#1{#3}\fi}
\def\@tpsdoloop#1#2\@@#3#4{\def#3{#1}\ifx #3\@nnil 
       \let\@nextwhile=\@psdonoop \else
      #4\relax\let\@nextwhile=\@tpsdoloop\fi\@nextwhile#2\@@#3{#4}}
% 
%
%%%%%%%%%%%%%%%%%%%%%%%%%%%%%%%%%%%%%%%%%%%%%%%%%%%%%%%%%%%%%%%%%%%
% file reading stuff from epsf.tex
%   EPSF.TEX macro file:
%   Written by Tomas Rokicki of Radical Eye Software, 29 Mar 1989.
%   Revised by Don Knuth, 3 Jan 1990.
%   Revised by Tomas Rokicki to accept bounding boxes with no
%      space after the colon, 18 Jul 1990.
%   Portions modified/removed for use in PSFIG package by
%      J. Daniel Smith, 9 October 1990.
%
\newread\ps@stream
\newif\ifnot@eof       % continue looking for the bounding box?
\newif\if@noisy        % report what you're making?
\newif\if@atend        % %%BoundingBox: has (at end) specification
\newif\if@psfile       % does this look like a PostScript file?
%
% PostScript files should start with `%!'
%
{\catcode`\%=12\global\gdef\epsf@start{%!}}
\def\epsf@PS{PS}
%
\def\epsf@getbb#1{%
%
%   The first thing we need to do is to open the
%   PostScript file, if possible.
%
\openin\ps@stream=#1
\ifeof\ps@stream\typeout{Error, File #1 not found}\else
%
%   Okay, we got it. Now we'll scan lines until we find one that doesn't
%   start with %. We're looking for the bounding box comment.
%
   {\not@eoftrue \chardef\other=12
    \def\do##1{\catcode`##1=\other}\dospecials \catcode`\ =10
    \loop
       \if@psfile
	  \read\ps@stream to \epsf@fileline
       \else{
	  \obeyspaces
          \read\ps@stream to \epsf@tmp\global\let\epsf@fileline\epsf@tmp}
       \fi
       \ifeof\ps@stream\not@eoffalse\else
%
%   Check the first line for `%!'.  Issue a warning message if its not
%   there, since the file might not be a PostScript file.
%
       \if@psfile\else
       \expandafter\epsf@test\epsf@fileline:. \\%
       \fi
%
%   We check to see if the first character is a % sign;
%   if so, we look further and stop only if the line begins with
%   `%%BoundingBox:' and the `(atend)' specification was not found.
%   That is, the only way to stop is when the end of file is reached,
%   or a `%%BoundingBox: llx lly urx ury' line is found.
%
          \expandafter\epsf@aux\epsf@fileline:. \\%
       \fi
   \ifnot@eof\repeat
   }\closein\ps@stream\fi}%
%
% This tests if the file we are reading looks like a PostScript file.
%
\long\def\epsf@test#1#2#3:#4\\{\def\epsf@testit{#1#2}
			\ifx\epsf@testit\epsf@start\else
\typeout{Warning! File does not start with `\epsf@start'.  It may not be a PostScript file.}
			\fi
			\@psfiletrue} % don't test after 1st line
%
%   We still need to define the tricky \epsf@aux macro. This requires
%   a couple of magic constants for comparison purposes.
%
{\catcode`\%=12\global\let\epsf@percent=%\global\def\epsf@bblit{%BoundingBox}}
%
%
%   So we're ready to check for `%BoundingBox:' and to grab the
%   values if they are found.  We continue searching if `(at end)'
%   was found after the `%BoundingBox:'.
%
\long\def\epsf@aux#1#2:#3\\{\ifx#1\epsf@percent
   \def\epsf@testit{#2}\ifx\epsf@testit\epsf@bblit
	\@atendfalse
        \epsf@atend #3 . \\%
	\if@atend	
	   \if@verbose{
		\typeout{psfig: found `(atend)'; continuing search}
	   }\fi
        \else
        \epsf@grab #3 . . . \\%
        \not@eoffalse
        \global\no@bbfalse
        \fi
   \fi\fi}%
%
%   Here we grab the values and stuff them in the appropriate definitions.
%
\def\epsf@grab #1 #2 #3 #4 #5\\{%
   \global\def\epsf@llx{#1}\ifx\epsf@llx\empty
      \epsf@grab #2 #3 #4 #5 .\\\else
   \global\def\epsf@lly{#2}%
   \global\def\epsf@urx{#3}\global\def\epsf@ury{#4}\fi}%
%
% Determine if the stuff following the %%BoundingBox is `(atend)'
% J. Daniel Smith.  Copied from \epsf@grab above.
%
\def\epsf@atendlit{(atend)} 
\def\epsf@atend #1 #2 #3\\{%
   \def\epsf@tmp{#1}\ifx\epsf@tmp\empty
      \epsf@atend #2 #3 .\\\else
   \ifx\epsf@tmp\epsf@atendlit\@atendtrue\fi\fi}


% End of file reading stuff from epsf.tex
%%%%%%%%%%%%%%%%%%%%%%%%%%%%%%%%%%%%%%%%%%%%%%%%%%%%%%%%%%%%%%%%%%%

\def\psdraft{
	\def\@psdraft{0}
	%\typeout{draft level now is \@psdraft \space . }
}
\def\psfull{
	\def\@psdraft{100}
	%\typeout{draft level now is \@psdraft \space . }
}
\psfull
\newif\if@prologfile
\newif\if@postlogfile
\def\pssilent{
	\@noisyfalse
}
\def\psnoisy{
	\@noisytrue
}
\psnoisy
%%% These are for the option list.
%%% A specification of the form a = b maps to calling \@p@@sa{b}
\newif\if@bbllx
\newif\if@bblly
\newif\if@bburx
\newif\if@bbury
\newif\if@height
\newif\if@width
\newif\if@rheight
\newif\if@rwidth
\newif\if@clip
\newif\if@verbose
\def\@p@@sclip#1{\@cliptrue}

%%% GDH 7/26/87 -- changed so that it first looks in the local directory,
%%% then in a specified global directory for the ps file.

\def\@p@@sfile#1{\def\@p@sfile{null}%
	        \openin1=#1
		\ifeof1\closein1%
		       \openin1=\figurepath#1
			\ifeof1\typeout{Error, File #1 not found}
			\else\closein1
			    \edef\@p@sfile{\figurepath#1}%
                        \fi%
		 \else\closein1%
		       \def\@p@sfile{#1}%
		 \fi}
\def\@p@@sfigure#1{\def\@p@sfile{null}%
	        \openin1=#1
		\ifeof1\closein1%
		       \openin1=\figurepath#1
			\ifeof1\typeout{Error, File #1 not found}
			\else\closein1
			    \def\@p@sfile{\figurepath#1}%
                        \fi%
		 \else\closein1%
		       \def\@p@sfile{#1}%
		 \fi}

\def\@p@@sbbllx#1{
		%\typeout{bbllx is #1}
		\@bbllxtrue
		\dimen100=#1
		\edef\@p@sbbllx{\number\dimen100}
}
\def\@p@@sbblly#1{
		%\typeout{bblly is #1}
		\@bbllytrue
		\dimen100=#1
		\edef\@p@sbblly{\number\dimen100}
}
\def\@p@@sbburx#1{
		%\typeout{bburx is #1}
		\@bburxtrue
		\dimen100=#1
		\edef\@p@sbburx{\number\dimen100}
}
\def\@p@@sbbury#1{
		%\typeout{bbury is #1}
		\@bburytrue
		\dimen100=#1
		\edef\@p@sbbury{\number\dimen100}
}
\def\@p@@sheight#1{
		\@heighttrue
		\dimen100=#1
   		\edef\@p@sheight{\number\dimen100}
		%\typeout{Height is \@p@sheight}
}
\def\@p@@swidth#1{
		%\typeout{Width is #1}
		\@widthtrue
		\dimen100=#1
		\edef\@p@swidth{\number\dimen100}
}
\def\@p@@srheight#1{
		%\typeout{Reserved height is #1}
		\@rheighttrue
		\dimen100=#1
		\edef\@p@srheight{\number\dimen100}
}
\def\@p@@srwidth#1{
		%\typeout{Reserved width is #1}
		\@rwidthtrue
		\dimen100=#1
		\edef\@p@srwidth{\number\dimen100}
}
\def\@p@@ssilent#1{ 
		\@verbosefalse
}
\def\@p@@sprolog#1{\@prologfiletrue\def\@prologfileval{#1}}
\def\@p@@spostlog#1{\@postlogfiletrue\def\@postlogfileval{#1}}
\def\@cs@name#1{\csname #1\endcsname}
\def\@setparms#1=#2,{\@cs@name{@p@@s#1}{#2}}
%
% initialize the defaults (size the size of the figure)
%
\def\ps@init@parms{
		\@bbllxfalse \@bbllyfalse
		\@bburxfalse \@bburyfalse
		\@heightfalse \@widthfalse
		\@rheightfalse \@rwidthfalse
		\def\@p@sbbllx{}\def\@p@sbblly{}
		\def\@p@sbburx{}\def\@p@sbbury{}
		\def\@p@sheight{}\def\@p@swidth{}
		\def\@p@srheight{}\def\@p@srwidth{}
		\def\@p@sfile{}
		\def\@p@scost{10}
		\def\@sc{}
		\@prologfilefalse
		\@postlogfilefalse
		\@clipfalse
		\if@noisy
			\@verbosetrue
		\else
			\@verbosefalse
		\fi
}
%
% Go through the options setting things up.
%
\def\parse@ps@parms#1{
	 	\@psdo\@psfiga:=#1\do
		   {\expandafter\@setparms\@psfiga,}}
%
% Compute bb height and width
%
\newif\ifno@bb
\def\bb@missing{
	\if@verbose{
		\typeout{psfig: searching \@p@sfile \space  for bounding box}
	}\fi
	\no@bbtrue
	\epsf@getbb{\@p@sfile}
        \ifno@bb \else \bb@cull\epsf@llx\epsf@lly\epsf@urx\epsf@ury\fi
}	
\def\bb@cull#1#2#3#4{
	\dimen100=#1 bp\edef\@p@sbbllx{\number\dimen100}
	\dimen100=#2 bp\edef\@p@sbblly{\number\dimen100}
	\dimen100=#3 bp\edef\@p@sbburx{\number\dimen100}
	\dimen100=#4 bp\edef\@p@sbbury{\number\dimen100}
	\no@bbfalse
}
\def\compute@bb{
		\no@bbfalse
		\if@bbllx \else \no@bbtrue \fi
		\if@bblly \else \no@bbtrue \fi
		\if@bburx \else \no@bbtrue \fi
		\if@bbury \else \no@bbtrue \fi
		\ifno@bb \bb@missing \fi
		\ifno@bb \typeout{FATAL ERROR: no bb supplied or found}
			\no-bb-error
		\fi
		%
		\count203=\@p@sbburx
		\count204=\@p@sbbury
		\advance\count203 by -\@p@sbbllx
		\advance\count204 by -\@p@sbblly
		\edef\@bbw{\number\count203}
		\edef\@bbh{\number\count204}
		%\typeout{ bbh = \@bbh, bbw = \@bbw }
}
%
% \in@hundreds performs #1 * (#2 / #3) correct to the hundreds,
%	then leaves the result in @result
%
\def\in@hundreds#1#2#3{\count240=#2 \count241=#3
		     \count100=\count240	% 100 is first digit #2/#3
		     \divide\count100 by \count241
		     \count101=\count100
		     \multiply\count101 by \count241
		     \advance\count240 by -\count101
		     \multiply\count240 by 10
		     \count101=\count240	%101 is second digit of #2/#3
		     \divide\count101 by \count241
		     \count102=\count101
		     \multiply\count102 by \count241
		     \advance\count240 by -\count102
		     \multiply\count240 by 10
		     \count102=\count240	% 102 is the third digit
		     \divide\count102 by \count241
		     \count200=#1\count205=0
		     \count201=\count200
			\multiply\count201 by \count100
		 	\advance\count205 by \count201
		     \count201=\count200
			\divide\count201 by 10
			\multiply\count201 by \count101
			\advance\count205 by \count201
			%
		     \count201=\count200
			\divide\count201 by 100
			\multiply\count201 by \count102
			\advance\count205 by \count201
			%
		     \edef\@result{\number\count205}
}
\def\compute@wfromh{
		% computing : width = height * (bbw / bbh)
		\in@hundreds{\@p@sheight}{\@bbw}{\@bbh}
		%\typeout{ \@p@sheight * \@bbw / \@bbh, = \@result }
		\edef\@p@swidth{\@result}
		%\typeout{w from h: width is \@p@swidth}
}
\def\compute@hfromw{
		% computing : height = width * (bbh / bbw)
		\in@hundreds{\@p@swidth}{\@bbh}{\@bbw}
		%\typeout{ \@p@swidth * \@bbh / \@bbw = \@result }
		\edef\@p@sheight{\@result}
		%\typeout{h from w : height is \@p@sheight}
}
\def\compute@handw{
		\if@height 
			\if@width
			\else
				\compute@wfromh
			\fi
		\else 
			\if@width
				\compute@hfromw
			\else
				\edef\@p@sheight{\@bbh}
				\edef\@p@swidth{\@bbw}
			\fi
		\fi
}
\def\compute@resv{
		\if@rheight \else \edef\@p@srheight{\@p@sheight} \fi
		\if@rwidth \else \edef\@p@srwidth{\@p@swidth} \fi
}
%		
% Compute any missing values
\def\compute@sizes{
	\compute@bb
	\compute@handw
	\compute@resv
}
%
% \psfig
% usage : \psfig{file=, height=, width=, bbllx=, bblly=, bburx=, bbury=,
%			rheight=, rwidth=, clip=}
%
% "clip=" is a switch and takes no value, but the `=' must be present.
\def\psfig#1{\vbox {
	% do a zero width hard space so that a single
	% \psfig in a centering enviornment will behave nicely
	%{\setbox0=\hbox{\ }\ \hskip-\wd0}
	%
	\ps@init@parms
	\parse@ps@parms{#1}
	\compute@sizes
	%
	\ifnum\@p@scost<\@psdraft{
		\if@verbose{
			\typeout{psfig: including \@p@sfile \space }
		}\fi
		%
		\special{ps::[begin] 	\@p@swidth \space \@p@sheight \space
				\@p@sbbllx \space \@p@sbblly \space
				\@p@sbburx \space \@p@sbbury \space
				startTexFig \space }
		\if@clip{
			\if@verbose{
				\typeout{(clip)}
			}\fi
			\special{ps:: doclip \space }
		}\fi
		\if@prologfile
		    \special{ps: plotfile \@prologfileval \space } \fi
		\special{ps: plotfile \@p@sfile \space }
		\if@postlogfile
		    \special{ps: plotfile \@postlogfileval \space } \fi
		\special{ps::[end] endTexFig \space }
		% Create the vbox to reserve the space for the figure
		\vbox to \@p@srheight true sp{
			\hbox to \@p@srwidth true sp{
				\hss
			}
		\vss
		}
	}\else{
		% draft figure, just reserve the space and print the
		% path name.
		\vbox to \@p@srheight true sp{
		\vss
			\hbox to \@p@srwidth true sp{
				\hss
				\if@verbose{
					\@p@sfile
				}\fi
				\hss
			}
		\vss
		}
	}\fi
}}
\def\psglobal{\typeout{psfig: PSGLOBAL is OBSOLETE; use psprint -m instead}}
\psfigRestoreAt








% \newcommand{\psfigpath}{/home/brian/images/general}
%
% 	I think this sets it up for draft mode rather than actually
% 	including the stuff
%\psdraft


%
%	All the math includes
%
%	From Chapter on Image Formation
%
%

%%%%%%%%%%%%%%%%%%%%%%%%%%%%%%%%
%
%
%	General vector and matrix notation
%
\newcommand{\vect}[1]{\mbox{${\bf #1}$}}
\newcommand{\vecti}[2]{\mbox{${#1}_{#2}$}}
\newcommand{\vectij}[3]{\mbox{${#1}_{#2,#3}$}}
\newcommand{\matr}[1]{\mbox{${\bf #1}$}}

\newcommand{\lin}{\mbox{{ \boldmath $L$ } }}
\newcommand{\twopN}{\mbox{${\frac{2 pi}{N}}$}}
\newcommand{\sumoneNn} {\sum_{n = 1}^{N}}
\newcommand{\sumoneNi} {\sum_{i = 1}^{N}}
\newcommand{\sumoneNj} {\sum_{j = 1}^{N}}
\newcommand{\sumoneNk} {\sum_{k = 1}^{N}}
\newcommand{\sumoneNf} {\sum_{f = 1}^{N}}
\newcommand{\sumoneNfx}{\sum_{fx = 1}^{N}}
\newcommand{\sumoneNfy}{\sum_{fy = 1}^{N}}
\newcommand{\sumzeroNn}{\sum_{n = 0}^{N - 1}}
\newcommand{\sumzeroNi}{\sum_{i = 0}^{N - 1}}
\newcommand{\sumzeroNj}{\sum_{j = 0}^{N - 1}}
\newcommand{\sumzeroNk}{{\sum_{k = 0}^{N - 1}}}
\newcommand{\sumzeroNf}{\sum_{f = 0}^{N - 1}}
\newcommand{\sumzeroNfx}{\sum_{fx = 0}^{N - 1}}
\newcommand{\sumzeroNfy}{\sum_{fy = 0}^{N - 1}}
\newcommand{\sumoverNi} {\sum_{i = \langle N \rangle}}
\newcommand{\sumoverNj} {\sum_{j = \langle N \rangle}}
\newcommand{\sumoverNk} {\sum_{k = \langle N \rangle}}

\newcommand{\Dzero}{\mbox{$(1,0,0, \ldots ,0)^t$}}
\newcommand{\Done}{\mbox{$(0,1,0, \ldots ,0)^t$}}
\newcommand{\DN}{\mbox{$(0,0,...,0,1)^t$}}
\newcommand{\Di}{\mbox{$(0,..,1,...0)^t$}}

%	Starting with Chapter on Optics
%
\newcommand{\lspread}{\mbox {$\bf l$} }
\newcommand{\lspreadi}[1]{\mbox{$l_{#1}$}}
\newcommand{\lspreadhat}{\mbox {$\hat{\bf l}$}}
\newcommand{\lspreadhati}[1]{\mbox {$\hat{l}_{#1}$}}
\newcommand{\Resmat}{\mbox {$\bf R$} }
\newcommand{\resvec}{\mbox {$\bf r$} }
\newcommand{\resveci}[1]{\mbox{$r_{#1}$}}
\newcommand{\resvechat}{\mbox {\bf $\hat{r}$}}
\newcommand{\resvechati}[1]{\mbox {$\hat{r}_{#1}$}}
\newcommand{\pixmat}{\mbox {$\bf P$} }
\newcommand{\pixvec}{\mbox {$\bf p$} }
\newcommand{\pixveci}[1]{\mbox{$p_{#1}$}}
\newcommand{\pixvechat}{\mbox {$\bf \hat{p}$}}
\newcommand{\pixvechati}[1]{\mbox{$\hat{p}_{#1}$}}
\newcommand{\Optics}{\mbox{$\bf O$}}
\newcommand{\pfiN}{\mbox{$\frac {2 \pi f i}{ N }$}}
\newcommand{\SfNi}{\mbox{$\sin  ( \frac {2 \pi f i}{ N } )$}}
\newcommand{\SfnNi}{\mbox{$\sin  ( - \frac {2 \pi f i}{ N } )$}}
\newcommand{\SfiNk}{\mbox{$\sin ( \frac {2 \pi f_i k}{ N } )$}}
\newcommand{\SfjNk}{\mbox{$\sin ( \frac{2 \pi f_j k}{ N } )$}}
\newcommand{\CfNi}{\mbox{$\cos (  \frac{2 \pi f i}{ N } )$}}
\newcommand{\CfiNk}{\mbox{$\cos ( \frac{2 \pi f_i k}{ N })$}}
\newcommand{\CfjNk}{\mbox{$\cos ( \frac{2 \pi f_j k}{ N } )$}}
\newcommand{\SfNj}{\mbox{$\sin (  \frac{2 \pi f j}{ N } )$}}
\newcommand{\CfNj}{\mbox{$\cos (  \frac{2 \pi f j}{ N })$}}
\newcommand{\SfNk}{\mbox{$\sin (  \frac{2 \pi f k}{ N })$}}
\newcommand{\SfnNk}{\mbox{$\sin ( - \frac{2 \pi f k}{ N })$}}
\newcommand{\CfNk}{\mbox{$\cos (  \frac{2 \pi f k}{ N })$}}
\newcommand{\SfNij}{\mbox{$\sin ( \frac{2 \pi f (i + j)}{ N } )$}}
\newcommand{\SfoneNk}{\mbox{$\sin ( \frac{2 \pi f_1 k}{ N } )$}}
\newcommand{\SftwoNk}{\mbox{$\sin ( \frac{2 \pi f_2 k}{ N } )$}}
\newcommand{\CftwoNk}{\mbox{$\cos ( \frac{2 \pi f_2 k}{ N } )$}}
\newcommand{\matCos}{\mbox{$ \bf C$}}
\newcommand{\matSin}{\mbox{$ \bf S$}}

%	From chapter on color
%

\newcommand{\Red}{\mbox{L}}
\newcommand{\Green}{\mbox{M}}
\newcommand{\Blue}{\mbox{S}}
\newcommand{\redw}{\Red ( \lambda )}
\newcommand{\greenw}{\Green ( \lambda )}
\newcommand{\bluew}{\Blue ( \lambda )}
\newcommand{\evec}{\mbox{$\bf e$}}
\newcommand{\eveci}[1]{\mbox{$e_{#1}$}}
\newcommand{\er}{e_r}
\newcommand{\eg}{e_g}
\newcommand{\eb}{e_b}
\newcommand{\test}{\mbox{${\bf t}$}}
\newcommand{\testi}[1]{\mbox{$t_{#1}$}}
\newcommand{\testw}{t ( \lambda )}
\newcommand{\lms}{\mbox{$\bf r$}}
\newcommand{\Iodopsin}{\mbox{$\bf B$}}
\newcommand{\Rhodopsin}{\mbox{$\bf A$}}
\newcommand{\Rhodopsinw}{\mbox{$\bf A(\lambda)$}}
\newcommand{\Scotopic}{\mbox{$\bf R$}}
\newcommand{\Scotopicw}{\mbox{$R ( \lambda )$}}
\newcommand{\Scotopici}[1]{R_{#1}}
\newcommand{\primaryint}{\mbox{$\bf e$}}
\newcommand{\primaryinti}[1]{\mbox{$e_{#1}$}}
\newcommand{\primary}{\mbox{$\bf p$}}
\newcommand{\primaryi}[1]{\mbox{${\bf p}_{#1}$}}
\newcommand{\Primarymat}{\mbox{$\bf P$}}
\newcommand{\Photopic}{\mbox{\bf C}}
\newcommand{\Monitor}{\mbox{$ \bf M $}}
\newcommand{\Calibration}{\mbox{$\matr{H}$}}
\newcommand{\monitori}[1]{\mbox{$\bf m_{#1}$}}
\newcommand{\Sensor}{\mbox{\bf S}}
\newcommand{\current}{\mbox{\bf i}}
\newcommand{\currentt}{\mbox{$i ( t )$} }
\newcommand{\nl}{\mbox{$n_\lambda$}}

%
%	From chapter  on physiology
%
\newcommand{\Pbeta}{P_\beta}
\newcommand{ \amp} { Amplitude }
\newcommand{ \sumxx} { \sum_{x = 0} to {x = N} }
\newcommand{ \sumtt} { \sum_{t = 0} to {t = N} }
\newcommand{ \Aftx} { A ( ft , x )}
\newcommand{ \BftxC} { {  W ( ft , C(x) ) } }
\newcommand{ \dt} { { delta t } }
\newcommand{ \fx} {f_{x}}
\newcommand{ \ft} {f_{t}}
\newcommand{ \phix} {p_{x} }
\newcommand{ \phit} {p_{t} }
\newcommand{ \contx} { a_{x} }
\newcommand{ \contt} { a_{t} }
\newcommand{ \intensityi}[1]{\mbox{$i_{#1}$}}
\newcommand{\mean}{\mbox{$m$}}
\newcommand{\pspread}{\mbox {$\bf p$} }
\newcommand{\pspreadi}[1]{\mbox{$p_{#1}$}}
\newcommand{\pspreadhat}{\mbox {\bf $\hat{p}$}}
\newcommand{\pspreadhati}[1]{\mbox {$\hat{p}_{#1}$}}
\newcommand{ \RF} {\mbox{ \bf R}}	%Receptive field
\newcommand{ \linex} {\delta(x)}
\newcommand{ \responsemat} {\bf F}
\newcommand{ \response} {\bf f}
\newcommand{ \responsei} {\mbox{$f_{i}$} }
\newcommand{ \responset} {f ( t ) }
\newcommand{ \contrast}{\mbox{$a$}}
\newcommand{ \contrasti}[1]{\mbox{$a_{#1}$}}
\newcommand{ \contrastvec}{\mbox{$\bf a$}}
\newcommand{ \contrastmat}{\mbox{$\bf A$}}
\newcommand{ \onecontrastmat} {\bf U}
\newcommand{ \LW} {\bf L}
\newcommand{ \As} {A_{s}}
\newcommand{ \Bs} {B_{s}}
\newcommand{ \RFs} {RF_{s}}
\newcommand{ \Ac} {A_{c}}
\newcommand{ \deltai} {(0,..,1,...0)^{t}}

%
%	From chapter on Pattern Vision
%
\newcommand{\svec}{\mbox{$\bf s$}}
\newcommand{\wvec} { \mbox{$\bf w$}}
\newcommand{\stim} { \mbox{$\bf s$} }
\newcommand{\stimi}[1]{ \mbox{$\bf s{_#1}$} }
\newcommand{\neural}{\mbox{$\bf n$}}		%Neural image vector
\newcommand{\neuralx}[1]{\mbox{$n_{#1}$}}
\newcommand{\length}{\mbox{$d$}}
\newcommand{\lengthi}[1]{\mbox{$d_{#1}$}}
\newcommand{\sinlengthi}[1]{\mbox{$a_{#1}$}}

%
%	From chapter on multiresolution

\newcommand{\gest}[1]{\mbox{$\hat{g}_{#1}$}}

%	From the chapter on Color Appearance
%
\newcommand{\recSens}[1]{\mbox{$R_{#1}$}}
\newcommand{\recResp}{\mbox{$\bf r$}}
\newcommand{\recRespi}[1]{\mbox{$r_{#1}$}}
\newcommand{\sensorMat}{\mbox{$\bf S$}}

\newcommand{\colsig}[1]{\mbox{$c(#1)$}}
\newcommand{\colsigi}[1]{\mbox{$c_{#1}$}}

\newcommand{\ill}[1]{\mbox{$e(#1)$}}
\newcommand{\illhat}[1]{\mbox{$\hat{e}(#1)$}}
\newcommand{\illi}[2]{\mbox{$e_{#1}(#2)$}}
\newcommand{\illvec}{\mbox{$\bf e$}}

\newcommand{\illveci}[1]{\mbox{$\bf e_{#1}$}}
\newcommand{\illBasis}{\mbox{$\bf B_e$}}
\newcommand{\illBasisi}[2]{\mbox{$E_{#1}(#2)$}}
\newcommand{\illCoef}{\mbox{${\bf \omega}$}}
\newcommand{\illCoefi}[1]{\mbox{$\omega_{#1}$}}
\newcommand{\illMat}[1]{\mbox{$\Lambda_{#1}$}}
\newcommand{\illMatinv}[1]{\mbox{$\Lambda^{-1}_{#1}$}}

\newcommand{\surf}[1]{\mbox{$s(#1)$}}
\newcommand{\surfvec}{\mbox{$\bf s$}}
\newcommand{\surfveci}[1]{\mbox{$s_{#1}$}}
\newcommand{\surCoef}{\mbox{${\bf \sigma}$}}
\newcommand{\surCoefi}[1]{\mbox{$\sigma_{#1}$}}
\newcommand{\surBasis}{\mbox{$\bf B_s$}}
\newcommand{\surBasisi}[2]{\mbox{$S_{#1}(#2)$}}


%	Motion chapter
%
\newcommand{\parder}[2]{\mbox{$\frac{\partial #1}{\partial #2}$}}

%	Appendix A:  Shift-invariance
%
\newcommand{\Cyclic}{\mbox{$\bf C$}}
\newcommand{\Cyclichat}{\mbox{ $\hat{{\bf C}}$}}
\newcommand{\Cyclichati}[1]{\mbox{ $\hat{{C_{#1}}}$}}

%	Appendix B:  Color calibration
%

%	Appendix C:  Classification
%
\newcommand{\data}{ \mbox{${\bf d}$}}
\newcommand{\stimA}{\mbox{$A$}}
\newcommand{\stimB}{\mbox{$B$}}
\newcommand{\muAi}{\mbox{$\mu_{A,i}$}}
\newcommand{\muBi}{\mbox{$\mu_{B,i}$}}


%
%	From chapter on Tools
%
\newcommand{\dvec}{ \bf d }
\newcommand{\sd}{\sigma}
\newcommand{\muA}{\mu_{A} }
\newcommand{\sdA}{\sigma_{A} }
\newcommand{\muB}{\mu_{B}}
\newcommand{\sdB}{\sigma_{B}}
\newcommand{\mui}{\mu_{i} }
\newcommand{\sdi}{\sigma_{i} }

%	Appendix D:  Signal Estimation
%

%	Appendix E:  Motion flow
%
\newcommand{\motFlow}[2]{\mbox{${\bf m}(#1,#2)$}}

\begin{document}
\doublesp

\section*{Proofs, page ??? Scotopic.mat Figure???}

The system matrix, call it $\Scotopic$, must have one row and $\nl$
columns.  The test light, system matrix, and primary intensity are
related by the product, $e = \Scotopic \test$.  We can write this
matrix equation using a {\bf matrix tableau} representation.  This
representation indicates the general shapes of the vectors and matrices
and is often a helpful method for understanding their shapes and
inter-relationships.  The matrix tableau that represents the scotopic
system matrix is:
\begin{equation}
  \left ( e \right ) =
  \left ( 
%    \mbox{``Scotopic matching system matrix''}
     r_1 ~~~ r_2 ~~~ \ldots ~~~ r_{{\nl}-1}~~~ r_{\nl}
  \right )
  \left (
   \begin{array}{c}
    \testi{1} \\
    \testi{2} \\
    \vdots \\
    \testi{\nl-1} \\
    \testi{\nl} \\
   \end{array}
  \right )
\end{equation}

We can also write the matrix product $\Scotopic \test$
as a summation over the sample wavelengths,
\begin{equation}
\label{e2:sm}
e = \sum_{i=1}^{i=\nl} \Scotopici{i} \testi{i} 
\end{equation}

We can relate the measurements in the scotopic matching experiment to
the entries of the system matrix as follows.  Suppose we use a
monochromatic test light of unit intensity,

\end{document}
