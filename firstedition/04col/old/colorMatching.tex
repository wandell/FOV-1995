
\section{The Color-Matching Experiment}

The fundamental experiment of human color vision, the experiment upon
which the rest of color science is founded, is the 
{\em color-matching} experiment.
This experiment provides the 
scientific basis for the design of television monitors.

%7.5 high 6.0 wide
\begin{figure}
\centerline {
\psfig{figure=../03col/fig/colorMatching.ps,clip=,height=3.0in}
}
\caption[Color Matching Experiment] {
The color-matching experiment.
The observer views a bipartite field and
adjust three primary lights to match
a test light.
(After Judd and Wyszecki, 1975).
%Figure 1.12 page 42, Color in Business, Science, and Industry
}
\label{f3:colormatch}
\end{figure}
The spatial arrangement of the stimuli used
in the color-matching experiment is illustrated in Figure \ref{f3:colormatch}.
The observer views two adjacent visual fields, typically
arranged so that from the observer's point of view these appear
as the right and left halves of a circle.
On one side, called the {\em test field},
the observer is presented with a test light.
The test light is the input to the experiment.
The test light may
consist of any spectral power distribution, $\testw$,
which we represent by the column vector $\test$.
\nocite{JuddWyszecki}

The other side of the bipartite field is called the 
{\em matching field}.
This matching field consists
of the mixture of a small number
of {\em primary lights}.
Just as in the design of a monitor, the relative spectral
power distribution of each primary light remains constant
throughout the experiment.
We can use the matrix tableau description in Figure \ref{f3:monitor.mat}
of monitor spectral power distributions 
to describe the spectral power distributions possible 
by mixing the primary lights.

The observer's task in the color-matching experiment is to adjust
the intensity of the primary lights
so that the two sides of the
bipartite field appear identical.
The smallest number of primary lights needed to assure
that a match can be obtained
depends upon the system being studied.
For rod vision, a single primary light suffices.
For cone vision, three primary lights are required.
The number of primaries required is determined by experiment.

In the color-matching experiment
observers make the two sides appear identical.
But the two sides are very different physically.
Since the light on the primary side of
the bipartite field
is the weighted sum of the primary lights,
and the light on the test side of the
bipartite field can be any light at all,
in general the two sides of the field are not
even close to being physical matches.

\begin{figure}
\centerline {
\psfig{figure=../03col/fig/tvMetamers.ps,clip=,height=3.5in}
}
\caption[Monitor Metamers]{
Metameric lights.
Two lights with these spectral power distributions
appear identical to most observers.
The curve on the left is an approximation
to the spectral power distribution of the sun.
The curve on the right
is the spectral power distribution of 
a light emitted from a monitor
set to the match appearance.
}
\label{f3:metamers}
\end{figure}
Pairs of lights that are visual
matches are called {\em metamers}.
Figure \ref{f3:metamers}
contains a pair of spectral power distributions
that match visually but differ physically, i.e.
a pair of metamers.
As the comparison in Figure \ref{f3:metamers} illustrates,
our wavelength encoding of the retinal image fails to discriminate
between certain
enormous differences in light spectral power distributions.
Yet, as we shall see later, we are capable of making
certain very fine wavelength discriminations.

The color-matching experiment defines
a mapping, as shown in Figure \ref{f3:matching.map}.
%height = 4.5in width = 6.0in
\begin{figure}
\centerline {
\psfig{figure=../03col/fig/matching.map.ps,clip=,height=2.25in}
}
\caption[Color Matching Mapping]{
The color-matching experiment defines a
mapping from the (high-dimensional) test spectral power distribution
to the three-dimensional vector of primary lights.
}
\label{f3:matching.map}
\end{figure}
The input to the color-matching experiment is the
spectral power distribution of the test light.
The input always consists
of a vector that defines the test spectral power distribution.
The output of the experiment is the vector
of primary intensity settings.
If we perform the color-matching experiment at very
low intensity levels, where a single primary light suffices,
then the output is a single real number.
If we perform the experiment at higher luminances,
where three primaries are required,
the output is a three dimensional vector containing
the three primary intensities.

