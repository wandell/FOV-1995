
\newpage
\section*{Exercises}

\be  %All Questions

\item Answer the following 
questions related to image properties on the retina.

 \be
 \item  Use a diagram to explain why the retinal image does not change
size when the pupil changes size

 \item Compute the visual angle swept out by a building that is
200 meters tall seen from a distance of 400 meters.

 \item  Suppose a lens has a focal length of 100mm
Where will the image plane of a line one meter from the
center of the lens be?
Suppose the line is 5 mm high.
Using a picture, show the size of the image.

 \item Use the lensmaker's equation
(from Chapter~\ref{chapter:imageformation}) to calculate
the actual height on the retina.

\comment{Lensmaker's equation:
1/F = 1/d0 + 1/d1
where F = focal length, d0 = distance to source, d1 = distance to image
1/100 = 1/1000 + 1/x
x = 1/9 meters, .111 m
Size of image is scaled to 1/9th of original size
from geometry drawing, so 5/9mm.
}
 \item Good quality printers generate output with 600 dots per inch.
How many dots is that per degree of visual angle.
(Assume that the usual reading distance is 12 inches.)
\comment{
4.76 deg/inch
125.96 pixels/deg
}

 \item Good quality monitors have approximately 1000 pixels on
a single line.  How many pixels is that per degree of visual angle.
(Assume that the usual monitor distance is 0.4 meters and the
width of a line is 0.2 meters.)
\comment{
angle = 26.5 deg.
1000/26.5 = 37.7
}
  \item Some monitors can only turn individual pixels on or off.
It may be fair to compare such monitors with the printed page
since most black and white printers can only place a dot or not
place one at each location.
But, it is not fair to compare printer output
with monitors capable of generating different gray scale levels.
Explain how gray scale levels can improve the accuracy of
reproduction without increasing the number of pixels.
Justify your answer using a matrix-tableau argument.
\comment{
The dimensions of the bitmap exceed the dimensions of
the screen with gray levels.  But, the solutions in the
bitmap case all must have 0 or 1 values.
}
 \ee

\begin{figure}
\centerline{
\psfig{figure=../03mosaic/fig/bluePhosphors.ps ,clip= ,height=2.0in}
}
\caption{
Choosing monitor phosphors.
}
\label{f2:bluePhosphors}
\end{figure}
\item A manufacturer is choosing between two different blue
phosphors in a display (B1 or B2).
The relative energy at different wavelengths of the
two phosphors are shown in Figure~\ref{f2:bluePhosphors}.
Ordinarily, users will be in focus for the red and green
phosphors (not shown in the graph) around 580 nm.

 \be

 \item Based on chromatic aberration, 
which of the two blue
phosphors will yield a sharper retinal image? Why?

 \item If the peak phosphor values are 400 nm and 450 nm,
what will be the highest spatial frequency imaged on the retina
by each of the two phosphors?
(Use the curves in Figure \ref{f1:otf.aberration}.)

 \item Given the highest frequency imaged at 450 nm, what is the
Nyquist sampling rate required to estimate the blue phosphor image?
What is the Nyquist sampling rate for a 400 nm light source?

 \item The eye's optics images
light at wavelengths above 500 nm much better than wavelengths
below that level.
Using the curves in Figure \ref{f2:rec.spec.sens},
explain whether you think the $\Blue$ cones 
will have a problem due to aliasing those longer wavelengths.

 \item (Challenge).  Suppose the eye is always in focus
for 580 nm light.
The quality of the image created by the blue phosphor will
always be quite poor.
Describe how you can design a new layout for the blue phosphor
mosaic on the screen to take advantage of the poor short-wavelength
resolution of the eye.
Remember, you only need to match images after optical defocus.
 \ee

\item Reason from physiology to behavior and back
to answer the following questions.

 \be

 \item Based purely on the physiological evidence from procion
yellow stains, is there any reason to believe that
the cones in Figure \ref{f2:blueConeMosaic} are the $\Blue$ cones?

 \item What evidence do we have that the measurements
of Williams et al. are due to the positions of the $\Blue$ cones 
rather than from the spacing of neural units in the visual pathways 
that are sensitivie to short-wavelength light?

 \ee

\comment{
Taken from cvis.4.tex interferometry...
The discussion in Jenkins and White is a good
and simple description of how to calculate the
intensity pattern of the interference fringe.
See pages 265 et seq.
A form of their figure 13F might make it into the problems section.

They derive from the assumptions that
that the amplitude waveform is $4a^2 \cos^2( \delta / 2 )$ where
$\delta$ is the phase difference between the two signals.
When
(a) the signals are far from the screen compared to
one another, and 
(b) the wavelength of the light is small
compared to the distance from the screen, 
then $\delta$ is
approximately proportional to the distance along the screen image.
In that case we can derive the intensity pattern
as a function of distance along the screen
using (CRC Handbook, page 137, double-angle relations)

\begin{eqnarray}
2 \cos^2( x ) = 2cos^2( x ) - ( cos^2(x) + sin^2(x) ) + 1 \nonumber \\
& = & cos^2(x) - sin^2(x) + 1 \nonumber \\
& = & cos( { x } + { x } ) + 1 \nonumber \\
& = & cos ( 2 x ) + 1 
\end{eqnarray}

Substituting this formula
the amplitude distribution is equal to $2 a^2 cos ( \delta ) + 1 $,
which is a cosinusoid modulated around a constant term.
This derivation should probably be in the problems section.
}

\item  Give a drawing or an explanation to each of the
following questions on aliasing.

 \be
 \item Draw an example of aliasing for a set of sampling
points that are evenly spaced, but do not use a sinusoidal
input pattern.

 \item Consider the sensor sample positions in Figure \ref{f2:hwSample}.
with the positions unevenly spaced, as shown.
Draw the response of this system to a constant valued input signal.
\begin{figure}
\centerline{
\psfig{figure=../03mosaic/fig/samplePoints.ps,clip= ,height=1.0in}
}
\caption[Homework Problem:  Sensor sample positions]{
Sample positions of a set of sensors.
}
\label{f2:hwSample}
\end{figure}

 \item  Now, draw a picture of a stimulus that is non-uniform
and that yields the same response as in the previous question.

 \item  What rule do you use to make sure the
stimuli yield equivalent responses?

 \item  Suppose that we put a lens that strongly
defocuses the stimuli prior to their arrival
at the sensor positions.
This defocus means that it will be impossible to generate
patterns that vary rapidly across space.
If this blur is introduced into
the optical path, will you be able to deliver your stimulus
to the sensor array?
Explain.

 \item Suppose that somebody asks you to invest in a
company.  The main product is a convolution operation
that is applied to the output of a digital discrete sensor array
built into a still camera.
The purpose of the filter is to eliminate aliasing due to
the sensors spatial sampling.
How much would you be willing to invest in the company?

 \ee

\item Perform the following aliasing calculations
\label{q:aliasing}
 \be

 \item In this chapter I asserted that
$\cos (2 \pi ( { N / 2}  + f) / N) = \cos (2 \pi ( { N / 2 } - f ) / N)$.
Multiply out the arguments of the
functions and write them both in the form of $\cos ( i + j )$.
\comment{
\cos ( { \pi } + { 2 \pi f / N } )
and
\cos ( { \pi } - { 2 \pi f / N } )
}

 \item  Use the trigonometric identity 
$\cos (i - j) = \cos(i) \cos(j) + \sin(i) \sin (j)$
to expand the two functions.

\comment{
ANSWER:
\cos( \pi ) \cos ( 2 \pi f / N ) + \sin ( \pi ) \sin( 2 \pi f / N )
\cos( \pi ) \cos ( - 2 \pi f / N ) + \sin ( \pi ) \sin( - 2 \pi f / N )
}

 \item What is the value of $\sin ( \pi )$?
What is the value of $\cos( \pi )$?
Use these values to obtain the final equality.
\comment{
ANSWER:
$\sin(pi) = 0$ and $\cos(pi) = 1.0$, so the result is simply
$\cos(2 \pi f / N )$ or $cos(- 2 \pi f / N )$ which 
are the same because the cosine function is even.
}

 \item Suppose that we represent a signal using a vector with
ten entries.  Suppose the signal is sampled at
five locations, and we describe the sampling operation
using a sampling matrix consisting of zeros and ones.
How many rows and columns would the sampling matrix have?

 \item Write out the sampling matrix for a one-dimensional
sampling pattern whose sample positions are at 1,3,5,7,9.

 \item Write out the sampling matrix for a non-uniform, one-dimensional
pattern in which the sample positions are spaced at locations
1,2,4,and 8.

 \ee

\item Answer each of the following questions
about the relationship between the
sampling mosaic and optics of the eye.

 \be

 \item From time to time, some investigators 
have thought that the long-wavelength
photopigment peak was near 620 nm, not 580 nm.
Using Figure \ref{f1:otf.aberration},
discuss what implication such a peak wavelength would
have for the Nyquist sampling rate
required of these receptors.

 \item In fact, as you can see from Figure \ref{f2:rec.spec.sens},
the $\Green$ and $\Red$ cones
both have peak sensitivities in the range near 550 nm to 580 nm.
What is required of their spacing in order to accurately
capture the retinal image?

 \item We have been assuming that the sensors in our array are
equally sensitive to the incoming signal.
Suppose that we have a sensor array that
consists of alternating $\Blue$ and $\Red$ cones.
Draw the response of this array to an uniform field
consiting of 450 nm light.
Now, draw the intensity pattern
that would have the same affect
when the light is 650 nm.

 \ee

\item Here are two Postscript programs, written by Arturo Puente,
to create squarewave patterns and sampling patterns.
Use the programs to print out the grids and patterns,
and then copy the printouts onto an overhead transparency.
View the patterns through the grids to see the effects
of aliasing.

\begin{verbatim}
%!PS-Adobe-1.0
% Description:
% Parameters:
%
/widthx1	10 def	% <- Value to change
/widthx2	 2 def	% <- Value to change

/x1		90 def
/y1		180 def
/x2		{ x1 widthx1 add } def
/y2		612 def
/numx		{432 widthx1 widthx2 add idiv} def

/form
{
 newpath
 x1 y1 moveto
 x1 y2 lineto
 x2 y2 lineto
 x2 y1 lineto
 closepath
 fill
/x1 x1 widthx1 widthx2 add add def
} def

numx {form} repeat

/Helvetica findfont 10 scalefont setfont
72 84 moveto
(Width solid lines =) show
widthx1 dup 3 string cvs show
72 72 moveto
(Width white lines =) show
widthx2 dup 3 string cvs show
showpage

%!PS-Adobe-1.0
% Description:
% Parameters:
%
/widthx		20 def	% <- Value to change
/widthy		10 def  % <- Value to change
/wall		1 def  % <- Value to change

/x1		90 def
/y1		180 def
/x2		{ x1 widthx add } def
/y2		{ y1 widthx add } def
/numx		{432 widthx idiv} def
/numy		{432 widthy idiv} def

/form {
 newpath
 x1 y1 moveto
 x2 y1 lineto
 x2 y2 lineto
 x1 y2 lineto
 x1 y1 lineto
 x1 wall add y1 wall add lineto
 x1 wall add y2 wall sub lineto
 x2 wall sub y2 wall sub lineto
 x2 wall sub y1 wall add lineto
 x1 wall add y1 wall add lineto
 closepath
 fill
/x1 x1 widthx add def
} def

/fileform
{
 numx {form} repeat
 /x1	90 def
 /y1 y1 widthy add def
} def

numy {fileform} repeat


%
%	Write out parameters
%
/Helvetica findfont 10 scalefont setfont
72 96 moveto
(Width of the rectangle in the x-axis = ) show
widthx dup 3 string cvs show
72 84 moveto
(Width of the rectangle in the y-axis = ) show
widthy dup 3 string cvs show
72 72 moveto
(Thickness of the wall = ) show
wall dup 3 string cvs show

showpage
%
\end{verbatim}



\ee  %All Questions
