These beautiful anatomical images of the cone mosaic
have only been acquired recently;
extracting a retina and fixing it without shrinkage
or damage is a difficult technical feat.
For nearly a hundred years before the biological measurements were possible,
a collection of behavioral studies were carried
out to estimate
the arrangement of the photoreceptors.
These behavioral studies are
still important for two reasons.
First, we cannot be sure that the spacing and regularity
of the photoreceptors in the living eye is the same as the
spacing and regularity in this prepared specimen.
The size and delicacy of the photoreceptors has made
the direct measurement of their positions in the retina
a challenging empirical problem.
The process of removing retinae from the eye and and maintaining
the delicate and precise arrangement of the photoreceptors is
a precarious art form.
Thus, to confirm the accuracy of these biological measurements,
we would like to estimate properties of the photoreceptor mosaic
while it remains functioning in the living eye.

A second significant aspect of these behavioral estimates
of the behavioral studies of cone sampling
is that they succesfully predicted the
biological features of the visual pathways
from behavior.
Consequently, these experiments serve as useful example
of how to reason about biology from behavior.
