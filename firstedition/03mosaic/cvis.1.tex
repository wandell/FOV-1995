%
%	Sampling and Aliasing.
%
\chapter{The Photoreceptor Mosaic}
\label{chapter:mosaic}

In Chapter~\ref{chapter:imageformation}
we reviewed Campbell and Gubisch's (1967) measurements of
the optical linespread function.
Their data are presented in Figure \ref{f1:cg.linespread},
as smooth curves, but the actual measurements
must have taken place at a series of finely spaced intervals
called {\em sample points}.
In designing their experiment, Campbell and Gubisch
must have considered carefully how to space their sample points
because they wanted to space their measurement samples only
finely enough to capture the
intensity variations in the measurement plane.
Had they positioned their samples too widely, then
they would have missed significant variations in the data.
On the other hand, spacing the
sample positions too closely would have made
the measurement process wasteful of time and resources.

Just as Campbell and Gubisch sampled their linespread measurements,
so too the retinal image is sampled by the nervous system.
Since only those portions of the retinal image that
stimulate the visual photoreceptors can influence vision,
the sample positions are determined by the positions of
the photoreceptors.
If the photoreceptors are spaced too widely, 
the image encoding will miss significant variation
present in the retinal image.
On the other hand, if the photoreceptors
are spaced very close to one another
compared to the spatial variation
that is possible given the inevitable optical blurring,
then the image encoding will be redundant,
using more neurons than necessary to do the job.
In this chapter we will consider
how the spatial arrangement of the photoreceptors,
called the {\em photoreceptor mosaic},
limits our ability to infer the spatial pattern of 
light intensity present in the retinal image.

We will consider separately the photoreceptor mosaics of each
of the different types of photoreceptors.
There are two fundamentally different types of photoreceptors
in our eye, the {\em rods} and the {\em cones}.
There are approximately 5 million cones
and 100 million rods in each eye.
The positions of these two types of photoreceptors differ
in many ways across the retina.
Figure \ref{f2:receptor.dens}
shows how the relative densities of cone photoreceptors
and rod photoreceptors vary across the retina.

The rods initiate vision under low illumination levels,
called {\em scotopic} light levels, while
the cones initiate vision
under higher, {\em photopic} light levels.
The range of intensities in which both
rods and cones can initiate vision
is called {\em mesopic} intensity levels.
At most wavelengths of light, the cones
are less sensitive to light than the rods.
This sensitivity difference, coupled with the fact that
there are no rods in the fovea,
explains why
we can not see very dim sources, such as weak starlight,
when we fixate our fovea directly on them.
These sources are too dim to be visible through the all cone fovea.
The dim source only becomes visible when it is placed in the
periphery and be detected by the rods.
Rods are very sensitive light detectors:
they generate a detectable photocurrent response when
they absorb a single photon of light (Hecht et al., 1942; Schwartz,
1978; Baylor et al. 1987).

The region of highest visual acuity in the human retina
is the {\em fovea}.
As Figure~\ref{f2:receptor.dens} shows, the fovea
contains no rods, but it does contain the highest
concentration of cones.
There are approximately 50,000 cones in the human fovea.
Since there are no photoreceptors at the optic disk,
where the ganglion cell axons exit the retina,
there is a blindspot in that region of the retina 
(see Chapter~\ref{chapter:retina}).
\begin{figure}
\centerline {
 \psfig{figure=../03mosaic/fig/rod.cone.distribution.ps,clip=,height=2.8in}
}
\caption[Rods and Cones]{
{\em The distribution of rod and cone photorceptors}
across the human retina.
(a) The density of the receptors is shown in degrees
of visual angle relative to the position of the fovea
for the left eye.
(b)  The cone receptors are concentrated in the fovea.
The rod photoreceptors are absent from the fovea and reach their
highest density 10 to 20 degrees peripheral to the fovea.
No photoreceptors are present in the blindspot.
% After Fig 7.2 p. 137, Cornsweet, but pretty different now
}
\label{f2:receptor.dens}
\end{figure}

\begin{figure}
\centerline{
\psfig{figure=../03mosaic/fig/photoreceptor.ps,clip=,height=3.5in}
}
\caption[Schematic of Rods and Cones]{
{\em Mammalian rod and cone photoreceptors} contain the light
absorbing pigment that initiates vision.
Light enters the photoreceptors through the inner segment and is
funneled to the outer segment that contains the photopigment.
(After Baylor, 1987) % Friedenwald lecture, Figure 1, p. 35
}
\label{f2:photoreceptor}
\end{figure}
Figure \ref{f2:photoreceptor}
shows schematics of a mammalian rod and a cone photoreceptor.
Light imaged by the cornea and lens
is shown entering the receptors through the {\em inner segments}.
The light passes into the {\em outer segment}
which contain light absorbing {\em photopigments}.
As light passes from the inner to the
outer segment of the photoreceptor,
it will either be absorbed by one of the photopigment
molecules in the outer segment or it
will simply continue through the photoreceptor and
exit out the other side.
Some light imaged by the optics will pass between the
photoreceptors.
Overall, less than ten percent of the light entering the eye
is absorbed by the photoreceptor photopigments (Baylor, 1987).
\nocite{BaylorProctor}

The rod photoreceptors contain a photopigment called
{\em rhodopsin}.
The rods are small, there are many
of them, and they sample the retinal image very finely.
Yet, visual acuity under scotopic viewing conditions is very poor
compared to visual acuity under photopic conditions.
The reason for this is that the signals from many rods
converge onto a single neuron within the
retina, so that there is a many-to-one relationship between rod
receptors and neurons in the optic tract.
The density of rods and the convergence of their
signals onto single neurons improves the
sensitivity of rod-initiated vision.
Hence, rod-initiated vision does not resolve fine spatial detail.

The foveal cone signals do not converge onto single neurons.
Instead, several neurons encode the
signal from each cone, so that 
there is a one-to-many relationship between the foveal cones
and optic tract neurons.
The dense representation of the foveal cones
suggests that the spatial sampling of the
cones must be an important aspect of the visual encoding.

There are three types of cone photoreceptors
within the human retina.
Each cone can be classified based on
the wavelength sensitivity of
the photopigment in its outer segment.
Estimates of the spectral sensitivity of the three types of
cone photoreceptors
are shown in Figure \ref{f2:rec.spec.sens}.
These curves are measured from the cornea, so they include
light loss due to the cornea, lens and inert materials of the eye.
In the next chapter we will study how color vision depends
upon the differences in wavelength selectivity of the three
types of cones.
Throughout this book I will
refer to the three types of photoreceptors
as the $\Red$, $\Green$ and $\Blue$ cones\footnote{
The letters refer to {\bf L}ong-wavelength, {\bf M}iddle-wavelength
and {\bf S}hort-wavelength peak sensitivity.
}.

\begin{figure}
\centerline{
\psfig{figure=../03mosaic/fig/rec.spec.sens.ps,clip=,height=3.5in}
}
\caption[Cone Spectral Sensitivities]{
{\em Spectral sensitivities of the $\Red$, $\Green$ and $\Blue$ cones}
in the human eye.
The measurements are based on a light source
at the cornea, so that the wavelength loss due to the
cornea, lens and other inert pigments of the eye
play a role in determining the sensitivity.
(Source:  Stockman and Macleod, 1993).
}
\label{f2:rec.spec.sens}
\end{figure}

\begin{figure}
\centerline{
\psfig{figure=../03mosaic/fig/coneMosaic.ps,clip=,height=3.0in}
}
\caption[Photoreceptor Sampling]{
{\em The spatial mosaic of the human cones}.
A cross-section of the
human retina at the level of the inner segments.
Cones in the fovea (a)
are smaller than cones in the periphery (b).
As the separation between cones grows,
the rod receptors fill in the spaces.
(c)  The cone density varies with distance from
the fovea.
Cone density is plotted as a function of eccentricity
for seven human retinae (After Curcio et al, 1990).
}
% Part a is fig. 1 from Curcio et al. 1990
% Part b is fig. 7 from the same.
% 5)  Curcio, C A, et al.  "Human photoreceptor topography."  [BA]
%      JOURNAL OF COMPARATIVE NEUROLOGY. 1990 292(4):497-523.
%
% Another good article is:
% 1)  Curcio, C A, et al.  "Packing geometry of human cone photoreceptors: 
%      Variation with eccentricity and evidence for local anisotropy."  [BA]
%      Visual Neurosci. 1992 9(2):169-180.
%  And there is a blue cone article by Curcio, too.
\label{f2:retinal.samples}
\end{figure}
\nocite{Curcio1990,Curcio1992}
Because light is absorbed 
after passing through the inner segment,
the position of the inner segment
determines the spatial sampling position of the photoreceptor.
Figure \ref{f2:retinal.samples} shows cross-sections
of the human cone photoreceptors at the level of the inner segment
in the human fovea (part a) and just outside the fovea (part b).
In the fovea, cross-section shows
that the inner segments are very tightly packed
and form a regular sampling array.
A cross-section just outside the fovea
shows that the rod photoreceptors fill
the spaces between the cones
and disrupt the regular packing arrangement.
The scale bar represents $10 \mu m$;
the cone photoreceptor inner segments in the
fovea are approximately $2.3 \mu m$ wide
with a minimum center to center spacing of about $2.5 \mu m$.
Figure \ref{f2:retinal.samples}c shows plots of the
cone densities from several different human retinae
as a function of the distance from the foveal center.
The cone density varies across individuals.

% height = 5.0 width = 7
\begin{figure}
\centerline{
 \psfig{figure=../03mosaic/fig/viewingangle.ps ,clip= ,height=2.5in}
}
\caption[Calculating Viewing Angle]{
{\em Calculating viewing angle.}
By trigonometry, the tangent of the viewing angle, $\phi$,
is equal to the ratio of height to distance in the right triangle shown.
Therefore, $\phi$ is the inverse tangent of that ratio
(Equation~\ref{e2:viewingAngle}).
}
\label{f2:viewingangle}
\end{figure}

\subsection*{Units of Visual Angle}
We can convert these cone sizes and separations
into degrees of visual angle as follows.
The distance from the effective center of
of the eye's optics to the retina is $1.7 \times 10^{-2} m$
(17 mm).
We compute the visual angle spanned by one cone,
$ \phi $, from the trigonometric relationship in 
Figure~\ref{f2:viewingangle}:
the tangent of an angle in a right triangle is
equal to the ratio of the lengths of the sides 
opposite and adjacent to the angle.
This leads to the following equation:
\begin{equation}
\label{e2:viewingAngle}
\tan ( \phi ) 
  = { ( 2.5 \times 10 ^ {-6} m  ) } / { ( 1.7 \times 10 ^ {-2} m ) }
  = 1.47 \times 10 ^ {-4}
\end{equation}
The width of a cone
in degrees of visual angle, $\phi$, is
approximately $0.0084$ degrees,
or roughly one-half minute of visual angle.
In the center of the eye, then,
where the photoreceptors are packed densely,
the cone photoreceptors are tightly packed and
their centers are separated by one-half minute of visual angle.


\section{The $\Blue$ Cone Mosaic}

\subsection*{Behavioral Measurements}
Just as the rods and cones have different spatial
sampling distributions, so too
the three types of cone photoreceptors
have different spatial sampling distributions.
The sampling distribution of the short-wavelength cones
was the first to be measured empirically, and it
has been measured both with behavioral and physiological methods.
The behavioral experiments were carried out as part of 
D. Williams dissertation at the University of California in San Diego.
Williams, Hayhoe and MacLeod (1981)
took advantage of several features
of the short-wavelength photoreceptors.
As background to their work, we first describe several features of the
photoreceptors. 

The photopigment in the short-wavelength photoreceptors
is significantly different from the photopigment in the
other two types of photoreceptors.
%Moved Figure...better fix up prose around here
Notice that the wavelength
sensitivity of the $\Red$ and $\Green$ photopigments
are very nearly the same (Figure~\ref{f2:rec.spec.sens}).
The sensitivity of the $\Blue$ photopigment
is significantly higher in the short-wavelength part
of the spectrum than the sensitivity of
the other two photopigments.
As a result, if we present the visual system with a very weak
light, containing energy only in the short-wavelength portion
of the spectrum, the $\Blue$ cones
will absorb relatively more quanta than the other two classes.
Indeed, the discrepancy in the absorptions is so large
that it is reasonable to suppose that when 
short-wavelength light is barely visible, at detection threshold,
perception is initiated uniquely from a signal that originates
in the short-wavelength receptors.
\nocite{BaylorNunnSchnapf}
\nocite{WilliamsHayhowMacLeod1981}

We can give the short-wavelength receptors an even
greater sensitivity advantage by presenting a blue test target
on a steady yellow background.
As we will discuss in later chapters,
steady backgrounds suppress visual sensitivity.
By using a yellow background, we can suppress
the sensitivity of the $\Red$ and $\Green$ cones
and the rods and yet
spare the sensitivity of the $\Blue$ cones.
This improves the relative sensitivity advantage
of the short-wavelength receptors in detecting the
short-wavelength test light.

A second special feature of the $\Blue$ cones
is that they are very rare in the retina.
From other experiments described in
Chapter~\ref{chapter:wavelength},
it has been suspected for many years that
no cones containing short-wavelength photopigment are present
in the central fovea.
It had been earlier suspected that the number of
cones containing the short-wavelength photopigment was quite small
compared to the other two classes.
If the $\Blue$ cones are widely spaced,
and if we can isolate them with these choices of test
stimulus and background, then we can measure
the mosaic of short-wavelength photoreceptors.

During the experiment,
the subjects visually fixated on a small mark.
They were then presented
with short-wavelength test lights that were
likely to be seen with a signal initiated by the
$\Blue$ cones.
After the eye was perfectly fixated,
the subject pressed a button and initiated a stimulus presentation.
The test stimulus was a tiny point of light,
presented very briefly (10 ms).
The test light was presented at different points in the visual
field.
If light from the short-wavelength test
fell upon a region that contained $\Blue$ cones,
sensitivity should be relatively high.
On the other hand, if that region of the retina contained no
$\Blue$ cones, sensitivity should be rather low.
Hence, from the spatial pattern of visual sensitivity,
Williams, Hayhoe and Macleod
inferred the spacing of the $\Blue$ cones.

\begin{figure}
\centerline{
\psfig{figure=../03mosaic/fig/williams.dat.ps,clip=,height=3.5in}
}
\caption[Short-wavelength Cone Mosaic:  Psychophysics]{
{\em Psychophysical estimate of the spatial mosaic of the $\Blue$ cones.}
The height of the surface represents
the observer's threshold sensitivity
to a short wavelength test light presented on a yellow background.
The test was presented at a series
of locations spanning a grid around the fovea (black dot).
The peaks in sensitivity probably correspond to the
positions of the $\Blue$ cones.
(From Williams, Hayhoe, and Macleod, 1981).
% Williams, MacLeod and Hayhoe (1981)
% Vis. Res. V. 21. pp 1357
% Figure 3, p. 1361
}
\label{f2:williams.dat}
\end{figure}
The sensitivity measurements are shown
in Figure \ref{f2:williams.dat}.
First, notice that in the very center of the visual field, in
the central fovea, there is a large valley of
low sensitivity.
In this region, there appear to be no short-wavelength cones
at all.
Second, beginning about half a degree from
the center of the visual field there are small, punctate spatial regions of
high sensitivity.
We interpret these results by
assuming that these peaks correspond to the
positions of this observer's $\Blue$ cones.
The gaps in between, where the observer has rather
low sensitivity are likely to be patches
of $\Red$ and $\Green$ cones.
Around the central fovea,
the typical separation between the
inferred $\Blue$ cones is about 8 to 12 minutes of visual
angle.
Thus, there are five to seven $\Blue$ cones per degree of visual angle.

\subsection*{Biological Measurements}
There have been several biological measurements
of the short-wavelength cone mosaic, and we can
compare these with the behavioral measurements.
Marc and Sperling (1977) used a stain that is taken up
by cones when they are active.
They applied this stain to a baboon retina and then
stimulated the retina with short-wavelength light
in the hopes of staining only the short-wavelength receptors.
They found that only a few cones
were stained when the stimulus was a short-wavelength light.
The typical separation between the stained cones
was about 6 minutes of arc.
This value is smaller than the separation 
that Williams' et al. observed and may be
a species-related difference.

\begin{figure}
\centerline{
\psfig{figure=../03mosaic/fig/blueConeMosaic.ps,clip=,height=2.8in}
}
\caption[Short-Wavelength Cone Mosaic:  Procion Yellow Stains]{
{\em Biological estimate of the spatial mosaic 
of the $\Blue$ cones} in the macaque retina.
A small fraction of the cones absorb
the procion yellow stain;  these are shown as the
dark spots in this image.
These cones, thought to be the $\Blue$ cones, 
are shown in a cross-section
through the inner segment layer of the retina.
(From DeMonasterio, Schein and McCrane, 1985)
% Use figure in Rodieck's review of the primate retina
% from DeMonasterio, McCrane, Newlander, Schein Density profile of
% blue-sensitive cones along the horizontal meridian of macaque
% retina.  IOVS v. 26 p. 289-302, 1985 in Rodieck The Primate Retina
% 1988 Comparative Primate Biology, V. 4, Neurosciences, p. 203-278
% fig 12. p. 218.  In fact, I think one of these might be that one scanned.
}
\label{f2:blueConeMosaic}
\end{figure}
F. DeMonasterio, S. Schein, and E. McCrane (1981)
discovered that when the dye procion yellow
is applied to the retina, the dye is absorbed in the outer
segments of all the photoreceptors, but it stains only
a small subset of the photoreceptors completely.
Figure \ref{f2:blueConeMosaic} shows
a group of stained photoreceptors in cross-section section.
\nocite{deMonasterioScheinMcCrane}

The indirect arguments identifying these special cones
as $\Blue$ cones are rather compelling.
But, a more certain procedure was developed
by C. Curcio and her colleagues.
They used a biological marker, developed based on knowledge
of the genetic code for the $\Blue$ cone photopigment,
to label selectively the $\Blue$ cones in the human retina
(Curcio, et al. 1991).
Their measurements agree well quantitatively with Williams'
psychophysical measurements, namely that the average spacing between
the $\Blue$ cones is 10 minutes of visual angle.
Curcio and her colleagues could also
confirm some early anatomical observations that the
size and shape of the $\Blue$ cones differ slightly from the $\Red$
and $\Green$ cones.
The $\Blue$ cones have a wider inner segment,
and they appear to be
inserted within an orderly sampling arrangement of their
own between the sampling mosaics of the other two
cone types (Ahnelt, Kolb and Pflug, 1987).

\subsection*{Why are the $\Blue$ cones widely spaced?}
The spacing between the $\Blue$ cones is 
much larger than the spacing between
the $\Red$ and $\Green$ cones.
Why should this be?
The large spacing between the $\Blue$ cones
is consistent with the strong blurring
of the short-wavelength component of the
image due to the axial chromatic aberration of the lens.
Recall that axial chromatic aberration
of the lens blurs the short-wavelength portion of
the retinal image, the part $\Blue$ cones are particularly
sensitive to, more than the middle- and long-wavelength portion of
the image (Figure~\ref{f1:otfTheoryData}).
In fact, under normal viewing conditions
the retinal image of a fine line at 450 nm 
falls to one half its peak intensity nearly 10 minutes of visual
angle away from the location of its peak intensity.
At that wavelength, the retinal image only contains 
significant contrast at spatial frequency components below 
3 cycles per degree of visual angle.
The optical defocus
force the wavelength components of the retinal image
the $\Blue$ cones encode to vary smoothly across space.
Consequently, the $\Blue$ cones can sample the image
only six times per degree
and still recover the spatial variation passed
by the cornea and lens.

Interestingly,
the {\em spatial} defocus of the short-wavelength component of the image
also implies that signals initiated by the $\Blue$ cones will
vary slowly over {\em time}.
In natural scenes, temporal variation occurs mainly
because of movement of the observer or an object.
When a sharp boundary moves across a cone position,
the light intensity changes rapidly at that point.
But, if the boundary is blurred, changing gradually over space,
then the light intensity changes more slowly.
Since the short-wavelength signal is blurred by the optics,
and temporal variation is mainly due to motion of objects,
the $\Blue$ cones will generally be coding
slower temporal variations than the $\Red$ and $\Green$ cones.

At the very earliest stages of vision, we see that the
properties of different components of the visual pathway
fit smoothly together.
The optics set an important limit on visual acuity,
and the $\Blue$ cone sampling mosaic
can be understood as a consequence
of the optical limitations.
As we shall see,
the $\Red$ and $\Green$ cone mosaic densities
also make sense in terms of the optical quality of the eye.

This explanation of the $\Blue$ cone mosaic flows from our
assumption that visual acuity is the main factor governing
the photoreceptor mosaic.
For the visual streams initiated by the cones, this is a 
reasonable assumption.
There are other important factors, however, that can play
a role in the design of a visual pathway.
For example, acuity is not the dominant factor
in the visual stream initiated by rod vision.
In principle the resolution available in the rod encoding is
comparable to the acuity available in the cone responses;
but, visual acuity using rod-initiated signals
is very poor compared to acuity using cone-initiated signals.
Hence, we shouldn't think of the rod sampling mosaic 
in terms of visual acuity.
Instead, the high density of the rods
and their convergence onto individual neurons
suggests that we think
of the imperative of rod-initiated vision
in terms of improving the signal-to-noise 
under low light levels.
In the rod-initiated signals,
the visual system trades visual acuity
for an increase in the signal-to-noise ratio.
In the earliest stages of the visual pathways, then,
we can see structure, function and design criteria
coming together.

When we ask why the visual system has a particular property,
we need to relate observations from the
different disciplines that make up vision science.
Questions about anatomy
require us to think about the behavior the 
anatomical structure serves.
Similarly, behavior must be explained in terms of
algorithms and the anatomical and physiological
responses of the visual pathway.
By considering the visual pathways from multiple points
of view, we piece together
a complete picture of how system functions.

\section{Visual Interferometry}

\begin{figure}
\centerline{
\psfig{figure=../03mosaic/fig/Young.ps,clip=,height=3.0in}
}
\caption[Interference and Double Slits]{
{\em T. Young's double-slit experiment}
uses a pair of coherent light sources
to create an interference pattern of light.
The intensity of the resulting image is nearly sinusoidal, 
and its spatial frequency depends
upon the spacing between the two slits.
}
\label{f2:Young}
\end{figure}
In behavioral experiments,
we measure threshold repeatedly through individual $\Red$ and $\Green$
using small points of light as we did the $\Blue$ cones.
The pointspread function distributes
light over a region containing about twenty cones, so that
the visibility of even a small point of light may involve
any of the cones from a large pool
(see Figures \ref{f1:cg.linespread} and \ref{f1:westheimer.ls}).
We can, however, use a method
introduced by Y. LeGrand in 1935
to defeat the optical blurring.
The technique is called {\em visual interferometry},
and it is based upon the principle of diffraction.

Thomas Young (1802),
the brilliant scientist, physician, and classicist
demonstrated to the Royal Society that when
two beams of coherent light generate an image on
a surface such as the retinal surface,
the resulting image is an interference pattern.
His experiment is often called the {\em double-slit} or
{\em double-pinhole} experiment.
Using an ordinary
light source, Young passed the light through a small pinhole first
and then through a pair of slits, as illustrated
in Figure \ref{f2:Young}.
In the experiment,
the first pinhole serves as the source of light;
the double pinholes then pass the light from the
common original source.
Because they share this common source,
light emitted from the double pinholes are in
a coherent phase relationship
and their wavefronts interfere with one another.
This interference results in an image that varies nearly
sinusoidally in intensity.

% 8 in by 10 in
\begin{figure}
\centerline {
\psfig{figure=../03mosaic/fig/interferometer.ps,clip=,height=2.5in}
}
\caption[Visual Interferometer]{
{\em A visual interferometer} creates an interference pattern
as in Young's double-slit experiment.
In the device shown here
the original beam is split into two paths shown as the solid
and dashed lines.
(a)  When the glass cube is at right angles to the light path,
the two beams traverse an equal path and are imaged at the
same point after exiting the interferometer.
(b)  When the glass is rotated, the two beams traverse slightly
different paths causing the images of the two coherent
beams to be displaced and thus create an interference pattern.
(After Macleod, Williams and Makous, 1992).
% Vision Research v. 32, no. 2 p. 347
% Inset in fig. 3 was the inspiration
}
\label{f2:interferometer}
\end{figure}
We can also achieve this narrow pinhole effect
by using a laser as the original source.
The key elements of a visual interferometer used by
MacLeod et al. (1992) are shown in Figure~\ref{f2:interferometer}.
Light from a laser enters
the beamsplitter and is divided into
one part that continues along a straight path (solid line)
and a second path that is reflected along
a path to the right (dashed line).
These two beams, originating
from a common source, will be the pair of sources
to create the interference pattern on the retina.

Light from each beam is reflected from a mirror towards a glass cube.
By varying
the orientation of the glass cube,
the experimenter can vary the path of the two beams.
When the glass cube is at right angles to the light
path, as is shown in part (a),
the beams continue in a straight path along
opposite directions and
emerge from the beamsplitter at the same position.
When the glass cube is rotated, as is shown in part (b),
the refraction due
to the glass cube symmetrically changes the beam paths;
they emerge from the beamsplitter at slightly different locations
and act as a pair of point sources.
This configuration creates two coherent beams that act like
the two slits in Thomas Young's experiment, creating
an interference pattern.
The amount of rotation of the glass cube controls the
separation between the two beams.

Each beam passes through 
only a very small section of the cornea and lens.
The usual optical blurring mechanisms do not
interfere with the image formation,
since the lens does not serve to converge the light
(see the section on lenses in Chapter \ref{chapter:imageformation})).
Instead, the pattern that is formed depends upon the
diffraction due to the restricted spatial region of the
light source.

\begin{figure}
\centerline{
\psfig{figure=../03mosaic/fig/interference.sinusoid.ps,clip=,height=1.7in}
}
\caption[Sinusoidal Interference Pattern]{
{\em An interference pattern.}  The image
was created using a double-slit apparatus.
The intensity of the pattern is nearly sinusoidal.
(From Jenkins and White, 1976.)
% Fundamentals of Optics
% Figure 13D page 262.
}
\label{f2:interference.sinusoid}
\end{figure}
We can use diffraction to create retinal images with
much higher spatial frequencies than are possible
through ordinary optical imaging by the cornea and lens.
Figure \ref{f2:interference.sinusoid}
is an image of a diffraction pattern
created by a pair of two slits.
The intensity of the pattern
is nearly a sinusoidal function of retinal position.
The spatial frequency of the retinal image can be
controlled by varying the separation between the focal points;
the smaller the separation between the slit,
the lower the spatial frequency in the interference pattern.
Thus, by rotating the glass cube in the interferometer and
changing the separation of the two beams we can control
the spatial frequency of the retinal image.

Visual interferometry
permits us to image fine spatial patterns at
much higher contrast than
when we image these patterns using ordinary
optical methods.
For example, Figure \ref{f1:otfTheoryData} shows that
a $60$ cycles per degree sinusoid cannot
exceed 10 percent contrast when imaged through the optics.
Using a visual interferometer, we can present
patterns at frequencies considerably higher
than $60$ cycles per degree at 100 percent contrast.

But a challenge remains:
the interferometric patterns are not fine lines or points,
but rather extended patterns (cosinusoids).
Therefore, we cannot use the
same logic as Williams et al.
and map the receptors by carefully positioning
the stimulus.
We need to think a little bit more
about how to use the cosinusoidal interferometric patterns
to infer the structure of the cone mosaic.

\section{Sampling and Aliasing}

% width = 8.0 height = 9.5in
\begin{figure}
\centerline{
\psfig{figure=../03mosaic/fig/aliasing.ps,clip=,height=4.75in}
}
\caption[Aliasing Examples]{
{\em Aliasing} of signals results when sampled values are
the same but in-between values are not.
(a,b) The continuous sinusoids on the left
have the same values at the sample positions indicated
by the black squares.
The values of the two functions at the sample positions
are shown by the height of the stylized arrows on the right.
(c) Undersampling may cause us to confuse various
functions, not just sinusoids.
The two curves at the bottom have the same values at the
sampled points, differing only in between the sample positions.
\label{f2:aliasing}
}
\end{figure}
In this section we
consider how the cone mosaic encodes
the high spatial frequency patterns created by visual interferometers.
The appearance of these high frequency patterns
will permit us to deduce the spatial arrangement of
the combined $\Red$ and $\Green$ cone mosaics.
The key concepts that we must understand to
deduce the spatial arrangement of the mosaic
are {\em sampling} and {\em aliasing}.
These ideas are illustrated in Figure~\ref{f2:aliasing}.

The most basic observation concerning sampling and aliasing is this:
we can measure only that portion of the input signal
that falls over the sample positions.
Figure \ref{f2:aliasing}
shows one-dimensional examples of aliasing and sampling.
Parts (a) and (b) contain two different cosinusoidal signals (left)
and the locations of the sample points.
The values of these two cosinusoids at the sample points
are shown by the height of the arrows on the right.
Although the two continuous cosinusoids are quite different,
they have the same values at the sample positions.
Hence, if cones are only present at the sample positions,
the cone responses will not distinguish
between these two inputs.
We say that these two continuous signals are an {\em aliased} pair.
Aliased pairs of signals
are indistinguishable after sampling.
Hence, sampling degrades our ability to discriminate between
sinusoidal signals.

Figure~\ref{f2:aliasing}c shows
that sampling degrades our ability to discriminate
between signals in general, not just between sinusoids.
Whenever two signals agree at the sample points,
their sampled representations agree.
The basic phenomenon of aliasing is this:
Signals that only differ between the sample points
are indistinguishable after sampling.

% 6 inches wide, 7 inches high
\begin{figure}
\centerline{
 \psfig{figure=../03mosaic/fig/aliasExample.ps,clip= ,height=4.9in}
}
\caption[Squarewave aliasing]{
{\em Squarewave aliasing.}
The squarewave on top is seen accurately
through the grid.
The squarewave on the bottom
is at a higher spatial frequency than the grid
sampling.
When seen through the grid,
the pattern appears at a lower spatial frequency
and rotated.
}
\label{f2:aliasExample}
\end{figure}
The exercises at the end of this chapter
include some computer programs
that can help you make sampling demonstrations like
the one in Figure~\ref{f2:aliasExample}.
If you print out squarewave patterns and various
sampling arrays, using the programs provided,
you can print various patterns onto
overhead transparencies and explore the effects
of sampling.
Figure \ref{f2:aliasExample} shows an example of
two squarewave patterns seen through a sampling grid.
After sampling, the high frequency pattern
appears to be a rotated, low frequency signal.

\paragraph{Sampling is a Linear Operation. }
The sampling transformation
takes the retinal image as input and generates
a portion of the retinal image as output.
Sampling is a linear operation as the following thought experiment reveals.
Suppose we measure the sample values at the cone
positions when we present image $A$;
call the intensities at the sample positions $S(A)$.
Now, measure the intensities at the sample positions for a second
image, $B$;
call the sample intensities $S(B)$.
If we add together the two images,
the new image, $A + B$, contains the sum of the intensities
in the original images.
The values picked out by sampling
will be the sum of the two sample vectors, $S(A) + S(B)$.

Since sampling is a linear transformation,
we can express it as a matrix multiplication.
In our simple description,
each position in the retinal image either
falls within a cone inner segment or not.
The sampling matrix consists of $N$ rows representing
the $N$ sampled values.
Each row is all zero except at the 
entry corresponding to that row's sampling position,
where the value is $1$.

\paragraph{Aliasing of harmonic functions.  }
For uniform sampling arrays
we have already observed that
some pairs of sinusoidal
stimuli are aliases of one another (part (a) of Figure \ref{f2:aliasing}).
We can analyze precisely which pairs of sinusoids
form alias pairs using a little bit of algebra.
Suppose that the  continuous input signal is $\cos ( 2 \pi f x )$.
When we sample the stimulus at regular intervals, the
output values will be the value of the cosinusoid
at those regularly spaced sample points.
Suppose that within a single unit of distance there are
$N$ sample points, so that our measurements of the stimulus
takes place every $ 1 / N$ units.
Then the sampled values will be
$S_{f} ( k ) = \cos ( { 2 \pi f } { k / N } )$.
A second cosinusoid, at frequency $f'$  will be an alias
if its sample values are equal, that is,
if $S_{f'} (k) = S_{f} (k)$.

With a little trigonometry, we can prove that the
sample values for any pair of cosinusoids with frequencies
${N / 2} - f$ and ${N / 2} + f $ will be equal.
That is,

\[
\cos ( \frac{2 \pi ({ N / 2 } + f )  k }{ N } ) = \cos ( \frac {2 \pi ({ N/ 2 } - f )  k }{ N }  ) 
\]

(To prove this we must use the cosine addition law to expand the
right sides of the following equation.
The steps in the verification are left as
exercise \ref{q:aliasing} at the end of the chapter.)
\comment{
Make sure this is in the problem section at the end of the chapter
}

The frequency $f = N / 2 $ is called the {\em Nyquist frequency}
of the uniform sampling array;
sometimes it is referred to as the {\em folding frequency}.
Cosinusoidal stimuli whose
frequencies differ by equal amounts above and below
the Nyquist frequency of a uniform sampling
array will have identical sample responses.

\paragraph{Experimental Implications.}
The aliasing calculations suggest an experimental method to
measure the spacing of the cones in the eye.
If the cone spacing is uniform,
then pairs of stimuli separated by equal amounts
above and below the Nyquist frequency
should appear indistinguishable.
Specifically, a signal $\cos (2 \pi ( { N / 2} + f ) )$ that is
above the Nyquist frequency
will appear the same as the signal
$\cos ( 2 \pi ( { N / 2 } - f) )$ that is an equal
amount below the Nyquist frequency.
Thus, as subjects view interferometric patterns
of increasing frequency, as we cross the Nyquist
frequency the perceived spatial frequency should
begin to decrease even though the
physical spatial frequency of the diffraction pattern increases.

Yellott (1982) examined the aliasing prediction
in a nice graphical way.
He made a sampling grid from Polyak's (1957)
anatomical estimate of the cone positions.
He simply poked small holes in the paper
at the cone positions in one of Polyak's
anatomical drawings.
We can place any image we like, for example
patterns of light and dark bars, behind the grid.
The bits of the image that we see are only those
that would be seen by the visual system.
Any pair of images that differ only in the regions between
the holes will be an aliased pair.
Yellott introduced the method
and proper analysis,
but he used Polyak's (1957) data on the outer segment
positions rather than on
the positions of the inner segments (Miller and Bernard, 1983).
\nocite{Yellot1982,Polyak1957,MillerBernard1983}

This experiment is relatively straightforward
for the $\Blue$ cones.
Since these cones are separated by about
$10$ minutes of visual angle,
there are about six $\Blue$ cones per degree
of visual angle.
Hence, their Nyquist frequency is
$3$ cycles per degree of visual angle (cpd).
It is possible to correct for chromatic aberration
and to present spatial patterns at these low frequencies through the lens.
Such experiments confirm the basic predictions
that we will see aliased patterns (Williams and Collier, 1983).

\section{The $\Red$ and $\Green$ Cone Mosaic}

\begin{figure}
\centerline{
\psfig{figure=../03mosaic/fig/aliasDrawings.ps,clip=,height=3.5in}
}
\caption[Drawings of Aliases]{
{\em Drawings of perceived aliasing patterns} by several
different observers.
Helmholtz' observed aliasing of fine patterns which
he drew in part H1.
He offered an explanation of his observations,
in terms of cone sampling, in H2.
% Vol II, page 35, from Helmholtz' Physiological Optics
Byram's (1944) drawings of three interference patterns at
40, 85 and 150 cpd are labeled B1, B2, and B3.
%JOSA v. 34, no. 12 Figure 5, p. 723 
\comment{
These are drawings of
of the appearance of high frequency interference gratings
at what he believed were 80 (40?) (a = 0.75 min)
170 (85?) (0.35), and 300 (150?) (0.20) cycles per degree
of visual angle.
}
Drawings W1,W2 and W3 are
by subjects in Williams' laboratory 
who drew their impression
of aliasing of an 80 cpd and two patterns at 110 cpd
% Williams, Vision Research, figure 4., 1985, p. 200 v. 25 no. 2
}
\label{f2:zebra}
\end{figure}
Experiments using a visual interferometer to image a
high frequency pattern at high contrast on the retina are
a powerful way to analyze
the sampling mosaic of $\Red$ and $\Green$ cones.
But, even before this was technical feat was possible,
Helmholtz' (1896) noticed that extremely fine patterns,
looked at without any special apparatus, can appear wavy.
He attributed this observation to sampling by the cone mosaic.
His perception of a fine pattern and his graphical
explanation of the waviness in terms of sampling by
the cone mosaic are shown in
part (a) of Figure~\ref{f2:zebra} (boxed drawings).

G. Byram was the first to describe the appearance of high frequency
interference gratings (Byram, 1944).
His drawings of the appearance of these patterns are shown in part (b)
of the figure.
The image on the left shows the appearance
of a low frequency pattern diffraction pattern.
The apparent spatial frequency of this stimulus
is faithful to the stimulus.
Byram noted that as the spatial frequency increases towards
60 cpd, the pattern still appears to
be a set of fine lines, but they are difficult to see (middle drawing).
When the pattern significantly exceeds the Nyquist frequency,
it becomes visible again but looks like the low-frequency pattern
drawn on the right.
Further, he reports that the pattern shimmers and is unstable, probably
due to the motion of the pattern with respect to the cone mosaic.
\nocite{Helmholtz,BYram,WilliamsAliasing}

Over the last 10 years D. Williams' group has replicated
and extended these measurements using an improved visual interferometer.
Their fundamental observations are consistent with
both Helmholtz and Byram's reports,
but greatly extend and quantify the earlier measurements.
The two illustrations on the left of part (c) of Figure~\ref{f2:zebra}
show Williams' drawing of 80 cpd and
110 cpd sinusoidal gratings created on the retina using a
visual interferometer.
The third figure shows an artist's drawing
of a 110 cpd grating.
The drawing on the left covers a large
portion of the visual field, and the appearance of the
patterns varies across the visual field.
For example, at 80 cpd the observer sees 
high contrast stripes at some positions,
while the field appears uniform in other parts of the field.
The appearance varies, but the stimulus itself is quite uniform.
The variation in appearance is 
due to changes in the sampling density of the cone mosaic.
Cone sampling density is lower in the periphery than in the
central visual field,
so aliasing begins at lower spatial frequencies in the periphery
than in the central visual field.
If we present a stimulus at a high enough spatial frequency
we observe aliasing in the central and peripheral visual field,
as the drawings of the 110 cpd patterns in Figure~\ref{f2:zebra} show.

There are two extensions of these ideas on aliasing you should consider.
First, the cone packing in the fovea occurs in two dimensions,
of course, so that we must ask what the appearance of the
aliasing will be at different orientations of the sinusoidal
stimuli.
As the images in Figure \ref{f2:aliasExample} 
show, the orientation of the low frequency alias
does not correspond with the orientation of the input.
By trying the demonstration yourself and rotating the
sampling grid, you will see that the direction of motion
of the alias does not correspond with the motion of the
input stimulus\footnote{
Use the Postscript program in the appendix
section to print out a grid and a fine pattern and try this experiment.
}.
These kinds of aliasing confusions have also been reported
using visual interferometry (Coletta and Williams, 1987).
\nocite{ColettaWilliams1987}

Second, our analysis of foveal sampling has been based on
some rather strict assumptions concerning the cone mosaic.
We have assumed that the cones are all of the
same type, that their spacing is perfectly uniform, and that
they have very narrow sampling apertures.
The general model presented in this
chapter can be adapted if any
one of these assumptions fails to hold true.
As an exercise for yourself,
a new analysis with altered assumptions might
change the properties of the sampling matrix.

\subsection*{Visual Interferometry: Measurements of Human Optics}
There is one last idea you should take away from this chapter:
Using interferometry, we can
estimate the quality of the optics of the eye.

Suppose we ask an observer to set the contrast of a
sinusoidal grating, imaged using normal incoherent light.
The observer's sensitivity to the target will depend on
the contrast reduction at the optics and the observer's
neural sensitivity to the target.
Now, suppose that we create the same sinusoidal
pattern using an interferometer.
The interferometric stimulus
bypasses the contrast reduction due to the optics.
In this second experiment, then, the observer's sensitivity
is limited only by the observer's neural sensitivity.
Hence, the sensitivity difference
between these two experiments
is an estimate of the loss due to the optics.

The visual interferometric method
of measuring the quality of the optics
has been used on several occasions.
While the interferometric estimates are similar
to estimates using reflections from the eye,
they do differ somewhat.
The difference is shown in Figure~\ref{f1:otfTheoryData}
which includes the Westheimer's estimate
of the modulation transfer function,
created by fitting data from reflections,
along with data and a modulation transfer function
obtained from interferometric measurements.
The current consensus is that
the optical modulation transfer function
is somewhat closer
to the visual interferometric measurements than
the reflection measurements.
The reasons for the differences are discussed in several papers
(e.g. Campbell and Green, 1965;  Williams 1985; Williams et al., 1995).

\section{Summary and Discussion}

The $\Blue$ cones are present at a much lower
sampling density, and they are absent in the very
center of the fovea.
Because they are sparse,
we can measure the $\Blue$ cone positions
behaviorally using small points of light.
The behavioral estimates of the $\Blue$ cones
are also consistent with anatomical
estimates of the $\Blue$ cone spacing.

The wide spacing of the $\Blue$ cones can be understood
in terms of the chromatic aberration of the eye.
The eye is ordinarily in focus for the 
middle-wavelength part of the visual spectrum, and there is very
little contrast beyond 2-3 cycles per degree
in the short-wavelength part of the spectrum.
The sparse $\Blue$ cone spacing is matched to the poor
quality of the retinal image in the short-wavelength
portion of the spectrum.

The $\Red$ and $\Green$ cones are tightly packed in the central fovea,
forming a triangular grid that efficiently
samples the retinal image.
Ordinarily, optical defocus protects us
from aliasing in the fovea.
Once aliasing between two
signals occurs, the confusion cannot be undone.
The two signals have created precisely
the same spatial pattern of photopigment absorptions;
hence, no subsequent processing, through cone to cone interactions
or later neural interpolation, can undo the confusion.
The optical defocus prevents
high spatial frequencies that might alias
from being imaged on the retina.

By creating stimuli with a visual interferometer,
we bypass the optical defocus and image
patterns at very high spatial frequencies on the cone mosaic.
From the aliasing properties of these
patterns, we can deduce some of the
properties of the $\Red$ and $\Green$ cone mosaics.
The aliasing demonstrations
show that the foveal sampling grid is regular and
contains approximately 120 cones per degree of visual angle.
These measurements, in the living human eye, are
consistent with the anatomical images obtained of
the human eye reported by Curcio and her colleagues
(Curcio, et al., 1991).

The precise arrangement of $\Red$ and $\Green$
cones within the human retina is unknown,
though data on this point should arrive shortly
(e.g., Bowmaker and Mollon, 1993).
Current behavioral
estimates of the relative number of $\Red$ and $\Green$
cones suggest 
that there are about twice as many $\Red$ cones
as $\Green$ cones (Cicerone and Nerger, 1989).

The cone sampling grid becomes more coarse and irregular outside
the fovea where rods and other cells enter the spaces
between the cones.
In these portions of the retina, high frequency
patterns presented through interferometry no longer
appear as regular low frequency frequency patterns.
Rather, because of the disarray in the cone spacing,
the high frequency patterns appear to be mottled noise.
In the periphery, the cone spacing falls off rapidly enough so that
it should be possible to observe aliasing without the use
of an interferometer (Yellott, 1982).

In analyzing photoreceptor sampling, we have ignored
eye movements.
In principle, the variation in receptor intensities
during these small eye movements can provide
information to permit us to discriminate between the alias pairs.
(You can check this effect by studying the images you observe when you
experiment with the sampling grids.)
The effects of eye movements are often minimized in experiments
by flashing the targets briefly.
But, even when one examines the interferometric pattern for
substantial amounts of time, the aliasing persists.
The information available from small eye movements
could be very useful;
but, the analysis assuming
a static eye offers a good account
of current empirical measurements,
This suggests that the nervous system does not integrate
information across minute eye movements
to improve visual resolution (Packer and Williams, 1992).
\nocite{PackerWilliamsVR1992v32}

