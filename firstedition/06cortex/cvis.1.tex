\chapter{The Cortical Representation}
\label{chapter:Cortex}

In this chapter we will review the representation of information in
visual cortex.  There have been many advances in our understanding of
visual cortex over the last twenty-five years.  Even today, our view
of visual cortex is changing rapidly; new results that change our
overall view sometimes seem to arrive weekly.  In the beginning of
this chapter, I will review what is commonly accepted concerning
visual cortex.  Towards the end, I will introduce some of the broader
claims that have been made about the relationship between visual
cortex and perception.  We will take up the issue of connecting
cortex, computation, and seeing again in the later chapters.

\section{Overview of the Visual Cortex}

\begin{figure}
\centerline{
 \psfig{figure=../06cortex/fig/brain.ps,clip= ,width=5.5in}
}
\caption[A view of the brain]{
{\em The cortex} is shown in lateral view.
Based on its overall shape,
anatomists divide the human brain into four regions
called the occipital, temporal, parietal and frontal lobes.
Based on its internal connections,
the cortex can be further divided into many 
anatomically distinct areas.
Visual input to the brain arrives in primary visual cortex,
area V1, which is located in the occipital lobe.
}
\label{f5:brain}
\end{figure}
A lateral view of the brain is sketched in Figure~\ref{f5:brain}. 
The human cortex is a 2mm thick sheet of neurons with a surface
area of 1400 square centimeters.
Rather than lining the skull, as
the retina lines the eye, the visual cortex is like a crumpled
sheet stuffed into the skull.
Each location where the
folded cortex forms a ridge visible from the exterior is called a {\em
gyrus}, while each shallow furrow that separates a pair of gyri is
called a {\em sulcus}.  The pattern of sulci and gyri differ
considerably across species:
the human brain contains more sulci than other primate brains.
There are also significant differences
between human brains, although the broad outlines of the sulcal
and gyral patterns are usually present and
recognizable across different people.
The gyri and sulci are convenient landmarks, but they probably
have no functional significance.

The most visible sulci are used as markers to partition the human brain
into four {\em lobes}.  The lobes are called {\em frontal,
parietal, temporal} and {\em occipital} to describe their relative positions
(see Figure~\ref{f5:brain}).  Each lobe contains many distinct brain
{\em areas}, that is contiguous groups of cortical neurons that appear to
function in an interrelated manner.
A cortical area is identified in several ways, though perhaps
the most significant is by its anatomical connections with other parts
of the brain.  Each brain area makes a distinctive pattern of
anatomical connections with other brain areas.  The inputs arriving to
one area come from only a few other places in the brain, and the
outputs emerging from that area are sent to a specific set of
destination areas.

In the primate, the great part of the
visual signal from the retina and
the lateral geniculate nucleus
arrives at a single area within the occipital lobe 
of the cortex called {\em area V1}, or
{\em primary visual cortex}.  
This is a large cortical area, comprising
roughly $1.5 \times 10^8$ neurons, many more than the
$10^6$ neurons in the lateral geniculate nucleus.
Area V1 can be identified by a prominent striation
made up of a dense collection of 
myelinated axons within
one of the layers of visual cortex.
The striation is coextensive with
area V1 and appears as a white band to the naked eye\footnote{
Because area V1 was defined by the presence of
this striation, it is sometimes called {\em striate cortex}.
The stripe is also called the {\em Stria of Gennari}.}.
Because of its prominence, important anatomical location and
large size (18 square centimeters),
area V1 has been the subject of intense study.
We will begin
this chapter with a review
of the anatomical and electrophysiological features of area V1.

In addition to area V1, more than twenty cortical areas have been
discovered that receive a strong visual input.  The anatomy,
electrophysiology and computational purpose of these areas are now
under active study and will be an important topic for study for many
years to come.  We will review some of the preliminary experiments
that have been performed in these visual areas at the end of this
chapter.  In later chapters concerning motion and color, we will
return to consider the functional role of these visual areas as well
(Zeki, 1978, 1990; Felleman and Van Essen, 1991).

Most of what we know about cortical visual areas comes from
experimental studies of cat and monkey.  There are significant
differences in the anatomy and functional properties of the
cortices of different species.
These differences can be demonstrated in simple
experimental manipulations.  For example, Sprague et al. (1977) have
shown that removal of the cat primary visual cortex does not blind the
cat: the animal jumps, runs, and appears normal to the casual
observer.  Humphrey (1974) has studied the behavior of a monkey whose
area V1 was removed.  Initially the lesion appeared to blind the
monkey completely.  Over time, however, the monkey recovered some
visual function and was able to walk around objects, climb a tree, and
even find and pick up small candy pellets in her play area.  In human,
the loss of area V1 is devastating to all visual function.
Because of these differences, I describe measurements
of the human brain whenever possible,
and mainly I have restricted this review to primate.

\section{The Architecture of Primary Visual Cortex} 

There is a great deal of precision
in the interconnections of cortical visual areas.
The specific pattern of connections received by area V1
from the two retinae via the lateral geniculate nucleus
results in certain regularities of the
architecture of primary visual cortex.
We review the anatomical structure of area V1 first.
Then, we review how the pattern of connections
from the two retinae imposes an overall organization
on the visual information represented in cortical area V1.

\subsection*{The layers of area V1}
\begin{figure}
\centerline{
  \psfig{figure=../06cortex/fig/cortexLayers.ps,clip= ,width=5.5in}
}
\caption[The layers of area V1]{
{\em Area V1 is a layered structure}.
(a) A stained cross-section of the visual cortex in macaque
shows the individual layers.
Each layer has a different proportions of cell bodies,
dendrites and axons and may be distinguished by the density
of the staining and other properties.
The light areas are blood vessels.
(Source: J. Lund, personal communication).
(b) The organization of the
neural inputs and outputs to area V1 are shown.
The parvocellular and magnocellular inputs make connections
in layer 4C.
The intercalated neurons make connections in the superficial layers.
The outputs are sent to other cortical areas,
back to the lateral geniculate nucleus and other subcortical
nuclei.
}
\label{f5:cortexLayers}
\end{figure}
Like cortex in general, area V1 is a layered structure.
Figure~\ref{f5:cortexLayers}a shows a cross section of the visual
cortex.
Several major layers can be identified easily.
Area V1 is segregated into six layers
based on differences in the relative density of neurons,
axons and synapses and 
interconnections to the rest of the brain.
The superficial layer 1 has very few neurons but many
axons, dendrites and synapses, which collectively are called
{\em neuropil}.
Layers 2 and 3 consists of a dense array of cell bodies
and many local dendritic interconnections.
These layers appear to receive a direct input from the
intercalated layers of the lateral geniculate as well
(Fitzpatrick et al., 1983; Hendry and Yoshioka, 1994),
and the outputs from layers 2 and 3
are sent to other cortical areas.
Layers 2 and 3 are hard to distinguish based on simple
histological stains of the cortex.
Functionally, layers 1-3 are often grouped
together and simply called the {\em superficial layers} of the cortex.

Layer 4 has been subdivided into several parts as the
interconnections with other brain
areas and layers have become clarified.
Layer 4C receives the primary input from the
parvocellular and magnocellular layers of the lateral geniculate.
The magnocellular neurons send their output to the upper
half of this layer, which is called 4C$\alpha$ while the
parvocellular neurons make connections in the lower half,
called 4C$\beta$.
Layer 4B receives a large input from 4C$\alpha$ and sends its
output to other cortical areas.
Layer 4B can be defined anatomically by the presence of the
large striation, called the {\em stria of Gennari},
which is composed mainly of cortical axons.

Layer 5 contains relatively few cell bodies compared to the
surrounding layers.
It sends a major output to the superior colliculus, a
structure in the midbrain.
Layer 6 is dense with cells and
sends a large output back to the lateral geniculate
nucleus (Toyoma, 1969).
As a general though not absolute rule,
forward outputs to new cortical areas tend to
come from the superficial layers and terminate in layer 4.
The feedback projections tend to come from the deep layers
and terminate in layers 1 and 6 (Rockland and Pandyaj, 1979;
Felleman and van Essen, 19XX).

The wiring diagram in Figure~\ref{f5:cortexLayers}b shows
that the signals to and from area V1 are complex and
highly specific.
One must suppose that the interconnections within area
V1 are specific, too.
Roughly twenty-five percent of the neurons in all
layers are inhibitory interneurons,
and their interconnections must
be governed by the presence of biochemical markers that identify
which neurons should connect and how.
Anatomical classification of the cell types within the visual
cortex, and identification of the local circuitry,
will provide us with many more clues about the
functional significance of this area.
%Source for 25percent figure is Movshon

\subsection*{The pathway to area V1}

The structure of the anatomical pathways leading from the two retinae
to the cortex defines many of the fundamental properties of area V1.
Among the most significant properties is that area V1 in each
hemisphere has only a restricted field of view.  Area V1 in the left
(right) hemisphere only receives visual input concerning the right
(left) half of the visual field.

We can see how this arises by considering how retinal signals make
their way to area V1.  The optic tract fibers from the two retinae
come together at the {\em optic chiasm}, as shown in
Figure~\ref{f5:cortexInput}.  There the fibers are sorted into two new
groups that each connect to only one side of the brain.  Axons from
ganglion cells whose receptive fields are located in the {\em left
visual field} send their outputs towards the lateral geniculate
nucleus on the right side of the brain, while axons of ganglion cells
with receptive fields in the right visual field communicate their
output to the left side of the brain.  Consequently, each lateral
geniculate nucleus receives a retinal signal derived from both eyes,
but only one half of the visual field.

\begin{figure}
\centerline{
 \psfig{figure=../06cortex/fig/cortexInput.ps ,clip= ,width=5.5in}
}
\caption[Inputs to area V1]{
{\em The signals from the two retinae are communicated to area V1}
via the lateral geniculate nucleus.
Points in the right visual field are imaged on the temporal
side of the left eye and the nasal side of the right eye.
Axons from ganglion cells in these retinal regions
make connections with separate
layers in the left lateral geniculate nucleus.
Neurons in the magnocellular and parvocellular
layers of the lateral geniculate send their outputs
to cortical layers 4C$\alpha$ and 4C$\beta$, respectively.
The signals from each eye are segregated into different
bands within area V1.
Signals from these bands converge on individual neurons in
the superficial layers of the cortex.
%Based on Hubel and Wiesel, 1972, fig. 18  Laminar and Columnar
%distribution of geniculo-cortical fibers in the Macaque Monkey J.
%Comp. Neurology v. 146 no. 4 p. 421-450.
%The projections from the different layers of the lateral geniculate
%nucleus are segregated into different columns at the cortex.
%(From Hubel and Wiesel, 1972).
}
\label{f5:cortexInput}
\end{figure}

The signals reaching the cortex from the retina respect three other
basic organizational principles.  The pattern of interconnections are
organized with respect to (a) the eye of origin, (b) the class of
ganglion cell, and (c) the spatial position of the ganglion cell
within the retina.  Figure \ref{f5:cortexInput} illustrates the
pattern of connections schematically, 
starting at the retinae and continuing to area V1.

\comment{
6 = ispi 5 = contra  4 = ipsi 3 = contra  , Parvo
2 = contra 1 = ipsi  , Magno
}
\paragraph{Eye of origin.  }
Within the lateral geniculate nucleus information about
the eye of origin is preserved since fibers from each
eye make connections in different layers of the lateral
geniculate nucleus.
The parvocellular and magnocellular
layers, which are numbered as 1-6,
receive input from the retina on the
[same,opposite,opposite,same,opposite,same] side of head,
respectively.
The connections of these layers for
the left lateral geniculate nucleus are illustrated in
Figure~\ref{f5:cortexInput}.
Why this particular pattern of ocular connections
exists is a mystery.
The eye-of-origin for the
intercalated layers, which fall between the
parvocellular and magnocellular layers,
has not yet been demonstrated.

The signals from the two eyes remain segregated
as they arrive at the input layers of area V1.
One can observe this segregation by measuring
the electrophysiological responses
of the units in layer 4C.
As the recording electrode travels within layer 4C, 
there is an abrupt shift as to which eye drives the unit.
In layer 4C The shift from one eye to the other takes place 
over a distance of less than 50 $\mu m$.
Above and below layer 4C the signals from the two eyes
converge onto single neurons, although there is still
a tendency for individual neurons to receive inputs
predominantly from one eye or another and this pattern
is aligned with the input pattern.
The transition between eye
of origin is less abrupt in the superficial layers,
perhaps extending over 100 $\mu m$.
The relative segregation of information across
the columns with respect to
the eye of origin is called {\em ocular dominance columns}
(Hubel and Wiesel, 1978; Bishop, 1984).	%Review article by bishop

\begin{figure}
\centerline {
\psfig{figure=../06cortex/fig/ocularDominance.ps,clip=,height=3.5in}
}
\caption[Ocular Dominance Columns]{
{\em The ocular dominance columns in area V1}
can be visualized using a
a radioactive marker, tritiated proline.
When the marker is injected into one eye it is
transported via the lateral geniculate
nucleus to the cortex.
The radioactive uptake is revealed
in this dark field photograph.
The light bands in this
tangential section show the places where
the radioactive marker was located and 
thus reveal the ocular dominance columns.
(Source: Hubel, Wiesel, and Stryker, 1978).
% Anatomical demonstration of orientation columns in macaque monkey.
% J. Comp. Neurol. 177, p. 361-380 1978
% though I took it from Bishop article figure 38, part B.
}
\label{f5:ocularDominance}
\end{figure}
In addition to evidence from electrophysiological measurements,
one also can use anatomical methods to
visualize the ocular dominance columns and demonstrate
their existence.
After injection
into one eye, the  the tritiated amino acid proline
will be transported from the retina to the cortex
across the synaptic connections.
By sectioning the visual cortex tangentially through at layer 4C,
and exposing the section to a photographic emulsion,
we can develop a pattern of light and dark stripes
that correspond to the presence and absence of the tritiated proline.
Figure~\ref{f5:ocularDominance} shows
a pattern of light bands that mark regions
receiving input from the injected eye;
the intervening dark
areas receive input from the opposite eye.
In the monkey these bands each span approximately 400 $\mu m$,
though in the human they span approximately one millimeter
(Hubel et al., 1978;  Horton and Hoyt, 1991).

In the superficial layers of area V1 many neurons
respond to stimuli from both eyes;
in the normal monkey eighty percent of the neurons in 
the superificial layers of area V1 are binocularly driven.
The development of the interconnections necessary to
drive the binocular neurons
depends upon experience during maturation.
Hubel and Wiesel (1965) showed that
artifically closing one eye or cutting an ocular muscle
strongly affects the development
of neurons in area V1.
Specifically, the binocular neurons fail to develop.
Behaviorally, if one eye is kept closed for a
critical period during development, 
the animal will remain blind in this eye for the rest of its life.
This is quite different from the result
of closing an adult eye for a few months;
this has no significant effect
(Hubel, Wiesel and Levay, 1977;
Shatz and Stryker, 1978;
Mitchell, 1988; 
Movshon and van Sluyters, 1981).
In the cat, normal development of ocular dominance columns,
and presumably the binocular interconnections as well,
depends upon neural activity originating in the two
retina (Stryker and Harris, 1986).

\paragraph{Ganglion cell classification.  }
Information from different classes of retinal ganglion cells
remains segregated along the path to the cortex.
Neurons in the magnocellular layers receive fibers
from the parasol cells;
neurons in the parvocellular layers
receive fibers from the midget ganglion cells.
It is uncertain precisely which retinal ganglion cells project
to the intercalated layers.
The segregation of signals continues to
the input of area V1.
Within layer 4C, the upper half (4C$\alpha$) receives the
axons from  the magnocellular layers while the
lower half (4C$\beta)$ receives the parvocellular input.
The neurons in the intercalated layers send their output
to the superficial layers 2 and 3.

\paragraph{Retinotopic organization}
The spatial position of the ganglion cell within the retina is
preserved by the spatial organization of the neurons within the
lateral geniculate nucleus layers.  
The back of the nucleus contains
neurons whose receptive fields are near the fovea.  As we measure
towards the front of the nucleus, the receptive field
locations become increasingly peripheral.  
This spatial layout is called {\em retinotopic} organization
because the topological organization of the
receptive fields in the lateral geniculate parallels the
organization in the retina.

The signals in area V1 are also
retinotopically arranged.
From electrophysiology in monkeys,
one can measure the location of receptive fields with an electrode that
penetrates tangentially through layer 4C, traversing through the
ocular dominance columns.  The receptive field centers of neurons
along this path are located
systematically from the fovea to the periphery.
This trend is interrupted locally by small, abrupt jumps at the ocular
dominance borders.  Within the first ocular dominance column the
receptive field center positions change smoothly; as one passes into
the next ocular dominance region there is an abrupt shift of the
receptive field positions equal to about half of the space spanned by
receptive fields in the first column.  Hubel and Wiesel (1977)
describe this organization and refer to it as ``two steps forward and
one step back.''

In the last fifteen years, it has become possible to estimate
spatially localized activity in the human brain.  Beginning with {\em
positron emission tomography} ({\em PET}) studies, and more recently
by using {\em functional magnetic resonance imaging} ({\em fMRI}), we
can measure activity in volumes of the cortex as small as 10 cubic
millimeters, containing a few hundred thousand neurons\footnote{Both
of these methods are based on indirect measures of neural activation.
With the PET method, an observer receives a low dose of radiation in
his blood stream and neural activity is indicated by brain regions
showing increased radioactivity.  The fMRI signal detects differences
in the local concentration of blood oxygen.  Both the increased
radioactivity and the change in local blood oxygenation are due to
vascular responses to the neural activity (Posner and Raichle, 1994;
Ogawa, 1992; Kwong, 1992).}.

\begin{figure}
\centerline{
 \psfig{figure=../06cortex/fig/calcarine.ps,clip=,height=3.5in}
}
\caption[Human area V1:  The calcarine sulcus]{
{\em Human area V1} is located mainly in the calcarine sulcus,
and in some individuals it may extend onto the occipital pole.
(a) Seen in sagittal view, the calcarine
is a long sulcus that extends roughly 4 cm.
The visual eccentricities of the receptive fields of
neurons at different locations in the calcarine are shown.
(b) In the coronal plane the calcarine sulcus appears
as an indentation of the medial wall of the brain.
At a given distance along the calcarine,
the receptive fields of neurons
fall along a semi-circle within the visual field.
Each hemisphere represents one half of the visual field.
Neurons with receptive fields on the
upper, middle, and lower sections of a semi-circle of
constant eccentricity
are found on the lower, middle and upper portions of the calcarine,
respectively.
}
\label{f5:calcarine}
\end{figure}
Human area V1 is located within the {\em calcarine}
sulcus in the occipital lobe.
The calcarine sulcus in my brain,
and its retinotopic organization,
is shown in Figure~\ref{f5:calcarine}.
Neurons with receptive
fields in the central visual field are located in the posterior
calcarine sulcus, while neurons with receptive fields in the periphery are
located in the anterior portions of the sulcus.  
At a given distance along the sulcus, the receptive fields are located
along a semi-circle in the visual field.  Neurons with receptive
fields on the upper, middle, and lower sections of the semi-circle,
are found on the lower, middle and upper portions of the calcarine,
respectively.  (Holmes, 1917, 1945; Horton and Hoyt, 1991;
Inouye, 1909)

\begin{figure}
\centerline{
 \psfig{figure=../06cortex/fig/retinotopy.ps,clip= ,width=5.5in}
}
\caption[Retinotopic organization of human area V1]{
{\em Receptive field locations of neurons in human calcarine sulcus}
can be measured by functional magnetic resonance imaging.
(a) The observer viewed a series of concentric expanding annuli
presented on a gray background.
Each annulus contained a
high contrast flickering radial checkerboard pattern.
As an annulus expanded beyond the edge
of the display, a new annulus emerged in the center
creating a periodic image sequence.
The sequence was repeated four times in a single experiment.
(b) An image within the plane of the calcarine sulcus.
The dark lines indicate points identified
as following the left calcarine sulcus.
(c) The fMRI temporal signal at different
points within the calcarine sulcus.
The fMRI signal follows the timecourse of
the stimulus;
the phase of the signal is delayed
as we measure from the posterior to the anterior
calcarine sulcus.
}
\label{f5:retinotopy}
\end{figure}
Engel et al. (1994) measured the human retinotopic organization from
fovea to periphery by using the stimulus 
shown in Figure~\ref{f5:retinotopy}a.
The stimulus consisted of a series of slowly expanding rings;
each ring was a collection of flickering squares.
The ring began as a small spot located at the fixation mark,
and then it grew until it traveled beyond
the edge of the visual field.
As a ring faded from view, it was replaced by
a new ring starting at the center.
Because of the retinotopic organization of the calcarine,
each ring causes a traveling wave of neural
activity beginning in the posterior calcarine
and traveling in the anterior direction.

We can detect the traveling
wave of activation by measuring the fMRI signal 
at different points along the calcarine sulcus.
Figure~\ref{f5:retinotopy}b is an image
of the brain within the plane of the calcarine sulcus.
Positions within the calcarine
sulcus are highlighted in black.
The fMRI signal at each point within the sulcus,
plotted as a function of time, is shown in the
mesh plot in Figure~\ref{f5:retinotopy}c.
Notice that the amplitude of the fMRI
signal covaries with the stimulus;
the fMRI signal waxes and wanes four times
through the four periods of the expanding annulus.
The temporal phase of the fMRI signal varies systematically
from the posterior to anterior portions of the sulcus:
Activity in the posterior portion of the sulcus
is advanced in time compared to activity
in the anterior portion.
This traveling wave occurs because 
the stimulus creates activity in the posterior
part of the sulcus first, and then later in the
anterior part of the sulcus.

\begin{figure}
\centerline{
 \psfig{figure=../06cortex/fig/cortMag.ps,clip= ,width=5.5in}
}
\caption[Cortical Magnification]{
{\em Several methods have been used
to estimate the receptive field location of neurons in the calcarine sulcus}.
The filled symbols show measurements from two observers using
the fMRI method (Engel et al., 1994).
The squares are from a microstimulation study
on a blind volunteer (Dobelle, 1978).
The diamonds are measurements averaged
from 5 observers using PET (Fox et al., 1984).
The dashed curve is an estimate based on
studying the locations of scotoma in stroke patients
and single-cell data from non-human primates (Horton and Hoyt, 1991).
(Source: Engel et al., 1994) }
%Bill suggests color for this figure
\label{f5:cortMag}
\end{figure}
In addition to fMRI,
there are several other estimates of the mapping from
visual field eccentricity to location
in the calcarine sulcus.
These estimates are compared in Figure~\ref{f5:cortMag}.  
The fMRI measurements from two observers are shown as the filled circles.
Estimates from direct electrical stimulation of the cortex
are shown as gray squares (Dobelle et al., 1978).
In these experiments
the volunteer observer's brain was stimulated and he
indicated the
location of the perceived visual stimulation within the visual field
(see also Brindley and Lewin, 1968).  
The three gray diamonds are show measurements using PET.
These data represent the average
of five different observers, normalized for differences
in brain size.
The dashed line shows an
estimate by Horton and Hoyt (1991)
by studying the positions of scotoma in observers
with localized brain lesions and extrapolating from monkey.
These estimates are in good agreement,
and they all show that
considerably more cortical area is allocated to the foveal
representation than to the peripheral representation.

The allocation of more cortical area to
the foveal than the peripheral representation 
seems a natural consequence of
the fact that more photoreceptors and
retinal ganglion cells represent
the fovea than the periphery.
W\"assle et al. (1990; see also Schein, 1988) suggested
that the expanded foveal representation can be explained
by assuming that every ganglion cell is allocated an equal
amount of cortical area.
More recently,
Azzopardi and Cowey (1993) suggest that there is
a further expansion of the foveal representation,
and that foveal ganglion cells
are allocated three to six times more cortical area
than peripheral ganglion cells.

\subsection*{Electrical stimulation of Human Area V1}
Direct electrical stimulation of the visual cortex causes the
sensation of vision.  When a visual impression is generated by
non-photic stimulation, say by pressing on the eyeball or by
electrical stimulation, the resulting perception is called a {\em
visual phosphene}.  In order to develop visual prostheses for
individuals with incurable retinal diseases, several research groups have
studied the visual properties of phosphenes created by electrical
stimulation of the visual cortex (Brindley and Lewin, 1968; Dobelle,
et al., 1978; Bak et al., 1990).

Brindley and Lewin (1968) describe experiments with a human volunteer
who was diabetic and suffered from bi-lateral glaucoma, a right
retinal detachment, and was effectively blind.  When she suffered a
stroke, she required an operation that would expose her visual cortex.
With the patient's consent, Brindley and Lewin built and implanted
a stimulator that could deliver current to the surface of her brain,
near the patient's primary visual cortex.  They asked her to describe
the appearance of the electrical stimulation following stimulation by
the different electrodes, at various positions within her primary
visual cortex.  She reported that electrical stimulation caused her to
perceive a phosphene that appeared to be a point of light or a blob in
space.  Her description of the visual impression caused by most of the
electrodes was ``like a grain of rice at arm's length.''  Occasionally
one electrode might cause a slightly longer impression, ``like half a
matchstick at arm's length.''

As might be expected from the retinotopic organization of the visual
cortex, the position of the phosphenes varied with the position of the
stimulating electrodes.  The observer told the experimenters where she
perceived the phosphenes to be using a simple procedure.  She grasped
a knob with her right hand and imagined she was fixating on that hand.
She then pointed to the location of the phosphenes relative to the
fixation point using her left hand.

\begin{figure}
\centerline {
\psfig{figure=../06cortex/fig/brindleyLewin.ps,clip=,height=3.5in}
}
\caption[Cortical Stimulation Experiments]{
{\em Electrical stimulation of human area V1} using chronically
implanted microelectrodes reveals the retinotopic organization of
human cortex.  The symbols are plotted at the electrode positions on
the medial wall of the brain.  The shading of the symbol indicates the
visual eccentricity of the phosphene created by electrode
stimulation.  A dot within the symbol means that the phosphene was
perceived in the upper visual field.  The dashed curve shows the inferred
position of the calcarine sulcus (Source: Brindley and Lewin, 1968).
%Redrawn from figs 1 and 3 in b and l, 1968
}
\label{f5:brindley.lewin}
\end{figure}
Figure \ref{f5:brindley.lewin} shows the positions of the electrodes
and the corresponding phosphenes.  The pattern of results follows the
expectations from the retinotopic organization of the calcarine
sulcus.  Stimulation by electrodes near the back of the brain created
phosphenes in the central five degrees; stimulation by forward
electrodes created phosphenes in more eccentric portions.  More
cortical area is devoted to the central than peripheral regions of
vision.

Brindley and Lewin tested the effects of superposition by stimulating
with separate electrodes and then stimulating with both electrodes at
once.  When electrodes were far apart, the visual phosphene generated
by stimulating both electrodes at once could be predicted from the
phosphenes generated by stimulating individually.  Superposition also
held for some closely spaced electrodes, but not all.  The test of
superposition is particularly important for practical development of a
prosthetic device.  To build up complex visual patterns from
stimulation of V1, it is necessary to use multiple electrodes.  If
linearity holds, then we can measure the appearance from single
electrode stimulations and predict the appearance to multiple
stimulations.  That superposition held approximately suggests that it
may be possible to predict the appearance of the multiple electrode
stimulation from measurements using individual electrodes.  Without
superposition, we have no logical basis for creating a an image from
the intensity at a set of single points.

There have been a few recent reports of stimulation of the human
visual cortex.  For example, Bak et al., (1990) stimulated using very
fine microelectrodes ($37.5 \mu m$) inserted within the cortex during
to stimulate visual percepts.  They experimented on patients who were
having epileptic foci removed.  These patients were under local
anaesthesia and could report on their visual sensations.  Bak et
al. observed that when the stimulation was embedded within the visual cortex,
visual sensations could be
obtained with quite low current levels.
Brindley and Lewin used about 2 mA of current, but Bak et al. found
thresholds about 100 times lower, near 20 $\mu A$.  The appearance of
the visual phosphene was steady in these patients, and some of them
appeared colored.  Time was quite limited in these studies and only a
few experimental manipulations were possible.  But, they report that
when the microelectrodes were separated by more than 0.7mm, the two
phosphenes could be seen as distinct, while separations of 0.3mm were
seen as a single spot.  For one subject, nearly all of the phosphenes
were reported to be strongly colored, unlike the phosphenes reported
by Brindley and Lewin's patient.  While the subjects were stimulated,
they could also perceive light stimuli.  The phosphenes were visible
against the backdrop of the normal visual field.

