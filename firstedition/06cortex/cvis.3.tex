\section{Visual Streams in the Cortex}

We have reviewed two major principles that characterize
the flow of information from retina to cortex.
First, visual information is organized into separate
visual streams.
These streams begin in the retina and continue
along separate neural pathways into the brain.
Second, 
the receptive field properties of
neurons become progressively more sophisticated.
Receptive fields of
cortical neurons show selective responses
to stimulus properties that are more complex than
retinal neurons.
The new receptive field properties are clues
about the specialization of the
computations performed within the visual cortex.

As we study visual processing within the cortex
we should expect to see both of these principles extended.
First, we should expect to find
new visual streams that play a role in the cortical
computations.
Some new visual streams will arise in visual cortex,
and some, like the rod pathway in the retina,
will have served their purpose and merge with other streams.
Second, as we explore the cortex
we should expect to find neurons
with new receptive field properties.
We will need to characterize these receptive fields
adequately in order to understand their computational
role in vision.

Our understanding of cortical visual areas is in 
an early and exciting phase of scientific study.
In this section, we will review some of the basic
organizational principles of the cortical areas.
In particular, we will review how information from
area V1 is distributed to other cortical areas
and we will review the experimental and logical methods
that relate activity
within these cortical areas to what we see.
We will review some of the more
recent data and speculative theories
in Chapters~\ref{chapter:Color} and \ref{chapter:Motion}.

\subsection*{The fate of the parvocellular and magnocellular pathways}
The segregation of visual information into separate streams
is an important organizing principle of neural representation.
Two of the best understood streams
are the magnocellular and parvocellular
pathways whose axons terminate in layers 4C$\alpha$ and 4C$\beta$
within area V1.
What happens to the signals from these pathways within the visual cortex?

Along one branch, signals from the magnocellular
pathway continue from area V1 directly
to a distinct cortical area.
The magnocellular pathway
in layer 4C$\alpha$ makes a connection to
neurons in layer 4B where there are many direction
selective neurons.
These neurons then send a strong
projection to cortical area MT (medial temporal).
It seems reasonable to suppose, then, that the information
contained within the magnocellular stream is of particular
relevance for the visual processing in area MT.
As we saw in Chapter~\ref{chapter:retina}, the magnocellular
pathway has particularly good information about the high
temporal frequency components of the image.
Earlier in this chapter we saw
that neurons in layer 4B show strong direction selectivity,
as do the neurons in area MT (Zeki, 1974).
Taken together, these observations have led to the hypothesis
that area MT plays a role in motion perception.
We will discuss this point more fully in Chapter~\ref{chapter:Motion}.

While one branch of the magnocellular stream continues on an
independent path,
another branch of this stream converges with the parvocellular pathway
in the superficial layers of area V1.
Maleplli, et al. (1981) and Nealey and Maunsell (1994)
made physiological measurements demonstrating
that signals from the parvocellular
and magnocellular streams converge on individual neurons.
In these experiments parvocellular or magnocellular signals
were blocked either by application of a local
anaesthetic (Malpelli et al., 1981; lidocaine hydrochloride)
or GABA (Nealey and Maunsell, 1994)
to small regions of the lateral geniculate nucleus.
Both studies report instances of neurons whose responses
are influenced by both parvocellular and magnocellular
blocking.
Anatomical paths for this signal have also been identified.
Lachica, Beck and Cassagrande (1992)
injected retrograde anatomical tracers into the superficial
layers of visual cortex,
that is tracers that are carried from
the injection site towards the inputs to the injection site.
They concluded that the magnocellular and parvocellular
neurons contribute inputs
into overlapping regions within
the superficial layers of the visual cortex.
Hence, these anatomical
pathways could be the route for the physiological signals.

Just as the rod pathways are segregated for a time,
and then they merge with the cone pathways, so too 
signals from the
magnocellular stream merge with parvocellular signals.
The purpose of the peripheral segregation of
the parvocellular and magnocellular signals, then,
may be to communicate rapidly certain
type of image information to area MT.
After the signal has been efficiently communicated,
the same information may
be used by other cortical areas, in combination
with information from the parvocellular pathways.

\subsection*{Cytochrome Oxidase Staining}
Livingstone and Hubel (1981, 1984, 1987, 1988) have
argued that several new visual streams begin
in area V1.
Their argument begins with a discovery made by Wong-Riley (1978)
who used a histochemical marker to detect the presence of
an enzyme called {\em cytochrome oxidase}.
She found that cytochrome oxidase
is present in a continuous and heavy
pattern in layer 4,
but that the enzyme is present mainly at a set of patches
in the superficial and deeper layers.
These patches can be seen
by staining for cytochrome oxidase and
viewing the superficial layers in tangential section.
In that case, the regions of high cytochrome oxidase density
appears as a set of darkened spots (see Figure~\ref{f5:blobs.vs.od}).
These darkened regions are called variously
``puffs'', ``blobs'' or ``CO rich'' areas.
These puffs are visible in several, but not all,
primates species, including the human.
Cytochrome-oxidase density
is correlated with regions of high neuronal activity,
though the enzyme itself
is not known to have any specific significance for visual function.
(Humphrey and Hendrickson, 1980;
Wong-Riley, 1978; Hendrickson, 1985;
Livingstone and Hubel, 1981).

\begin{figure}
\centerline {
\psfig{figure=../06cortex/fig/blobs.vs.od.ps,clip=,width=5.5in}
}
\caption[Cytochrome Oxidase Staining and Ocular Dominance]{
{\em Cytochrome oxidase puffs are located 
in the middle of ocular dominance columns.} 
The three panels show tangential sections of area V1.
(a) The locations of the ocular dominance columns in
layer 4C correspond to the light and dark patterns.
The columns were identified
by a monocular injection of the radioactive
tracer tritiated proline.
(b) The puffs are the darkened spots
identified by a high concentration of cytochrome oxidase.
The image is a tangential section in the superficial layers.
(c) An overlay of (a) and (b),
placed in register by comparing the positions
of larger blood vessels.
shows that the puffs are located in
the center of the ocular dominance columns.
(Source: Horton and Hubel, 1981).
%Fig. 2  Regular patchy distrubtion of cytochrome oxidase staining in
%primary visual cortex of macaque monkey.  Nature, v. 292 p.762-747
}
\label{f5:blobs.vs.od}
\end{figure}
Several lines of evidence suggest that
cytochrome oxidase labeling is correlated with
the presence of new visual streams in the cortex.
First, Horton and Hubel (1981, 1982) discovered that
the cytochrome-oxidase staining pattern is related to
at least one aspect of visual function:
the ocular dominance columns.
Figure~\ref{f5:blobs.vs.od}a shows the ocular dominance columns
measured by injecting one eye of a monkey with a radioactive
tracer, tritiated proline.
The section is through layer 4C
(and a little bit of layer 5) where the ocular
dominance columns are best segregated.
The data in this figure replicate the demonstration of ocular
dominance columns described near Figure~\ref{f5:ocularDominance}.
Figure~\ref{f5:blobs.vs.od}b shows the puffs in a tangential
section from layers 2 and 3, just above the region
shown in Figure~\ref{f5:blobs.vs.od}a.
Figure~\ref{f5:blobs.vs.od}c shows an overlay
of the ocular dominance columns and the puffs.
The images were placed into spatial registration 
by finding blood vessels common to the two images
to align their positions.
Figure~\ref{f5:blobs.vs.od}c demonstrates that the puffs trace a path
down the center of the ocular dominance columns.

The second suggestion that the presence of cytochrome
oxidase identifies new visual streams comes from
studying the connectivity between neurons within area V1.
Burkhalter and Bernardo (1989) report that
the neurons within the V1 puffs in human are connected to
other neurons in the puffs, while neurons
between the puffs are connected to other neurons
between the puffs (Rockland, 1985)

A third piece of evidence is the relationship between
the cytochrome oxidase puffs and connections
to other cortical areas.
Like area V1, cytochrome oxidase is distributed
unevenly in a second cortical area, V2, 
adjacent to area V1.
In area V2, staining for cytochrome oxidase yields
a regular three striped pattern defining
regions that stain to different degrees.
The three types of stripes are
labeled {\em thick stripes, thin stripes} and {\em interstripes}.
The thick and thin stripes contain more
cytochrome oxidase than the interstripe regions.
This pattern of stripes is visible
in human area V2 as well as some species of monkey.

Livingstone and Hubel (1984, 1987a) demonstrated the
specificity of these interconnections.
Regions with a high density of cytochrome oxidase in area V1,
that is regions with high metabolic activity,
are connected to high density regions in area V2.
Neurons in layer 4B of area V1 send outputs to the thick stripes of
area V2.
Neurons in the puffs send outputs to the thin strips.
Neurons in between the puffs send outputs to the interstripes.

Finally, the pattern of cytochrome oxidase staining of area V2
correlates with the connections from area V2 to other
visual areas.
In the monkey, visual area MT receives a strong
projection from the thick stripes in area V2.
Area V4 receives signals mainly from the thin stripes
and the interstripes (DeYoe and van Essen, 1988;
Merigan and Maunsell, 1993).
These measurements suggest that there is considerable
organization of the signals communicated
within individual cortical areas.

\subsection*{Areas Central to Primary Cortex}
Figure~\ref{f5:extraStriate} shows a few of the
many cortical visual areas.
Often, the connections between cortical areas are reciprocal;
ascending axons  make connections with the primary input layer, 4C,
and descending axons make connections in layers 1 and 6.
A single visual area can have connections with several
other cortical areas (Rockland and Pandya, 1979;
Felleman and Van Essen, 1991).

Figure~\ref{f5:extraStriate} is arranged to emphasize
one aspect of the segregation of visual information within
cortex: namely,
that information from the visual areas in the occipital lobe
separate into two visual streams.
One stream sends its outputs mainly to the posterior
portion of parietal lobe,
and the second stream makes its connection
mainly in the inferior portion of the temporal lobe.
\begin{figure}
\centerline{
 \psfig{figure=../06cortex/fig/extraStriate.ps,clip= ,width=5.5in}
}
\caption[Visual Cortex Central to V1]{
{\em An overview of the organization of visual areas}
is shown.
The signal from the occipital lobe areas,
are sent along two major streams.
One stream passes into
the posterior parietal lobe and a second to the inferior temporal
lobe.
The visual signals in the temporal and parietal lobe areas
appear to arise from different neurons
within the occipital lobe.
Abbreviations: MST = medial superior temporal area, 
MT  = medial temporal area, 
VIP = ventral intraparietal area, 
PIT = posterior inferotemporal area
LIP = lateral intraparietal area
CIT = central inferotemporal area.
(Source:  Merigan and Maunsell, 1993).
}
\label{f5:extraStriate}
\end{figure}

There are several known
connections between these two streams,
but a study by Baizer et al. (1991) shows
that the segregation is impressive.
These authors injected large amounts of
two retrograde tracers into individual monkey brains:
one tracer was injected
into the posterior parietal and the other into
the inferior temporal cortex.
They examined where the two types of tracers could be found
in visual areas within the occipital lobe, including areas
V1, V2, V4 and MT.
They report finding almost no neurons that contained both tracers,
suggesting that the signals from individual neurons are communicated
mainly to either the parietal or temporal lobes.
Baizer et al. (1991) report that
neurons in the parietal lobe received information
mainly from neurons with receptive fields located in the
periphery, while neurons in the temporal lobe received information 
mainly from neurons with receptive fields located near the fovea.

Baizer et al.'s (1991) observations support
Ungerleider and Mishkin (1982) proposal that
the parietal and temporal
streams serve different visual functions.
Ungerleider and Mishkin observed that clinical damage within the 
parietal stream of one hemisphere causes difficulties in visual
and motor orienting.
It also causes {\em hemineglect}, a condition in which
the observer appears to be unaware of stimulation arising in
the hemifield that projects to that parietal lobe.
Patients also have trouble orienting towards or reaching
for objects in the visual periphery.
The clinical symptoms associated with
damage to the temporal stream are
quite different.
In this case, patients have impaired form discrimination
or recognition.
They also have problems with visual memory.
These behavioral deficits are very different,
so that neurologists have
supposed that the visual function of parietal and temporal
lobes are quite different.
One brief characterization of the distinction is this:
the parietal system defines {\em where}, while the
temporal system defines {\em what} 
(Ungerleider and Mishkin, 1982; Merigan and Maunsell, 1993).

The anatomical segregation of the neural signals
coming from the occipital lobe into these two streams
shows that the temporal and parietal areas receive different
information about the visual image.
Hence, it seems likely that the computations in these two
portions of the brain must serve different goals though
a more refined analysis of these differences will be helpful.

\section{Cortical Representations and Perception}
The brilliant visual scientist
William Rushton enjoyed needling his colleagues.
On one occasion, he challenged neuroscientists with the
assertion that only the hope of
understanding perception and consciousness
makes neuroscience worth doing.
Much of the work I have reviewed
was inspired by the desire to understand conscious perception. 
In this sense, some neuroscientists think of themselves
as philosophers studying the mind-body problem.
They form a field one might call experimental philosophy;
it is the expensive branch of philosophy.

More recently Crick (1993) has taken up Rushton's call
and pressed us to consider the question:
what aspect of the cortical
response corresponds to conscious experience?
Since neuroscientists nearly all make the assumption that
consciousness is a correlate of cortical activity,
Crick points out that identifying the
relationship between consciousness and neural activity
is properly an {\em experimental} question.

What experimental tests
might we perform to answer questions about the relationship between
our conscious awareness and the activity of our brain?
One way to study this question is to compare the information available
within a cortical area with the visual experience we have.  An
interesting example of an experiment concerning
consciousness is the comparison of conscious awareness with
the information available in area V1.  
From the anatomy and physiology of V1, we have learned that
there is plenty of information in V1
that we can use to deduce which eye is the source of a
visual signal.  
Entirely different sets of neurons, confined to the
ocular dominance columns, respond depending on which eye sends the
signal.  
Is this information available to us?

We can answer this question experimentally
by asking subjects to
discriminate between visual signals originating in the right and left
eyes.
If we can accurately decide on the eye-of-origin,
then we might conclude that
the information in area V1 is part of our conscious
experience.
If we cannot, then we should conclude
that information within layer 4C can be lost
prior to reaching our conscious experience.

Notice that eye-of-origin information is certainly available to us
unconsciously.
For example, Helmholtz (1866) pointed out
that eye-of-origin information is necessary
and used for the computation of stereopsis.  The question, therefore,
is not whether the information is present but whether it is accessible
to conscious experience.  Based on his own introspections,
Helmholtz answered the question in the negative.  
And, while there have been occasional reports that some
discriminations are possible, Ono and Barbeito's (1985) 
careful experiments suggest that no reliable eye-of-origin discriminations
are possible.
Hence, the
massive information available in the input layers of area V1 (and
earlier) about the eye-of-origin is not part of our conscious
experience.
Through this negative result, we have made a small
amount of progress in localizing consciousness.

\subsection*{The function of the visual areas}
Even when the computation performed in a visual area is not
part of our conscious experience, we would still like to
know what the area does.
Over the last fifteen years,
there have been a broad variety of hypotheses concerning the
perceptual significance of the cortical areas.
Mainly, we have seen a flurry of proposals suggesting
that individual visual areas are responsible for
the computation of specific perceptual features,
such as color, stereo, and form and so forth.

\nocite{Barlow1972}
What is the logical and experimental basis for reasoning about the
perceptual significance of visual areas?
Horace Barlow (1972) has set forth one specific doctrine
to relate neurons to perception,
the {\em neuron doctrine}.
This doctrine asserts that
{\em a neuron's receptive field describes the
percept caused by excitation of the neuron.}
You will see the idea expressed many times
as you read through the primary
literature and study how investigators
interpret the perceptual significance of neural responses.

Our understanding of the peripheral representation
lends little support to the neuron doctrine.
For example, the principle does not serve us well
when analyzing color appearance.
In that case,
we know with some certainty that a large response from an $\Red$
photoreceptor does not imply that the observer will
perceive red at the corresponding location in the visual
field.
Rather, the color appearance depends upon stimulation at
many adjacent points of the retina.
The conditions for a red percept include a pattern of
peripheral neural responses, including more $\Red$ and less
$\Green$.
Data from the periphery is generally more consistent with
the notion of a distributed representation in which an
experience depends on the response of a collection of neurons.

Oddly, the failure of the neuron doctrine in the periphery,
is often used to support the neuron doctrine.
After all, the argument goes,
the periphery is not the site of our conscious awareness.
so failures of the doctrine in the periphery are to be expected.
The neuron doctrine's significance depends on the idea
that there will be a special place,
probably located in the cortex,
where the receptive fields of a neuron
predicts conscious experience when that neuron is active.
This location in the brain should only exist at a point after
the perceptual computations needed to see features
we perceive --- color, form, depth --- have taken place.

In the past, secondary texts sometimes
used Hubel and Wiesel's work in area V1
as a location where the neuron doctrine might hold.
The receptive fields in area V1 seem
like basic perceptual features;
orientation, motion selectivity, binocularity, complex cells,
all emerge for the first time in area V1~\footnote{
See Hubel's Nobel lecture for a marvelous
description of the paradigm prior to their work.
}.
Consequently,
secondary texts often described the receptive fields in area V1
as a theory of vision, with the receptive fields defining
salient perceptual features.
The logical basis for this connection between V1 receptive fields
and visual features is the neuron doctrine.

By 1979 the significance of the 
other cortical areas had become undeniable
(Zeki, 1974, 1978; Felleman and Van Essen, 1991; van Essen et al., 1992).
In reviewing the visual pathways, Hubel and Wiesel wrote
\begin{quote}
The lateral geniculate cells in turn send their
axons directly to the primary visual cortex.
From there, after several synapses, the messages are
sent to a number of further destinations:  neighboring cortical areas
and also several targets deep in the brain.
One contingent even projects back to the lateral geniculate bodies;
the function of this feedback path is not known.
The main point for the moment is that the primary visual
cortex is in no sense the end of the visual path.
It is just one stage, probably an early one in terms of the degree of
abstraction of the information it handles.  (Hubel and Wiesel, 1979).
\end{quote}

Acknowledging this point leads one to ask what 
is the function of these cortical areas.
The answer to this question
has relied, mainly, on the neuron doctrine.
For example, when Zeki (1980; 1983; 1993) found that
color contrast was a particularly effective
stimulus in area V4,
he argued that this area is responsible for color perception.
Since movement was particularly effective in 
stimulating neurons in area MT,
that become the motion area (Dubner and Zeki, 1971).
The logic of the neuron doctrine permits one to
interpret receptive field properties in terms of
perceptual function.

Among the most vigorous application of the neuron
doctrine is contained in articles by
Livingstone and Hubel (1984, 1987, 1988).
They supported Zeki's basic view and added new hypotheses of their own.
Their hypothesis, which continues to evolve,
is summarized in the elaborate
anatomical/perceptual diagram
shown in Figure \ref{f5:specialization}.
In this diagram
anatomical connections in visual cortex are
labeled with perceptual tags, including color, motion, and
form.
The logical basis for associating perceptual tags
with these anatomical streams is the neuron doctrine.
Receptive fields of neurons in one stream were orientation selective,
hence the stream was tagged with form perception.
Neurons in a different stream were motion selective
and hence the stream was tagged with motion perception.
%8 inches high, 10 inches wide
\begin{figure}
\centerline{
\psfig{figure=../06cortex/fig/specialization.ps,clip= ,width=5.5in}
}
\caption[Functional Specialization]{
{\em An anatomical-perceptual model of the visual cortex.}
In this speculative model,
visual streams within the cortex are identified with
specific perceptual features.
The anatomical streams are identified using anatomical markers;
the perceptual properties are associated with the streams
by applying the neuron doctrine
(Source: Livingstone and Hubel, 1988).
%Segregation of form, color, movement and depth:  anatomy, physiology
%and perception.  Science, v. 240, p. 740-750.
}
\label{f5:specialization}
\end{figure}

The perceptual-anatomical
hypotheses proposed by Zeki and Livingstone and Hubel
define a new view of cortex.
On this view, the relationship between cortical neurons and
perception should be made at the level of perceptual features.
These investigators did not study the computation within the
neural streams, but rather, like tailors labeling a suit, they
summarized what they felt were the main features of the pathway
(see Hubel and Wiesel, 1977 for a description of this approach).

The use of the neuron doctrine to interpret brain function
is very widespread, but
there is very little evidence in direct support of the doctrine
(Martin, 1992).
The main virtue of the hypothesis
is the absence of an articulated alternative.
The most frequently cited alternative is
the proposal that perceptual experience
is represented by the activity of many neurons,
so that no individual neuron's response corresponds to
a conscious perceptual event.
These types of models are often called
{\em distributed} processing models;
they are not widely used by neurophysiologists
since they do not provide the specific guidance
for interpreting experimental measurements from single neurons,
the neurophysiologist's stock-in-trade.
The neuron doctrine, on the other hand, provides
an immediate answer.

In my own thinking about brain function, I am more inclined to
wonder about the brain's computational methods
than the mapping between perceptual features and 
tentatively identified visual streams.
I find it satisfying to learn that the magnocellular pathway
contains the best representation of high temporal frequencies,
but less satisfying to summarize the pathway as the motion pathway
since this information may also be used in many other types of
performance tasks.
The questions I find fundamental concerning computation
are how, not where.
How are essential signal processing tasks,
such as multiplication, addition and signal synchronization,
carried out by the cortical circuitry?
What means are used to store temporary results, and what
means are used to represent the final results of computations?
What decision mechanisms are used to route information from one
place to another?

My advice, then,
as you read and think about brain function is this:
Don't be distracted by the neuron doctrine or its application.
The doctrine is widely used because it is an easy tool to relate
perception and brain function.
But, the doctrine distracts us from
the most important question about visual function:
how do we {\em compute} perceptual features like color, stereo and form?
Even if it turns out that a neuron's receptive field is predictive
of experience, the question we should be asking is how the
neuron's receptive field properties arise.
Answering these computational questions will help us
most in designing practical applications
that range from sensory prostheses to robotics applications.
We should view the specific structures within
the visual pathways as a means of implementing these principles,
rather than as having an intrinsic importance.

Hubel and Wiesel once expressed something like this view.
While reviewing their accomplishments in the study of area V1,
they wrote:
\begin{quote}
What happens beyond the primary visual area, and how is the
information on orientation exploited at later stages?  Is one to
imagine ultimately finding a cell that responds specifically to some
very particular item?  (Usually one's grandmother is selected as the
particular item, for reasons that escape us.)  Our answer is that we
doubt there is such a cell, but we have no good alternative to offer.
To speculate broadly on how the brain may work is fortunately not the
only course open to investigators.  To explore the brain is more fun
and seems to be more profitable.

There was a time, not so long ago, when one looked at the millions of
neurons in the various layers of the cortex and wondered if anyone
would ever have any idea of their function.  Did they all work in
parallel, like the cells of the liver or the kidney, achieving their
objectives by pure bulk, or where they each doing something special?
For the visual cortex the answer seems now to be known in broad
outline:  Particular stimuli turn neurons on or off;  groups of
neurons do indeed perform particular transformations.  It seems
reasonable to think that if the secrets of a few regions such as this
one can be unlocked, other regions will also in time give up their secrets.
[ibid., p. 23].
\end{quote}

In the remaining chapters, we will see how other areas of
vision science, based on behavioral and computational studies,
might help us to unlock the secrets of vision.
