%
%	References go here
%
\comment{
Nothing on visual development, ocular dominance development,
and so forth.
Do this in the questions by reference and so forth.
}

\nocite{Wong-Riley1984}
\nocite{Hubel1972a,Hubel1972b,Hubel1974,Hubel1974b}
\nocite{Wiesel1966}
\nocite{Devalois1982b,DeValois1988}
\nocite{Movshon1978a,Movshon1978b,Movshon1985}
\nocite{Britten1992}
\nocite{Brindley1968}
\nocite{Bak1990}
\nocite{Poggio1984a}
\nocite{Albright1984}
\nocite{Freeman1990}
\nocite{Tolhurst1983}
\nocite{Parker1985}
\nocite{Hawken1990}
\nocite{Vogels1990}
\nocite{Salzman1992}
\nocite{Livingstone1984,Livingstone1987}
\nocite{Martin1988}  %From enzymes to visual perception:  a bridge too for?
\nocite{Zeki1971,Zeki1983c,Zeki1990Zeki1991}

\nocite{Barlow1972} %Neuron Doctrine

\newpage
\section*{Exercises}

\be %All Questions

%	Add in cortical development from Hubel and Wiesel
%	Define Amblyopia here

\item Answer these questions about forming
cortical receptive fields from the inputs of lateral geniculate neurons.

 \be

 \item What is an orientation column?

 \item Suppose the cortical cell response is the
sum of two geniculate neurons outputs.
Describe the lateral geniculate neurons' receptive fields
needed for the cortical cell receptive field to be circularly symmetric?

 \item Draw a matrix tableau
to describe how the receptive field of a linear
cortical neuron depends on the linear receptive fields
of a collection of lateral geniculate neurons.

 \item  Use the matrix tableau
to design a weighted combination of lateral geniculate
responses to assign how much each lateral geniculate
neuron's response should contribute to the
cortical cell's response in order to achieve
different receptive field functions.

 \ee

\item Answer these questions about retinotopic organization.

 \be

 \item What is a retinotopic map?

 \item Certain neurons in the central nervous
system have very large receptive fields, 
spanning 20 degrees of visual angle.
Can visual areas containing such neurons have retinotopic maps?
Experimentally, how would you convince yourself
such an area had a retinotopic organization?

 \item Draw a picture describing the logical organization
of an area of visual cortex that is retinotopic and also
has orientation columns.
Are the two types of organization necessarily linked
to one another, or can you imagine that they would
each follow their own independent layout?

 \item How many different kinds of organization can be
simultaneously superimposed within a single cortical region?

 \ee

\item Lesion studies have been an important source of information
about neural function.
The logical foundation of lesion studies, however, is quite
involved.
Often, it is difficult to be precise about the conclusions
one may draw from a lesion study.
The problems of interpreting lesion studies can be
illustrated by this old joke.

A scientist once decided to study binocular vision.
He removed the right eye of a cat an observed that
the cat lost its stereo vision.
He then wrote an article describing how stereovision
was localized to the write eye.

To follow on his groundbreaking work, this scientist
studied the animal's behavior again and noted that
the animal could perform monocular tasks.
He enucleated the left eye and observed that the animal
could no longer perform such tasks.
The scientist wrote a second paper describing how monocular vision
is localized to the left eye.

Write a paragraph that describes what conclusions the
scientists should have drawn.
Choose your words carefully.

\item Answer these questions about color coding.

 \be

 \item How would you measure the wavelength responsivity
of a cortical neuron?

\item Recall the definition of separability for space and time
in the receptive fields of neurons from Chapter~\ref{chapter:retina}.
Make an analogous definition for wavelength-space separability.

 \item Suppose you measure the wavelength encoding of four
neurons in the cortex.
Recall that the  initial encoding of wavelength  is based on
three types of cones.
Do the wavelength encoding properties of any two of the neurons
have to be precisely the same?

\item If none of the four neurons
have precisely the same, will
this have any implications for trichromacy?

 \item Suppose the neurons respond linearly to wavelength mixtures.
Describe how you might be able to infer the connections
between the neurons and the three difference cone classes.

 \ee

\item Development of the visual system.
I have been horribly remiss by failing
to educate you about the development of the visual system.
Go to the library, now, and read the chapter
Hubel's book on visual development.

\item {\em Functional specialization} refers to the notion
that different brain areas are specialized for certain
visual functions, such as the perception of motion or color.

 \be

 \item What interconnections would you anticipate
to find between different brain areas if each one
is specialized to represent a different perceptual
function?

 \item What properties do you think a neuron should
exhibit before we believe it is specialized for a visual function?

 \item If we are in the mood of assigning functional specialization,
how about the lateral geniculate nucleus?
What special role might it play?
Could it be connected with eye movements?  Saccadic
suppression?  Attentional gating? Synchronization of the
images from the two eyes?  How would you study this question?

 \ee

\ee %All Questions

