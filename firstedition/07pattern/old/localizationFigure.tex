Figure \ref{f6:LocalizationTheory} sketches how the
physical constraints differ between
the two tasks.
\begin{figure}
\centerline{
  \psfig{figure=../06space/fig/LocalizationTheory.ps ,clip= ,height=3.5in}
}
\caption[Localization:  Physical Constraints]{
Our ability to resolve two lines (spatial acuity)
is not limited by the same physical constraints as our
ability to localize an object (vernier acuity)
In spatial acuity experiments quanta from the
two signals become mixed, and physical information about
the two targets is unrecoverable.
In the vernier alignment task, quanta from the
targets remain segregated.
It continues to be possible, by
sufficient averaging, to discriminate the target locations.
}
\label{f6:LocalizationTheory}
\end{figure}

