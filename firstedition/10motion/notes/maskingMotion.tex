\subsection*{Spatiotemporal filter estimation:  Masking measurements}

There have been several attempts to measure the sensitivity of
putative space-time oriented filters involved in motion perception.
For example, Anderson et al. (1991a) estimated the spatial frequency
and orientation selectivity using a visual masking paradigm (see
Chapter~\ref{chapter:space}).  Their experiments are based on the
assumption that when a visual mask interferes with the perception of a
moving object, the interference arises from unwanted excitations of a
space-time oriented receptive fields.  Were this so, the receptive
field of the relevant neuron or small population of neurons might be
estimated from the influence of different visual masks at various
spatial frequencies and orientations (see also Anderson, et al.,
1991b; Daugman, 1984; Harvey and Doan, 1990).

Anderson et al. (1991a) used a test stimulus consisting of a vertical sine
wave grating, drifting either to the right or the left.  Observers
were asked to judge whether they could see the motion of the test
stimulus, not just the presence of the stimulus.  The experimenter
made the task more difficult by adding a masking stimulus consisting
of a sine wave at a different spatial frequency and orientation.  The
masking stimulus jumped back and forth randomly across the screen,
making it difficult to see whether the low contrast test grating was
moving to the right or to the left.  The observer adjusted the
contrast of the test stimulus to a level at which it was just possible
to be confident about the direction of motion of the test.

\begin{figure}
\centerline{
 \psfig{figure=../09motion/fig/motionMask.ps,clip= ,height=3.5in}
}
\caption[Direction of Motion in the Presence of Masking Stimuli]{
{\em Motion masking.}  Observers adjusted the contrast of a 1 cpd
sinusoidal test grating until they could just discern whether the test
was moving to the right or to the left.  Test threshold was adjusted
in the presence of a masking stimulus, whose spatial frequency and
orientation relative to the test stimulus are shown as the two
horizontal axes on the graph, and with no masking stimulus.  The
height of the surface shows the ratio of contrast needed to perceive
the direction of motion in these two cases, that is, the reduction in
sensitivity due to the mask.  Masks that are similar in spatial
frequency and orientation elevate contrast threshold best (Source:
Anderson et al. 1991).  }
\label{f9:motionMask}
\end{figure}
The data in Figure~\ref{f9:motionMask} show how masks at different
spatial frequencies and orientations reduced the visibility of a 1 cpd
test grating moving at 8 Hz.  The two horizontal axes measure the
spatial frequency and the orientation of the mask.  The vertical axis
measures the threshold ratio in the presence and absence of the mask.
Under these conditions, as for most of the visual range, the mask is
most effective when it is similar to the target in both spatial
frequency and orientation\footnote{ Though, at very high test
frequencies (30 cpd) masking by 15 cpd stimulus at 25 percent contrast
is more effective than masking by a 30 cpd stimulus at 25 percent
contrast.  This may be due to the poor visibility of high frequency
gratings (Anderson and Burr, 1989).}

Using methods like those described in Chapter~\ref{chapter:space}, it
is possible to explain the masking data in Figure~\ref{f9:motionMask}
in terms of the response of linear filters.  Even without such a
theory, by examining the pattern of threshold elevation we can see
that the mask effect is qualitatively consistent with the spatial and
orientation tuning of neurons in area V1.  But, the relationship
between visual masking in motion experiments and these neurons is far
from proven.  Also, there have been no mixture experiments performed
to verify that one can generalize from measurements with these masks
to measurements with other masks.  In Chapter~\ref{chapter:space} I
review some of the difficulties in interpreting masking experiments,
and all of those difficulties apply in this case, too.

