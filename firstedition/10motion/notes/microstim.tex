There is a growing collection of papers that
probe the relationship between behavior and neural
responses (e.g. Tolhurst, 1983; Parker, 1985; Hawken, 1990; Vogels,
1990; Newsome et al. 1992).
The paradigm for one type of study is this.
The investigator chooses an experimental task
and measures the performance both
of a single neuron and an entire animal.
The investigator compares the neural and behavioral performances
in response to a changing stimulus.
The design is correlational;
when the neural and behavior responses covary with
stimulus manipulation, the investigators infer
a cause-effect relationship.
Namely, they infer that the neural response is
responsible for the behavioral response

Much useful science, particularly in the social sciences,
has relied on observed correlations.
But observational data are weaker than direct experimental
manipulations frequently available in natural science.
A second method was described recently by Newsome
and his collaborators (Salzman et al., 1992).
Their method goes beyond the correlational
studies to by altering the neural responses during
behavioral trials.

\begin{figure}
\centerline{
  \psfig{figure=../05cortex/fig/newsome.ps ,clip= ,height=3.5in}
}
\caption[Microstimulation Experiments]{
The graph plots the monkey's performance
as a function of the percentage of dots
moving in the test direction.
The open symbols plot the probability of a behavioral
response in the principal direction without microstimulation
of the area.
The filled symbols plot the same probability
when microstimulation was administered.
Microstimulation was equivalent to increasing the
fraction of dots moving in the correlated direction.
(Data from Salzman and Newsome).
}
\label{f5:newsome}
\end{figure}
Newsome and his colleagues studied how
an alert, behaving monkey discriminates
the direction of motion of a target.
The target consisted of a set of dots moving in random directions;
the independent variable was the proportion of dots 
moving in the test direction (see Figure \ref{f5:newsome}).
At the start of the experiment
the investigators isolated a neuron in area MT
(a visual area beyond the V1, presumed to be
involved in motion processing).
In area MT, as in other areas we have reviewed, nearby
neurons tend to have similar receptive fields (see e.g. Albright, 1984).
The experimenters used a test stimulus whose direction
corresponded to the best direction
of the isolated neuron;
the presumption is that this direction
defines the best direction of the receptive
field for most of the nearby neurons.

The monkey's task was to discriminate in which of
two directions the stimulus moved.
During half of the trials the correlated dots moved
in the direction preferred by the neuron,
and during the
remaining trials the stimulus was presented moving
in the opposite direction.
When only a small fraction of the dots
move in the test direction, the task is quite difficult
and performance is near chance ($0.5$).
When all of the dots move together
in the test direction (100\% correlation),
the task is quite simple and performance is nearly perfect ($1.0$).
The open symbols in the graph in Figure \ref{f5:newsome} 
show the increase in performance
as the fraction of dots moving in the test direction increases.

The most important and novel feature of the experiment is this:
on one half of the trials, randomly selected, the investigator
injected a small amount of current, called a {\em microstimulation},
into area MT.
The small current injection changes the state of 
the nearby neurons, and
the data in Figure \ref{f5:newsome}
show that the microstimulation also
changes the monkey's performance.
The filled symbols in part (b) show the
monkey's performance on trials when current
was injected.
The monkey was more likely to say the stimulus moved in the
direction preferred by the neurons in the
presence of microstimulation than in its absence.
In this particular experimental condition,
the microstimulation was equivalent
to increasing the percentage of dots moving in
the test direction by ten percent.

There are at least two reasons why
this method is very significant.
First, the method does not involve inferences about
causality from correlation.
Rather, the investigator actively controls the state of
the neurons and observes a change in the behavior.
Second, the method brings us much closer to the problem
of designing visual prosthetic devices and advancing
Brindley and Lewin's work.
By understanding the perceptual consequences
of visual stimulation,
we may be able to design visual prosthetic devices
that generate predictable and controlled visual sensations.
