Linear model descriptions of spectral data are particularly
helpful in color calculations for two reasons.
First, they
provide a way of describing the approximations
efficiently, using fewer numbers than the original measurement.
In the case we are about to work through, the original
measurements consist of $31$ numbers.
Yet, the vector of weights we will derive to describe
the measurements will provide an excellent approximation
using only three numbers.

Second, linear model representations work very
smoothly with the matrix calculations that are
central to color calculations.
There are other algorithms that can
represent the data efficiently; but,
linear model representations both compress
the representation and retain the simplicity of the computations.

%%%%%%%%%%%%%%%%%%%
\paragraph{Efficient Calculations with Daylights. }
Apart from compressing the representation,
the linear model representations works
well as part of common color calculations.
For example, suppose that
we compute the photopigment absorption rates from the linear
model representation.
We can express the linear model
approximation to a signal, $\ColSig{j}$ using matrix notation,

\begin{equation}
\label{e8:linMod}
\ColSig{j} \approx {\Emod} \Ewgt{j}
\end{equation}

where the first column of the matrix $\Emod$ contains
the mean of the data,
$E_0$, and the next two columns contain
the two basis functions, $E_1, E_2$.
The vector of weights $\Ewgt{j}$ is
$(1 , \epsilon_{1,j} \epsilon_{2,j})^{t}$.
The precise absorption rate we expect from the
light is $\Sensor^t \ColSig{}$, and
an estimation of the absorption rate,
$\hat{\rec}$ based on subsituting the linear model, is 

\begin{equation}
\label{e8:linModForward}
\hat{\rec} = \Sensor^t \Emod \Ewgt{j} ~~.
\end{equation}

Figure \ref{f8:sigEst} illustrates the calculation
in matrix tableau.
When we make many calculations with a linear model, 
we can group the two large matrices into a single matrix
small matrix, $\Sensor^t \Emod$.
This grouping leads to efficient calculations.
\begin{figure}
%\centerline{
%  \psfig{figure= ,clip= ,width=3.5in,height=3.5in}
%}
\caption[Linear Model Calculation Tableau]{
Matrix tableau of linear model calculations
used in creating receptor responses and in estimating
the illuminant from within a linear model that gives
rise to the sensor catch.}
\label{f8:sigEst}
\end{figure}


Judd et al. found that the daylight illuminants
we encounter only vary in certain limited ways.
This is an important observation;
if the spectral wavelength functions in
our environment vary widely, like white noise functions,
then a system containing
a small number of photoreceptors will
not be very useful in distinguishing
among signals incident at different times.
But, if there is great regularity in the signals, however,
then a few spectral measurements
may be enough to distinguish between important
functions and to identify important classes of objects.
Judd et al.'s measurements inform us that
daylights are restricted.
We need only a few measurements to
obtain an excellent approximation of the signal.
