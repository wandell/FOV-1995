\section{The dimensionality of color appearance}

The color-matching experiment offers us
a precise way to define the dimensionality
of color:  we can build a three-dimensional
color representation based on matching because
we can match any light using three primaries in the
color-matching experiment.

For many authors, the notion of color appearance
extends well beyond color matching.
One can agree that we can represent lights
using the three numbers in the color-matching experiment,
and yet believe that the experience of color appearance
requires a much higher dimensional representation
that includes other factors like gloss, roughness,
transparency, and so forth.

A high-dimensional representation of color
should not seem odd, or even unlikely, to you.
We have seen several examples in this chapter and book
that demonstrate how color appearance depends on the spatial
pattern of the stimulus.
Since the spatial pattern includes the photoreceptor responses
at more than one location, there is no reason to suppose that
color appearance depends on just three numbers defining
the relative quantum absorption
rates in the three classes of photoreceptors.
Color appearance is built upon a much larger collection
of photoreceptor responses, spread across space, so the
dimensionality of what we consider color could be much
larger than three simple numbers.

