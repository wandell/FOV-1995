The core problem of spatial pattern vision
is simply stated, if difficult to solve.
How do we decide which areas within a visual image
describe the properties of a single object or process,
\footnote{
To many readers the use of the word ``object'' will seem
easy to understand, while the word ``process'' is obscure.
Unfortunately, to me both are vague.
There are many simple examples
of an object, like a rock or a chair, that make
the word seem comfortable.
But we also see things like ocean waves or fields of grain which
are not precisely objects, but rather physical
structures or textures that are more of a process
than an object.
I use the word ``process'' to cover
these kinds of organization that are just out of reach
of my definition of ``object.''
}
and once we have discovered how to identify
these regions as part of a common object,
how can we use the image data to infer the object
properties and identify the object or process.

We begin our study of spatial pattern vision by
discussing some of the basic features of an
image that govern the perceived brightness
or contrast of a spatial pattern.
First, consider the pattern in Figure \ref{fig4:sweep}.
\begin{figure}
\centerline{
  \psfig{figure=../04space/sundraw/sweep.VarCon.ps ,clip= ,height=2.5in}
}
\centerline{
  \psfig{figure=../04space/sundraw/sweepCon.ps ,clip= ,height=3.0in}
}
\centerline{
  \psfig{figure=../04space/sundraw/sweepRetImg.ps ,clip= ,height=3.0in}
}
\caption{
(a) The spatial frequency of this pattern increases from
left to right but is the same across each row.
The contrast of the pattern is smaller at the top row
and increases towards the bottom of the image.
(b) The normalized physical contrast of each row.
(c) The normalized retinal image of each row, calculated using
Westheimer's linespread function.
}
\label{fig4:sweep}
\end{figure}
Across each row the pattern
consists of a set of light and dark bars, 
with increasing spatial frequency as we look from the left
to the right of the pattern.
The contrast of the spatial pattern increases as we look
from the top of the image to the bottom.
The normalized spatial contrast pattern for each row is
shown in part (b) of figure \ref{fig4:sweep}.
This pattern is called a {\em sweep frequency} pattern.
Sweep frequency patterns are commonly used to
test signal for spatial pattern performance of various
devices, including monitors, cameras and printers.

The physical contrast of the sweep frequency pattern in part (a)
of the figure is constant across each row of the image.
Although the physical contrast remains constant,
the perceptual contrast of the image varies.
The finely spaced lines on the upper right appear
to have lower contrast and are difficult to see.

The loss of apparent contrast of the finely spaced lines
becomes may be due to
the blur of the optical elements of the eye
and perhaps of the neural sensitivity as well.
We can compute the effect that the optical blurring
will have on the sweep frequency by convolving the
contrast pattern with the linespread function of the optics.
Assuming that you are sitting at a point where the pattern
sweeps out one-degree of visual angle, 
the contrast pattern at your retina
will be approximately that shown in part (c) of Figure \ref{fig4:sweep}.

You can verify that the loss of apparent contrast
is due, at least in part, to optical blurring
by varying your viewing distance from the page.
Put the page across the room and judge where the lines
become invisible.
Then come up close to the page and judge again.
You will see that the point
at which the lines remain visible
depends on your viewing distance but remains constant
in terms of number of lines per degree of visual angle.
If this result holds for your copy of the book, then
the reproduction is a good one.

Part (d) of Figure \ref{fig4:sweep}
contains a sweep frequency pattern of uniformly high contrast.
You might not be surprised that it appears as it does,
of uniform contrast across the image, because
that is what the pattern is.
But remember that
the blurring imposed by the optics
makes the retinal image very uneven in contrast,
strongly reducing the contrast of the finely spaced lines.
Indeed, when the pattern sweeps out one-degree of angle at your eye
the retinal contrast of the highest frequency components of
the sweep frequency is more than a factor of 10 smaller than the
contrast of the low frequency part of the pattern.
Thus, you should be surprised that the pattern appears to
have roughly constant contrast!

The apparent contrast does not follow the retinal image contrast.
Rather, the appearance is more similar
to the physical contrast, as if,
when the signal is strong, the nervous system
compensates for the anticipated optical defocus.

Figure \ref{fig4:craikedge}
\begin{figure}
\centerline{
  \psfig{figure=../04space/sundraw/Edge.ps ,clip= ,height=1.5in}
}
\centerline{
  \psfig{figure=../04space/sundraw/Edge.cont.ps ,clip= ,height=2.0in}
}
\centerline{
  \psfig{figure=../04space/sundraw/CraikEdge.ps ,clip= ,height=1.5in}
}
\centerline{
  \psfig{figure=../04space/sundraw/CraikEdge.cont.ps ,clip= ,height=2.0in}
}
\centerline{
  \psfig{figure=../04space/sundraw/BlurredEdge.ps ,clip= ,height=1.5in}
}
\centerline{
  \psfig{figure=../04space/sundraw/BlurredEdge.cont.ps ,clip= ,height=2.0in}
}
\caption{
Craik-O'Brien Cornsweet edge illusion 
(a) Real edge and contrast graph (appears like an edge)
(b) Fake edge and contrast graph (appear like an edge).
(c) Smoothed edge, should be the difference.
This could be (a) = (b) + (c) type of image, though it is not currently.
Instead the blurred image is a gaussian blur on the edge.
The Craik image is back to back exponentials.
\label{fig4:craikedge}
}
\end{figure}
contains a second example in which the physical contrast
of a pattern is unlike the apparent contrast.
In parts (a) and (b) of Figure \ref{fig4:craikedge} we perceive a contrast edge;
the central region of the figures appears darker than the surround.
but the physical
contrast of the patterns in the two images are entirely different.
The image in (a) is a contrast pattern that
we expect to be perceived as an edge.
The image in (b) consists of a peculiar contrast pattern,
whose properties I will describe below,
that physically is quite unlike the edge pattern.
Yet, our percept is of an edge.
Moreover, the areas on either side of the edge -- which
have the same contrast -- appear to have different brightnesses.
(To convince yourself that the contrasts on the two side
of this illusory edge are the same, place your finger
on the paper and cover the apparent edge in the contrast pattern.
You will be able to confirm that the two sides
of the figure have physically the same contrast, despite their appearance.)

Physically, the image in (c) is similar to the edge in (a);
the contrast levels on either side of the blurred edge
are as different as the contrast levels on either side of the
edge in part (a).
Yet we see no difference in the contrast levels.
You can convince yourself that there is an intensity
difference between the two sides of the figure
by placing a small strip of paper
(or your finger) across the blurred edge.
By introducing a sharp boundary,
you will be able to see the contrast difference between the two sides.

There are several lessons to be learned
from these demonstrations.
First, the apparent contrast 
does not necessarily follow the retinal image contrast.
Both contrast patterns in Figure \ref{fig4:sweep}
have substantially less contrast at high spatial
frequencies.
This difference is evident at low contrast levels,
but the difference is not apparent
at high contrast levels.

The images in Figure \ref{fig4:craikedge} also illustrate
that we cannot infer the apparent
contrast at a single point in the image
just by knowing the physical contrast at that point.
Which regions of the image look lighter and darker depend
upon the spatial structure of the context,
not just the contrast at a point in the image.

We saw this same phenomenon, of course, when we
reviewed the relationship between color-matching and color-appearance.
The Albers painting in the color section demonstrates that the
color-appearance of a point depends upon the spatial context
in which the point is seen.

To understand the point by point
appearance of different spatial patterns,
we must try to understand how the nervous system
incorporates spatial information across the image.
The spatial receptive field of visual neurons gives us
some hints about how image data is pooled across space
by the visual pathways.
But most of what we know about this process comes from
experimental evidence using human subjects.

