Sacks describes a patient diagnosed with
cerebral achromatopsia who could identify all of the
test patterns in the Ishihara plates.
Sacks used this observation to justify the claim
that the subject had three functional cone classes.
Yet, the patient reported severe color abnormality.

Interpretations:
  Patients who can see the Ishihara plates, but report no color.
Ishihara plates are based on a failure to equate at the
cones.  A non-linear combination of the cones to a univariate
representation keeps them visible.
Even a linear combination of the cones that does not exclude
any cone class keeps them visible.
The visibility with the Ishihara plates suggests that all
three cone classes are working.

Why, then, is the individual's experience is of a gray world?
The argument is that the interpretation of the three cone signals
must be deficient.
If the lesion is localized, then the argument must be that
the interpretation of the three signals, which gives
rise to our EXPERIENCE, must be localized as well.
In the absence of a special signal from
the color interpretation center,
our perception is of a shade of gray.

Presumably, the level or shade of gray is a monotonic function
of the signals from the three cone classes.


\nocite{Meadows1974,Critchley1965}
In 1974, J.C. Meadows drew the scientific world's
attention to a visual deficit called
{\em cerebral achromatopsia}.
Patients with this syndrome
complain that they cannot see colors.
In some milder cases, colours lose their saturation.

Of course, in principle
here are many reasons why a patient's color
vision may be defective.
Of these, the absence
of a photopigment being the most common
(see Chapter \ref{chapter:wavelength}) and best understood.
Because we know the spectral absorption
curves of the photopigments,
we can predict the performance of
individuals who are simply missing a photopigment
using a variety of simple behavioral tests. 
Patients with this type of specific color loss will not usually
be diagnosed as having cerebral achromatopsia.

Mollon et al. ----  get and read this ----

Our understanding of photopigment loss and peripheral
colorblindness is relatively recent.
It is not surprising, therefore, that there was very
little detailed study of the color defect in the early
literature.
The rarity of these cases, and the absence of evidence
distinguishing the reports from photopigment loss,
caused several senior investigators to deny 
the syndromes' existence.
In his book and a long article,
Zeki has argued that the denial of cerebral achromatopsia
is due to deeper character flaws.
But, I tend to agree with Meadows' gentler assessment;
the early evidence is spotty and poorly argued.
In any event, Meadows' paper together with Zeki's studies of V4,
have opened a new and fascinating set of questions
concerning color processing in the central visual pathways.




Use the Meadows paper to
introduce the achromatopsia.  He is fair, even-handed, and 
was the first to basically make a good case.

An important point Meadows makes: there appear to be several
kinds of achromatopsia.  Some in which the colors appear wrong, some
which are hemi, others are complete, bilateral.  This is the kind of
thing Zeki completely passes over.  Also, Meadows doesn't try to
castrate Holmes and others for denying achromatopsia.  He points out
that much of the evidence is quite weak and sloppy.  But that is
something that never really bothers Zeki, after all.

Can Zeki spell "Persecution Complex"?

Use the Zeki color appearance cells in V4 but not in V1 as an
example that parallels the Movshon analysis of units in
MT/V1 for plaids.
Check the other V4 papers, though, to see if anyone has
confirmed/denied aspects of his analysis.

Make the big picture point that the level of argument
is very poor.

Orientation ->  area involved in form perception
Wavelength selectivity -> area involved in color perception
Neurons prefer moving targets -> area involved in motion perception.

Level of analysis is very poor and doesn't meet the test
of physical realizability:  we understand this thing when we can
make an device that behaves the same way.

Go for it with your own appearance work and analyses.
Say what you want and don't worry so much about
even-hnadedness.

