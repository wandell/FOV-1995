\subsection{Perceptual Constancies}
%
%	Reads oddly according to Norma... I think the
%	Boring illusion should go at the back, and that
%	the Sullivan-Georgeson, Poirson, Lowry and DePalma
%	material should be expanded in the earlier section,
%	along the lines of Allen's work.
%
It seems tautological that our
perception of objects
should reflect the properties of the objects.
Hence, we should not attribute the properties of
early or intermediate neural representations
to the objects.
For example, the effects of lens de-focus, photoreceptor sampling,
and so forth are properties of the visual system's
method for assessing the object properties.
These effects influence the retinal image
and neural encoding.
But, their influence should not be attributed
to the object.

To the extent that visual processing succeeds in removing
unwanted distortions
that are unrelated to the properties of the object, 
we say that the visual system has managed to achieve
a {\em perceptual constancy}.
The term constancy is used because frequently
the retinal image changes while the object itself remains constant.
The retinal image can change due to variations
in viewing distance, position on the retina, or lighting.
All of these factors introduce changes in the
the neural response.
An important challenge 
is to identify which changes are due to changes in
imaging, and which changes are genuine
variations of the object properties.

There are several important examples of perceptual constancies;
the contrast-matching experiment illustrates one of them.
We have just seen from the Georgeson and Sullivan experiment 
that the appearance of high contrast
patterns is more closely predicted by the contrast properties
of the physical stimulus prior to the retinal image
than by the contrast properties of the retinal image.
In contrast constancy, perceptual mechanisms compensate
for the loss of image contrast.

\begin{figure}
\centerline{
  \psfig{figure=../06space/fig/Boring.Fig1.ps,clip= ,height=3.5in}
}
\caption[Boring Size Illusion]{
Boring size illusion
}
\label{f6:Boring}
\end{figure}
A second instance of a perceptual constancy
occurs when we assess the physical size of an object.
As Figure \ref{f6:Boring} 
illustrates, the apparent size of an object is not judged
directly by the size of the object in the retinal image,
but rather the visual system tries to make an assessment
based on a variety of clues.
This famous picture, created by E.G. Boring,
illustrates the phenomenon of size constancy.
The two pictures are made by simply clipping out and re-positioning
within the image the woman seated in the background.
Since this part of the image has simply been translated within
the page, the retinal image size is unchanged.
The apparent size of woman and her chair at these two
locations is very different.
When she is placed in context, we are unaware
that the retinal image swept out is very small compared
to the retinal image for the woman seated in the front of the hall.
The surprising discrepancy in the retinal image size is 
made apparent to us by sliding her down the page.
Most observers are surprised at the disparity in the image
size precisely because, when the image is seen in its
natural position, we discount the retinal image size.
The discounting of the retinal image size
in favor of an estimate of an estimate of the
object's true size is another example
of a perceptual constancy.
We will return to discuss perceptual constancies in
Chapter \ref{chapter:Color} on color appearance.


