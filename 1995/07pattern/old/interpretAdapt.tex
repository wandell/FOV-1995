\nocite{GreenleeMagnussen}
Greenlee and Magnussen have performed a collection of studies
that cast doubt on the hypothesis that the neural mechanisms
responsible for limiting threshold detection and
the neural mechanisms responsible for pattern
adaptation are precisely the same.
The results I review here are part of a sequence
of articles by these authors that challenge 
the conventional interpretation of the neural image as being organized
in terms of multiple-channel spatial-frequency receptive fields.

Greenlee and Magnussen measured observer's
sensitivity to low frequency squarewaves, at 0.33 cpd.
By looking at the contrast sensitivity function in
\ref{f6:robson.data}, you will find that
the visual system's sensitivity to the third harmonic and
fifth harmonics of the square wave, at 1 cpd and 3 cpd respectively,
is more than three times greater than sensitivity
at the fundamental, 0.33 cpd.
For a very low-frequency square-wave, therefore, one
might expect that the energy at the third harmonic and fifth harmonics
would contribute to the visibility of the square wave.
If adaptation reduces sensitivity of the same
neurons that limit sensitivity,
then adapting to a 1 cpd target should reduce the visibility
of the 0.33 cpd square wave grating.

\begin{figure}
\centerline{
  \psfig{figure=../06space/fig/GM.Fig2.ps ,clip= ,height=3.5in}
}
\caption[Adpatation to Low Frequencies]{
Thresholds are elevated following adaptation to 1 cpd and 3cpd
sinusoidal targets
(open symbols) as well as inter-leaved adaptation to 1 and 3cpd
sinusoids (filled symbols).
\comment{Greenlee and Magnussen 2 }
}
\label{f6:GM.Fig2}
\end{figure}
Thresholds following adaptation to a 1 cpd grating are shown
as the open squares in Figure \ref{f6:GM.Fig2}.
The open circles show the results of adapting their subjects
for 4 minutes to a 3 cpd sinusoidal grating.
Each of these curves repeats Blakemore and Campbell's original
observation that threshold elevation is tuned in
spatial frequency.

\begin{figure}
\centerline{
  \psfig{figure=../06space/fig/GM.Fig4.ps ,clip= ,height=3.5in}
}
\caption[Adaptation to Low Frequency Mixtures]{
Adaptation to 1 and 3 cpd does not elevate threshold
to a square wave or sinusoid at 0.33 cpd (bars on the left)
although it does elevate threshold to sinusoids at
1 and 3 cpd (bars on the right).
The panels from two different observers.
\comment{Figure 4 from Greenlee and Magnussen}
}
\label{f6:GM.Fig4}
\end{figure}
The filled symbols show the results of raising threshold
using a procedure that adapts the subject
simultaneously to 1 cpd and 3 cpd sinusoidal gratings.
The observer views each of these patterns separately,
for 2.5 seconds, during the adapting period.
The two patterns are interleaved, and the entire
adaptation period takes eight minutes.
Thresholds to sinusoids following this type of adaptation
follows the general pattern observed
when adaptation experiments are performed
using only a single pattern.
When the observer adapts to the interleaved
pattern, he or she shows a loss of
sensitivity to patterns near both of the adapting
frequencies.
This is shown again in Figure \ref{f6:GM.Fig4}.
The filled bars on the right show the elevation of
threshold to a 1 cpd and a 3 cpd target
following adaptation to the interleaved 1 cpd and 3 cpd gratings.
There is a considerable adaptation effect at
both of these frequencies.

The key question, then, is what happens to threshold
to the low frequency square wave grating?
Since observers are more sensitive at 1 cpd and 3 cpds
than at 0.33 cpd, we expect that the adaptation at 1 and 3 cpd
will make the low frequency square more difficult to see.
But there is no threshold elevation for the low
frequency squarewave.
Even though sensitivity is raised for all test frequencies
between 1 and 8 cycles per degree,
we find that when sensitivity to sinusoidal gratings
at these frequencies is reduced, sensitivity to
the squarewave is unaffected.
These measurements are difficult to reconcile with
the multiresolution hypothesis as applied to low frequency
targets.

