\section{Photopic Wavelength Encoding}
When the cones initiate vision,
under photopic conditions,
we do encode some information about
the relative spectral power distribution of the incident light.
Changes in the relative spectral power distributions
result in changes of the color appearance of the light.
Several of the important properties of
color appearance can be traced
to the way cone photoreceptors
encode the relative spectral power distribution of light\footnote{
N.B. Color appearance is not a simple consequence
of the spectral power distribution of the incident light.
We will discuss color appearance broadly in
Chapter~\ref{chapter:Color}.
}.

We will relate the human ability to discriminate
lights to the properties of the cones
just as we did with the rods.
First, we will review the matching experiments
that characterize how well people
can discriminate between lights
with spectral power distributions.
When we measure under photopic conditions, the
experiment is called the {\em color-matching} experiment.
The color-matching experiment is the 
foundation of color science and of direct significance
to many color applications (see the Appendix).
Second, we will relate the properties of the
color-matching experiment to the properties of the
cone photopigments.
The analysis of photopic wavelength encoding
parallels the analysis of scotopic wavelength encoding.
The main differences are that (a) we must keep track of the
photopigment absorptions in three cone photopigments
rather than the single rod photopigment, and (b) 
until quite recently
the cone photopigments were not present
in sufficient quantity to define
their properties with any certainty (Merbs and Nathans, 1992).
Hence, the problem of relating color-matching and the
cone photopigments was solved using
other indirect biological measurements.
We can learn a great deal from studying the logic
of these methods.

%7.5 high 6.0 wide
\begin{figure}
\centerline {
\psfig{figure=../04col/fig/colorMatching.ps,clip=,height=2.5in}
}
\caption[Color Matching Experiment] {
{\em The color-matching experiment.}
The observer views a bipartite field and
adjusts the intensities of the three primary lights to match
the appearance of the test light.
(a)  A side-view of the experimental apparatus.
(b)  The appearance of the stimuli to the observer.
(After Judd and Wyszecki, 1975.)
%Figure 1.12 page 42, Color in Business, Science, and Industry
}
\label{f3:colormatch}
\end{figure}
Figure~\ref{f3:colormatch} shows a simple apparatus that can be
used to perform the color-matching experiment.
The observer views a bipartite visual field
with the test light on one side.
The test light may have any spectral power distribution.
The second half of the bipartite field 
contains a mixture of {\em three}
primary lights.
Throughout the experiment, the relative spectral
power distribution of each primary light is constant;
only the absolute level of the primary lights
can be adjusted.
The observer's task is to adjust
the intensities of the three primary lights
so that the two sides of the
bipartite field appear identical.

\begin{figure}
\centerline {
\psfig{figure=../04col/fig/tvMetamers.ps,clip=,height=2.0in}
}
\caption[Monitor Metamers]{
{\em Metameric lights.}
Two lights with these spectral power distributions
appear identical to most observers and are called {\em metamers}.
The curve in part (a) is an approximation
to the spectral power distribution of the sun.
The curve in part (b)
is the spectral power distribution of 
a light emitted from a conventional television monitor
whose three phosphor intensities were
set to match the light in (a) in appearance.
}
\label{f3:metamers}
\end{figure}
When the observer has completed setting an appearance match,
the lights on the two sides of the bipartite field are not
physically the same.
The test light can have any spectral power distribution,
while the mixture of primaries can only have a limited
number of spectral power distributions determined by the
possible weighted sums of the three primary light spectral
power distributions.
Lights that are photopic appearance matches, but
that are physically different, are called {\em metamers}.
Figure \ref{f3:metamers}
contains a pair of spectral power distributions
that match visually but differ physically, i.e.
a pair of metamers.

The metamers in Figure \ref{f3:metamers} illustrate
that even under photopic viewing
conditions we fail to discriminate
between very different spectral power distributions.
To understand the behavioral aspects of photopic
wavelength encoding, we must try to predict
which spectral power distributions we can discriminate.
The first question we ask is whether we can
predict performance in the
photopic color-matching experiment using linear systems methods.

\begin{figure}
\centerline {
\psfig{figure=../04col/fig/phot.superposition.ps,clip=,height=3.5in}
}
\caption[Color Matching Superposition]{
{\em The color-matching experiment satisfies the principle of
superposition.} In parts (a) and (b) test lights are matched by
a mixture of three primary lights.  In part (c) the sum of
the test lights is matched by the additive mixture of the
primaries, demonstrating superposition.
}
\label{f3:phot.superposition}
\end{figure}
We can define the experimental measurements in the
color-matching experiment in direct analogy with 
the definitions we used in the scotopic matching experiment.
The input variable in the color-matching experiment
is the light $\test$, just as in scotopic matching.
In the color-matching experiment, however, the subject's responses
consist of three numbers, not just one.
So, we record the responses using a three-dimensional
vector, $\primaryint$.
The entries of $\primaryint$
are the intensities of the three primary lights 
$( \primaryinti{1} , \primaryinti{2} , \primaryinti{3} )$.
To test superposition in the color-matching experiment
we follow the logic
illustrated in Figure \ref{f3:phot.superposition}.
We obtain a match to a $\test$
by adjusting the primary intensities to the levels in $\primaryint$.
We then obtain a match to $\test '$ by adjusting
the three primary intensities to $\primaryint '$.
We test additivity by verifying that the match
to $\test + \test '$;
is $\primaryint + { \primaryint '}$.
Photopic color-matching satisfies homogeneity and superposition.
We honor the person who first understood the importance
of superposition in color-matching by calling this
empirical property {\em Grassmann's additivity law}.

% height = 6.0in width = 9.0in
\begin{figure}
\centerline {
\psfig{figure=../04col/fig/photopic.ps,clip=,height=2.0in}
}
\caption[Color Matching Matrix Tableau]{
{\em Matrix tableau of color-matching.}
The photopic color-matching experiment defines a
linear mapping from the test light spectral power distribution
to the intensity of the three primary lights.
The rows of the $3 \times \nl$ system matrix are called
the {\em color-matching functions}.
These functions can be estimated by setting matches to many different
test lights and solving a set of linear equations.
}
\label{f3:photopic}
\end{figure}
Because the color-matching experiment
linearly maps the physical stimulus $\test$
to the primary intensities, $\primaryint$,
there must be a system matrix that maps the input vector
$\test$ to the output vector $\primaryint$.
Figure \ref{f3:photopic} shows the input-output relationship
for the photopic color-matching experiment in matrix tableau.
We will call the $3 \times \nl$ system matrix $\Photopic$.

We can estimate the system
matrix $\Photopic$ from the color-matches in the
same way as we estimated the scotopic system matrix:
by setting matches to a collection of monochromatic test lights
with unit intensity.
Since the vector representing a monochromatic test light
is zero at each entry but one,
the product of the system matrix and the monochromatic test light
vector equals a single column of the system matrix.
Thus, by matching a series of unit intensity monochromatic lights,
we can define each of the columns of the system matrix, $\Photopic$.

It is also useful to think of the system matrix
in terms of its rows, which
are called the {\em color-matching functions}.
Each row of the matrix
defines the intensity of a single primary light
that was set to match the monochromatic test lights.
We will relate the rows of the photopic system matrix
to the properties of the cone photopigments just as we related
the single row of the scotopic system matrix to the
rhodopsin photopigment.
However, to make the connection between the cone photopigments and the
color-matching functions will require a little more work.

\subsection*{Measurements of the Color-Matching Functions}
Two important caveats arise
when we measure the color-matching functions.
These are only a minor theoretical nuisance,
but they have important implications for the laboratory experiment
and for practical applications.

The first issue concerns
the selection primary lights.
We should chose lights that are visually {\em independent}:
that is,
no additive mixture of two of the primary lights should 
be a visual match to the third primary.
This is an obvious but important constraint:
it would be unreasonable to choose the second
primary light that looked the same
as the first except for an intensity scale factor.
This choice would be foolish since
we could always replace the second light by
an intensity-scaled version of the first primary light,
adding nothing to
the range of visual matches we can obtain.
Similarly, a primary that can be matched by a mixture
of the first two adds nothing.
We must choose our primary lights so that they are
independent of one another.

Even among collections of primary lights that are independent,
some are more convenient than others.
Empirically, it turns out that no matter which primary
lights we choose, there will always be some test lights
that cannot be matched by an additive mixture of the three
primaries.
To match these test lights,
we must move one or even two
of the primary lights from the matching side of
the bipartite field to the test side of the bipartite field.
Thus, ordinarily we obtain a visual matches of the form

\begin{equation}
\test = 
 \primaryinti{1} \primary_{1} + 
 \primaryinti{2} \primary_{2} + 
 \primaryinti{3} \primary_3 .
\end{equation}

Shifting one of the primaries to the other side of the bipartite
field means that our match has the form

\begin{equation}
\label{e3:ShiftPrimary}
\test + \primaryinti{1} \primary_{1} = 
   \primaryinti{2} \primary_{2} + \primaryinti{3} \primary_3 .
\end{equation}

To a mathematician Equation~\ref{e3:ShiftPrimary},
is the same as

\begin{equation}
\label{e3:NegativePrimary}
\test = 
   - \primaryinti{1} \primary_1 +
     \primaryinti{2} \primary_2 + 
     \primaryinti{3} \primary_3 .
\end{equation}

Hence, when we encode the intensity of the primary light that
has been shifted to the other side of the test field we denote
the match using a negative intensity value\footnote{
Changing the sign of the primary intensity
is a trivial matter for the theorist.
It is a nuisance in the laboratory, however,
and usually impossible in technological applications such
as color displays.
Thus, the issue of selecting primaries to minimize
the number of times we must make this adjustment is
of great practical interest.}.

\begin{figure}
\centerline {
\psfig{figure=../04col/fig/cie.rgb.ps,clip=,height=3.0in}
}
\label{f3:cie.rgb}
\caption[Stiles and Burch 10 Deg Color Matching Functions]{
{\em The color-matching functions are the rows
of the color-matching system matrix.}
The functions measured by Stiles and Burch (1959) using
a $10^\circ$ bipartite field
and primary lights at the wavelengths 
645.2 nm, 526.3 nm, and 444.4 nm
with unit radiant power are shown.
The three functions in this figure are called
$\bar{r}_{10}(\lambda)$,
$\bar{g}_{10}(\lambda)$, and
$\bar{b}_{10}(\lambda)$.
% Data are from Wyszecki and Stiles figure 5(3.3.3) page 140.
}
\end{figure}
Figure \ref{f3:cie.rgb} plots color-matching functions
measured by Stiles and Burch (1959)
using three monochromatic primary lights at
645.2 nm, 525.3 nm and 444.4 nm.
Each function describes the intensity of one of the primary
lights used to match various monochromatic test lights.
Notice that the intensity of the red primary, at 700nm, 
is negative over a region of test light wavelengths,
indicating that over this range of test lights
the 700 nm primary light was added to the test field.

The color-matching functions are extremely important
in color technology, such as creating images on
color monitors and color printers.
I review the application of these methods to
color monitors in the Appendix.

\subsection*{Uniqueness of the Color-Matching Functions}

Suppose two research groups measure the color-matching functions
using different sets of primary lights.
One group measures the color-matching functions using
three primary lights $\primaryi{i}$, while
the second group uses a different set of primary lights, $\primaryi{i}'$.
Because the groups use different primary lights,
they will measure different sets of
color-matching functions, $\Photopic$ and $\Photopic '$.
How will the two sets of color matching functions be related?

We can answer this question by the following thought experiment.
First, create a matrix
whose columns contain the spectral power
distributions of the first group's
primary lights, and call this matrix $\Primarymat$.
The spectral power distribution of a mixture of the primaries,
with primary intensities $\primaryint$
is the weighted sum of the columns.
We can express this mixture using
the matrix product $\Primarymat \primaryint$.
Now, we can use the color-matching functions to predict
when a test light light will match the mixture
of three primaries.
The test and primaries will match when
\begin{equation}
\label{e3:match1}
\Photopic \test = \Photopic \Primarymat \primaryint .
\end{equation}
Suppose the second group of researchers can also
established matches to this test light
using their primaries.
To describe their measurements,
we create a second matrix whose columns
contain the spectral power
distributions of the second group's primary lights,
$\Primarymat '$.
Call the primary intensities used to match
the test with the second primaries is $\primaryint '$.
Since the light $\Primarymat ' \primaryint '$ is a visual
matched to the test light, we know that
\begin{equation}
\label{e3:match2}
\Photopic \test = \Photopic \Primarymat ' \primaryint '
\end{equation}
By combining Equations~\ref{e3:match1} and \ref{e3:match2},
we find that the two vectors
of primary intensities, $\primaryint$ and $\primaryint '$,
are related by a linear transformation,
\[
\primaryint = 
  (\Photopic \Primarymat )^{-1}\Photopic \Primarymat ' \primaryint ' .
\]

Finally, recall that the vectors in the
columns of the color-matching functions
are the primary intensity settings
necessary to match the monochromatic lights.
We have just shown that the primary intensities used
to make matches with the two different sets of primaries
are related by a linear transformation.
Hence, the system matrices containing
the color-matching functions $\Photopic$ and
$\Photopic '$ must be related a linear transformation.

With a little more algebra, one can show that the
color-matching functions are related by the following
linear transformation:
\begin{equation}
\Photopic = ( \Photopic \Primarymat ' ) \Photopic ' .
\end{equation}

The $3 \times 3$ matrix relating the two sets of 
color-matching functions, $\Photopic \Primarymat '$,
has a simple empirical interpretation;
its columns contain the
intensities of the new primaries
needed to match the original primaries.
To see this,
remember that each column of $\Primarymat '$ 
is the spectral power distribution
of one of the primary lights, $\primaryi{i} '$.
Thus, the first column of $\Photopic \Primarymat '$ is
the vector of intensities of the first group of primaries
needed to match $\primaryi{1} '$.
Similarly the second and third columns of $\Photopic \Primarymat '$
contain the intensities of the first group of primaries
needed to match the corresponding primaries in $\primary '$.
The matrix $\Photopic \Primarymat '$ contains three columns
equal to the primary intensities of $\primaryi{i}$
needed to match the new primary lights, $\primaryi{i} '$.

The photopic color-matching functions are not unique;
when we measure using different sets of primaries we
will obtain different color-matching functions.
But, the color-matching functions are not completely
free to vary either, since
different pairs of color-matching functions
will always be related by a linear transformation.
We say that the color-matching functions are unique up to a
free linear transformation.

\subsection*{A Standard Set of Color-Matching Functions}
\begin{figure}
\centerline {
\psfig{figure=../04col/fig/XYZ.ps,clip=,height=3.0in}
}
\caption[CIE Standard XYZ Functions]{
{\em The XYZ standard color-matching functions.}
In 1931 the CIE standardized a set of color-matching functions
for image interchange.
These color-matching functions
are called $\bar{x}(\lambda)$, $\bar{y}(\lambda)$, 
and $\bar{z}(\lambda)$.
Industrial applications commonly describe the color properties
of a light source using the three primary intensities needed to match
the light source that can be computed from the XYZ color-matching functions.
}
\label{f3:XYZ}
\end{figure}
When the members of the Committe Internationale d'Eclairage (CIE;
an international standards organization) met in 1931, they were fully
aware that the color-matching functions were not unique.
To facilitate communication about color,
the CIE defined a standard system of color representation
based on one particular set of color-matching functions, 
that everyone should uses.
This set of color-matching
functions defines the {\em XYZ tristimulus coordinate system}.
The color-matching functions
in this system are called 
$\bar{x}(\lambda), \bar{y}(\lambda)$ and $\bar{z}(\lambda)$
respectively.
They define one of the many possible system matrices
of the color-matching experiment.
Figure \ref{f3:XYZ} shows the three standard color-matching functions,
$\bar{x}(\lambda), \bar{y}(\lambda)$ and $\bar{z}(\lambda)$.

The standard color-matching
functions were chosen for several reasons.
First, $\bar{y}(\lambda)$ is a rough approximation
to the brightness of monochromatic lights of equal size and duration.
A second important reason is that the
functions are non-negative
which simplified some aspects of the
design of instruments to measure the tristimulus coordinates.
But, as with any standards decision, there are some
irritating aspects of the XYZ color-matching functions
as well.
One serious drawback is that
there is no set of physically realizable primary lights
that by direct measurement will yield the color-matching functions.
Primary lights that would yield these functions
would have to have negative energy at some wavelengths
and cannot be instrumented.
Another problem is that these early estimates have been
improved upon.
Specifically, D. B. Judd (1951) noted that the functions are
inaccurate in the short-wavelength region and he proposed a
modified set of functions that are often used by scientists,
although they have not displaced the industrial standard.
Also, and perhaps most significantly,
there is very little that is intuitive about the XYZ
color-matching functions.
Although they have served us quite well as a technical standard,
and are understood by the mandarins of our discipline,
they have served us quite poorly in explaining the discipline
to new students and colleagues.

\subsection*{The Biological Basis of Photopic Color-matching}

\begin{figure}
\centerline {
\psfig{figure=../04col/fig/photoreceptor.mat.ps,clip=,height=2.0in}
}
\caption[Photoreceptors and Color Matching]{
{\em Cone photopigments and the color-matching functions.}
If we measure the wavelength sensitivity of
each of the cone photopigments, we can
create $3 \times \nl$ system matrix to describe the cone absorptions.
Then, we can evaluate whether the
cone absorption system matrix can serve 
to explain the results of the color-matching experiment.
}
\label{f3:photoreceptor.mat}
\end{figure}
Just as we can explain the scotopic color-matching experiment
in terms of the light absorption properties of the rhodopsin photopigment,
we also would like to explain the photopic color-matching experiment
in terms of the light absorption properties of the cone photopigments.
We related the photopigments and the behavior by studying
the system matrices of the two experiments.
We found that two lights were scotopic matches when
$\Scotopic \test = \Scotopic \test'$,
and we then showed that the 
entries in the $1 \times \nl$ scotopic matching
matrix, $\Scotopic$, was the same
as the rhodopsin absorption function $\Rhodopsin$.
For photopic vision, we use the same general approach.
But, there are two complications:
there are three cone photopigments, not just one;
the photopic matching matrix is not unique.

Extending our analysis
to account for three cone photopigments 
instead of one rod photopigment is straightforward.
We measure the absorption properties of
each of the three cone photopigments,
and we create a system matrix, $\Iodopsin$,
with three rows to define
the three cone photopigment absorption functions.
This matrix generalizes the
rhodopsin system matrix $\Rhodopsin$.
Then, we compare the cone absorption system matrix,
$\Iodopsin$, with the color-matching experiment system matrix, $\Photopic$.
We must evaluate whether the cone photopigment matrix 
can explain the color-matching data.

From our analysis of color-matching,
we know that the color-matching system matrix is not unique;
there is a collection of equivalent system matrices, all
related by a linear transformation.
Hence, to  evaluate whether the cone absorption matrix
can explain the color-matching experiment,
we must evaluate whether
the color-matching system matrix, $\Photopic$,
is related to $\Iodopsin$ by a linear transformation.
Our next task, then, is to measure the cone absorption
system matrix, $\Iodopsin$.

\subsection*{Measuring Cone Photocurrents}
\begin{figure}
\centerline {
\psfig{figure=../04col/fig/electrode.ps,clip=,height=3.5in}
}
\caption[Photoreceptor Measurements]{
{\em Measurement instruments
for detecting the photocurrent generated by a macaque photoreceptor.}
The image shows a portion of the retina suspended in solution.
A single photoreceptor from this retinal section
has been drawn into
the microelectrode and is being stimulated by
a beam of light passing transversely through the
photoreceptor and microelectrode (Image courtesy of D. Baylor).
}
\label{f3:electrode}
\end{figure}
Currently, the best
estimate of the cone photopigment absorptions
is derived from measurements
of the cone photocurrent, that is the
change in the current flow through the membrane of
individual cones as they are stimulated by light.
Relating the photocurrent to the photopigment absorptions
requires some careful analysis because
the photocurrent depends nonlinearly on the
photopigment absorptions in the cone.
In this section we will develop new theoretical methods to
interpret the nonlinear cone photocurrent measurements
and infer the linear properties of the cone photopigments.

Baylor, Nunn and Schnapf (1987; Baylor, 1987) were the first to
measure the cone photocurrents in the macaque retina.
The macaque has three types of cones and its behavior
on most color tasks is quite similar to human behavior
Thus, the comparison between the properties of the
macaque photoreceptors and human behavior is a good place to begin
(DeValois and Jacobs, 1971).

To measure the cone photocurrent, Baylor, Nunn and Schnapf
removed the retina from the eye and chopped into
fine pieces about 100 $\mu m$ across.
The pieces are
placed in a chamber containing special solutions
that support the metabolism of the cells.
Even though the retina has been dissected from
the eye and chopped into pieces,
the electrical response of the photoreceptors remain
vigorous for several hours.
Baylor and his colleagues recorded the photocurrent of individual cells
using the experimental technique pictured in Figure \ref{f3:electrode}.
The figure shows
glass micro-pipette approaching a single photoreceptor.
The inner diameter of the micropipette
is between 2 and 6 $\mu m$,
only ten times as wide as
the wavelength of visible light.
A single photoreceptor outer segment is held
inside the micropipette,
and there is a thin
ray of light passing transversely through the photoreceptor
and stimulating it.

\begin{figure}
\centerline{
 \psfig{figure=../04col/fig/cone.timecourse.ps,clip=,height=2.0in}
}
\caption[Temporal Response of Photocurrent]{
{\em The timecourse of cone photocurrent} in response
to a brief test flash is biphasic.
The amplitude of the photocurrent response increases with
the stimulus intensity.
The response functions are
the same for different wavelengths of light,
such as at 500 nm and 659 nm in parts (a) and (b), respectively.
The stimulus timecourse is shown below the photocurrent plots.
(Source: Baylor Nunn and  Schnapf, 1987)  
% Figure 1, J. Physiol. v. 390 Figure 1, p. 149, p. 145-160
}
\label{f3:cone.timecourse}
\end{figure}
Figure \ref{f3:cone.timecourse}
shows the result of stimulating the photoreceptor
with a brief impulse of light.
The curves illustrate
the membrane photocurrent following a brief light flash.
The curves in part (a) of the figure
plot the response to 500 nm light
at a range of intensities.
The curves in part (b)
plot the response to 659 nm light at a range of intensities.
Before the stimulus is presented, there is a steady
inward flow of current consisting
of a stream of
positively charged sodium ions entering the photoreceptor
through ion channels in the cell membrane.
This steady level in the absence of
light is called the {\em dark current}.
It represents a baseline level
and is denoted as zero in the graph.
The plotted values are {\em biphasic}, varying
both above and below the baseline.

When the photopigment absorbs light from the flash, the inward
flow of sodium ions is slowed.
The sodium current in darkness reduces the negative
electrical polarization of the cell interior.
When light blocks the inward flow,
the negative voltage difference between the inside and outside
of the cell increases.
Thus, the initial photoreceptor response
to light is a {\em hyperpolarization}.
After the initial blockage of inward
flowing sodium current,
the current flow is actively restored.
The mechanism that restores balance overcompensates;
during the second phase of the response
the total photocurrent flow reverses direction.
Thus the photocurrent response first flows in one
direction and then the opposite direction,
leading to the biphasic impulse response.

In this experiment the test light is the input, $\test$,
and the cone photocurrent response is the output.
We can evaluate whether the input-output
relationship satisfies one of the requirements of a linear
system, homogeneity,
from the graphs in Figure~\ref{f3:cone.timecourse}.
Suppose the input signal is $\test$
and the photocurrent response is $\current$,
aa vector representing the photocurrent as a function
of time following the stimulus.
To test homogeneity we should measure the
response to the scaled the input, $k \test$.
If the system is linear, then we
expect that the photocurrent response will be
$k \current$.
From a visual inspection of the curves
in Figure~\ref{f3:cone.timecourse} we can see that
homogeneity fails.
There are two features of the curves
that should make this evident to you.
First, notice that as the test intensity increases, 
the peak deviation reaches a maximum of about 25 pA 
and then saturates.
Saturation is inconsistent with a strictly linear relationship
between input intensity and output photocurrent.
A second way to see that linearity fails is to consider
the point of the biphasic response at which the output
crosses the zero level at baseline.
If the output photocurrent is proportional to the input intensity,
points with a zero response level should always have a zero response level:
multiplying zero by any intensity still yields zero.
Hence, we expect that the zero-crossing
should not change its position as we increase the test intensity.
This prediction is true for lower test intensities,
but as the input intensity increases to fairly high levels,
the zero-crossing shifts its position in time.

How surprising: Human performance in the color-matching experiment
satisfies the principles of a linear system, homogeneity and
superposition,
yet the cone photocurrent responses a part
of the chain of biological events that
mediate the behavior, fail the simplest tests of linearity.
How can the behavior be linear when the components mediating
the behavior are nonlinear?
We will answer this question in the following section.
The answer is given specifically
for color-matching, but
the principles we will review are quite general.
They will be helpful again when we consider
the relationship between behavior and other neural responses
throughout this book.

\subsection*{Static Nonlinearities:  Photocurrents and Photopigments}
\begin{figure}
\centerline {
 \psfig{figure=../04col/fig/univariance.ps,clip=,height=2.0in}
}
\caption[Univariance of Photocurrent]{
{\em The principle of univariance} states that
a photon absorption leads to the same neural response,
no matter what the wavelength of the photon.
The principle predicts that two stimuli at different wavelengths
can be adjusted to equate
the photocurrent response throughout its timecourse.
This is shown here as the
match between photocurrents in response to
550 nm (dashed) and 659 nm (solid)
test lights set to a nine to one intensity ratio (Source: Baylor et al., 1987).
% Figure 2B, p. 150 from Baylor et al. 1987 J. Physiol. is part of this.
}
\label{f3:univariance}
\end{figure}
By comparing the sets of photocurrent responses on the top
and bottom of Figure~\ref{f3:cone.timecourse}, 
it appears that as we vary the
level of the test signal we sweep out the same set of curves.
The similarity of the measured photocurrent responses to
the two test lights suggests that we can perform a color-matching
experiment at the level of the photocurrent response.
We can perform such an experiment by choosing a test
light and a primary light
and adjusting the intensity of the primary
light light until the
photocurrent responses of the test and primary are the same.
The curves in Figure~\ref{f3:univariance}
show one example of such a match
using a primary light at 500 nm and a test light at 659 nm.

\begin{figure}
\centerline {
\psfig{figure=../04col/fig/phot.homogeneity.ps,clip=,height=2.0in}
}
\caption[Homogeneity of Photocurrent]{
{\em A matching experiment using the cone photocurrent as response.}
Lights at different wavelengths have equivalent effects
on the cone photocurrent when the light
intensity ratio is set properly.
For this cone,
the 659 nm light must
be nine times more intense than the 500 nm light
to have an equivalent effect (Source: Baylor et al., 1987).
%Figure 2A from Baylor et al. 1987
}
\label{f3:phot.homogeneity}
\end{figure}
The physiological preparation is very delicate and
it is difficult to keep the photoreceptors alive and functioning.
This makes it
impossible to set full photocurrent matches
for arbitrary test and primary combinations.
But, it is possible to carry out
an efficient approximation of the full experiment.
The two curves in Figure~\ref{f3:phot.homogeneity}
summarize the photocurrent responses to a
659 nm test light and the 500nm primary light
at a series of different intensity levels.
The data points shows the peak value of the photocurrent response
as a function of intensity;
the peak value summarizes the full photocurrent timecourse.
The smooth curves drawn through the
points interpolate the peak response at any
intensity level.
From these interpolated measurements,
we can estimate the intensity levels needed to
obtain complete matches
between the test and primary lights.

If the matching experiment performed at the
level of the photocurrent satisfies homogeneity,
the intensity of the test and primary lights that match
should be proportional to one another.
We can estimate the intensity of the test and primary lights
that match at different response levels by drawing a horizontal
line across the graph and noting the intensity levels
that produce the same peak photocurrent.
The curves through the two sets of data points
in Figure~\ref{f3:phot.homogeneity}
are parallel on a logarithmic intensity axis,
so that the intensities of pairs of points that match
are separated by a constant amount.
Since the axis is logarithmic,
equal separation implies that when the photocurrents
match the test and primary light intensities are
in a particular ratio, precisely as required by homogeneity.
Hence, the photocurrent matching experiment satisfies
homogeneity even though the photocurrent response itself is
nonlinear.

From the separation between the two curves,
we see that more 659 nm
photons are needed than 500 nm photons
to evoke the same response.
For this pair of wavelengths
the curves are separated by $0.97$ log units
that corresponds to a ratio of $9.3$.
It takes $9.3$ times as many 659 nm quanta to
equal the photocurrent produced by a number of 500 nm quanta.
By repeating this experiment using test lights
at many different wavelengths, we can estimate the 
complete spectral responsivity curves for
the cone photoreceptors.
From a collection of such measurements we can
estimate the wavelength sensitivity of the cone receptor.
The wavelength sensitivity is due to the properties
of the cone photopigment, so in this way
we can derive the cone photopigment
absorption function from the photocurrent measurements.

\begin{figure}
\centerline {
\psfig{figure=../04col/fig/cone.sp.sens.ps,clip=,height=2.5in}
}
\caption[Cone Photopigment Spectral Sensitivities]{
{\em Cone photocurrent spectral responsivities.}
The measurement range spans a factor of one million.
The $\Blue$ cone sensitivity at short wavelengths is high compared
to behavioral measurements because in behavioral conditions
the lens absorbs short wavelength light strongly.
(After Baylor, 1987).
% This is Figure 15, p. 45 in the Baylor Proctor lecture.
}
\label{f3:cone.sp.sens}
\end{figure}
The reader will not be surprised to learn
that Baylor, Nunn and Schnapf found cones with
three distinct spectral response functions:
these measurements are plotted
in Figure \ref{f3:cone.sp.sens}.
Notice that the vertical axis spans six logarithmic units,
so that they measured sensitivities varying
over a factor of one million,
an extraordinary technical measurement achievement.

\subsection*{Static Nonlinearities:  The principle}
We can analyze the photopigment sensitivity
from the photocurrent response because the 
nonlinear relationship between the test
light and the photocurrent signal is very simple:
The photons are absorbed
by a linear process;
the linear encoding is followed by a nonlinear process
that converts the photopigment absorption rate into
membrane photocurrent.
The properties of the nonlinear process
are independent of the linear encoding step, and thus
we call this process a {\em static nonlinearity}.
When a system is a linear process followed by a static
nonlinearity,
we can characterize the system performance completely.

It is worth spending a little time thinking about
why we can characterize this type of nonlinear system.
First, consider the linear process of photopigment absorption.
There is a photopigment system matrix, say, $\Rhodopsin$,
that maps the test light into a photon absorption rate,
$\Rhodopsin \test$.
Second, the static nonlinearity converts the photopigment
absorption rate into a peak photocurrent response.
Together, these two processes define
the nonlinear system response, $F( \Rhodopsin \test )$.

When we set a match between the peak photocurrent
from the test light and the primary light, we establish
an equation of the form
\begin{equation}
F( \Rhodopsin \test) = F( a \Rhodopsin \primary ),
\end{equation}
where $a$ is the intensity of the primary light
needed to match the test light.
Since the nonlinear function $F$ is monotonic,
we can remove it from both sides of the Equation and write
\begin{equation}
\label{e3:static.homogeneity}
\Rhodopsin \test = a \Rhodopsin \primary .
\end{equation}
From this equation we see that there is a linear relationship
between the primary and test light
intensities, since
if $\test$ matches $a \primary$, then $k \test$ will
match $k a \primary$.
Thus, even if a system has a static nonlinearity,
the system's performance in a matching experiment will
satisfy the test of homogeneity.
We can also show that in a matching experiment
a system with a static nonlinearity will satisfy superposition.

\subsection*{Cone Photopigments and Color-matching}
How well do the spectral sensitivity of the cone photopigments
predict performance in the photopic color-matching experiment?
We predict that two lights are metamers
when they have the same effect on the three types of cone
photopigments.
To answer how well the cone photopigments predict the
color-matching results, we can perform the following
calculation.

Create a matrix, $\Iodopsin$, whose three rows are the
cone photopigment spectral sensitivities.
We use this matrix to calculate the cone absorptions
to a test light, so that $\Iodopsin \test$,
is a $3 \times 1$ vector
containing the cone sensitivities to the test light.
We predict that
two lights $\test$ and $\test '$ will be a visual match
when they have equivalent effects on the cone photopigments.
Thus, two lights will be metamers when
\[
\Iodopsin \test = \Iodopsin \test' .
\]
It follows that the cone absorption matrix $\Iodopsin$
is a system matrix for the color-matching experiment.
This is precisely what we mean when we say that the
cone photopigments can explain the color-matching experiment.
Earlier in this chapter, we showed that
the color-matching functions are all related
by a $ 3 \times 3$ linear transformation.
Thus, there should be a linear transformation, say $\bf Q$,
that maps the cone absorption curves
to the system matrix of the
color-matching experiment, namely
$\Photopic = {\bf Q } \Iodopsin$.

\begin{figure}
\centerline {
\psfig{figure=../04col/fig/cone.to.cie.ps,clip=,height=2.5in}
}
\caption[Color Matching and Photocurrent]{
{\em Comparison of cone photocurrent responses and the
color-matching functions.}
The cone photocurrent spectral responsivities
are within a linear transformation
of the color-matching functions, after a correction has been made
for the optics and inert pigments in the eye.
The smooth curves show the Stiles and Burch (1959)
$10^\circ$ color-matching functions.
The symbols show the matches predicted from the photocurrents
of the three monkey cones.
The predictions included a correction for absorption by the
lens and other inert pigments in the eye (Source:  Baylor, 1987).
% Figure 4B, p. 154
}
\label{f3:cone.to.cie}
\end{figure}
Baylor, Nunn and Schnapf made this comparison\footnote{
After correcting for the absorptions by the lens and
inert pigments in the eye.}.
They found a linear transformation to convert
their cone photopigment measurements into the color-matching functions.
Figure~\ref{f3:cone.to.cie} shows
the color-matching functions along with
the linearly transformed cone photopigment sensitivity curves.
From the agreement between the two data sets,
we can conclude that the photopigment spectral responsivities
are a satisfactory biological basis
to explain the photopic color-matching experiment.

\subsection*{Why this is a big deal}
The methods we have
used to connect cone photopigments
and color-matching are a wonderful example
of how to relate
physiological and behavioral data precisely.
To make the connection between the behavioral
and physiological data we have had to reason
through some challenging issues.
Still, we have obtained a close quantitative agreement
between the behavioral and physiological measurements.

Notice that as we began our analysis,
the properties of the neural measurements and the
behavioral measurements appeared different.
The linearity of the color-matching experiment contrasts
with the nonlinearity of the photocurrent response.
But by comparing stimuli that cause equal-performance responses, 
rather than comparing behavioral matches
with raw photocurrent levels,
we can see past the dissimilarities and understand
their profound relationship.
In this case, we know how to connect these two
different measurements and the simplicity and clarity
of the relationship is easy to see.
It makes sense, then, to ask what we can learn from
this successful analysis that might help us when
we move on to try to relate other
behavioral and biological measurements.

We should remember that the relationship
between behavior and biology may not always
be found at the level of the measurements that
are natural within each discipline.
Direct comparisons between the intensity of the
primary lights and the photocurrent signals do not
help us to explain the relationship, even though each
measure is natural within its own experiment.
To make a deep connection we needed to look at the
structural properties of the experiment.
When we perform the color-matching experiment,
we learn about the equivalence of different stimuli.
This equivalence is preserved
under many transformations.
Thus, we succeed at comparing the behavior and the
biology when we compare the results at this
level, although they seem different when we use
other quantitative measures.

How do we know when we have the right set of biological
and behavioral measures?
There are many related physiological measures
we might use to characterize the photoreceptors,
and there are many variants
of the behavioral color-matching experiment.
For example, we could have asked the subject whether the
brightness of the test light doubles when we double
the intensity (the answer is no).
Or we could have asked the subject to assess the change
in redness or greenness.
Just as the input-output relationship of the photocurrent
may violate linearity for intense stimuli,
so too many behavioral measures violate linearity.
Finding the right measures to reveal the common properties
of the two data sets is in part science and in part art.
We learn about connections between
these disciplines by trying to recast our experiments
using different methods until the relationships become evident.

As we study the neural response in more central parts of
the nervous system, you may be tempted
to interpret a physiological measurement
as a direct predictor of some percept.
The rate at which a neuron responds or the stimulus
that drives a neuron powerfully
are natural biological measures.
Remember, however, that
there is no simple relationship
between the photocurrent response
and the intensity level of a primary light.
We achieved a good
link between the physiological and behavioral measures
by structuring a theory of the information that is preserved
in each set of  experimental measurements.
Understanding our measurements in terms of this level
of abstraction -- what information is present in the
signal -- is a harder but better way to forge links
between different disciplines.
Color science has been fortunate to have workers in both
disciplines who seek to forge these links.
We should take advantage of their experience
when we relate behavior and biology in other domains.

\section{Color Deficiencies}

I have emphasized the fact that for most observers color-matching
under the standard viewing conditions requires
three primary lights to form a match and we call
color vision {\em trichromatic}.
There are some viewing conditions in which
only two different primary lights are necessary.
Under these viewing conditions, color vision is {\em dichromatic}.
Finally, when only a single primary is required, as under
rod viewing conditions, performance is {\em monochromatic}.

\subsection*{Small field dichromacy}
Perhaps the most important of case of dichromacy occurs when
we reduce the size of the bipartite
field used in the color-matching experiment.
If the field is greatly reduced in size, from
2 degrees to only 20 minutes of visual angle,
then observers no longer need three independent
primary lights:
two primary lights suffice.
Under these circumstances, observers act as if they have
only two classes of photoreceptors rather than three.

Why should observers behave as if they had only two
classes of receptors when the field is very small?
If this observation surprises you, go back to
Chapter~\ref{chapter:mosaic}
and re-read the section on the $\Blue$ cone mosaic.
You will find that there are very few short-wavelength cones,
and there are none
in the central fovea.
Oddly,
we encode less about the spectral properties of the incident
light in the central fovea
than we record just slightly peripheral to the fovea.
In the central fovea, people are dichromatic.

\subsection*{Dichromatic observers}
Some observers find that they can perform the color-matching
experiment using only two primary lights
throughout their entire visual field.
Such observers are called {\em dichromats}.
The vast majority of dichromats are male.
By studying the family relationships of dichromats, it has
been found dichromacy is a sex-linked genetic trait
(Pokorny, Smith, Verriest, and Pinckers 1979).
Dichromatic observers can be missing the
long-wavelength photopigment ({\em protanopes}),
the middle-wavelength photopigment ({\em deuteranopes}),
or short-wavelength photopigment ({\em tritanopes}).
Tritanopes are much more rare than either protanopes
or deuteranopes.
The difference in the probabilities
arises because
the gene responsible for the creation of the short-wavelength
photopigment is on a different chromosome (Nathans, et al., 1992).

It is possible to use the color-matching functions measured
from dichromatic observers to estimate the photoreceptor
spectral responsivities.
Suppose we have two dichromatic observers:  the first observer
has only the $\Red$ and $\Green$ photoreceptors,
and the second observer
has only the $\Red$ and $\Blue$ photoreceptors.
Since the photoreceptor sensitivities are linearly related
to the color matching functions, a weighted sum of the
first observer's color-matching functions
will equal the $\Red$ cone absorption function,
and a different weighted sum of the second
observer's color-matching functions
will equal the $\Red$ cone absorption function, too.
This establishes a linear
equation we can use
to estimate the $\Red$ cone absorption function.
Similarly, from a pair of dichromats who share
only the $\Green$ cones,
we can estimate the $\Green$ cone sensitivity, and so forth.

\paragraph{Pseudoisochromatic Plates}
For some purposes,
we do not need the complete
results of a color matching experiment to
learn about the observer's color vision.
A much simpler test for dichromacy is to have a subject
examine a set of colored images
called the {\em Ishihara Plates}.
These plates were designed based on the results of the
color matching experiment and can be used to identify
different types of dichromats
based on a few simple judgments.

Each plate consists of a colored test
pattern drawn against a colored background.
The test and background are both made up of circles
of random sizes;
the test and background are distinguished
only by their colors.
The color difference on each plate is
invisible to one of the three classes of dichromats.
Hence, when a subject fails to see the test pattern, 
we conclude that the subject is missing
that cone class.

\paragraph{Farnsworth-Munsell 100 Hue test}
\begin{figure}
\centerline{
 \psfig{figure=../04col/fig/farnsworth.ps,clip=,height=3.0in}
}
\caption[Farnsworth Error Plot]{
{\em Representing errors in the Farnsworth-Munsell 100 Hue test.}
Each of the test objects, called {\em caps},
is assigned a position around the circle.
The error score is indicated by the radial distance of the
line from the center of the circle.
Observers with normal color vision rarely have an error
score greater than two.
Errors characteristic of an observer missing the $\Red$ cone photopigment
(protanope), the $\Green$ cone photopigment
(deuteranope) and the $\Blue$ cone photopigment
(tritanope) are shown.  (Source:  Kalmus, 1965).
}
\label{f8:farnsworth}
\end{figure}
The {\em Farnsworth-Munsell 100 Hue test} is also
commonly used to test for dichromacy.
In this test, which is much more challenging than the Ishihara plates,
the observer is presented with a collection of
cylindrical objects, roughly the size of bottle caps
and often called {\em caps}.
The colors of the caps can be organized into a hue circle,
from red, to orange, yellow, green, blue-green, blue, purple and back to red.
Despite the name of the test,
there are a total of 85 caps, each numbered
according to its position around the hue circle.
The color of the caps differ by roughly equal perceptual steps.

The observer's task is to take a random
arrangement of the caps and to place
them into order around the color circle.
At the beginning of the task,
four of the caps (1,23,43, and 64) are used to establish
anchor points for the color circle.
The subject is asked to arrange the remaining
color caps ``to form a  continuous series of colors.''

The hue steps separating the colors of the
caps are fairly small;
subjects with normal color vision often make mistakes.
After the subject finishes sorting the caps, the experimenter
computes an error for each of the 85 positions along the hue circle.
The error is equal to
the sum of the differences
between the number on the cap and its neighbors.
For example, in a correct series, the caps are ordered, say 1-2-3.
In that case the difference between the cap
in the middle and the one on the left is -1,
and the one on the right is +1. 
The error score is 0 in this case.
If the caps are ordered 1-3-2,
the two differences are +2 and +1 and the error is 3.
Normal observers do not produce an error greater than 2
or 3 at any location.

The subject's error scores are plotted at 85 positions
on a circular chart as in Figure~\ref{f8:farnsworth}.
An error score of zero plots at the innermost circle
and increasing error scores plot further away from the center.
Subjects missing the $\Red$ cones (protanopes),
$\Green$ cones (deuteranopes), and $\Blue$ cones (tritanopes)
show characteristically different error patterns
that cluster along different portions of the hue circle.

\subsection*{Anomalous Observers}
Dichromatic observers have only two types of cones.
The slightly larger population of observers,
who are called {\em anomalous},
have three types of cones and require three primaries
in the color-matching experiment.
The matches that they set are stable,
but they are well outside of the range
set by most of the population.
These observers have cone photopigments
that are slightly different in structure from
most of the population, which is why they are
called anomalous.
The color-matching functions for anomalous observers are
not within a linear transformation of the  normal color-matching functions.
This is equivalent to the experimental observation that
lights that visually match for these observers do
not match for normal observers, and vice versa.

Neitz, Neitz and Jacobs (1993) have argued on genetic
grounds that many people contain small amounts of the 
anomalous photopigments
so that there are more than three cone photopigments present 
in the normal eye.
Because the anomalous photopigments are not very different
from the normal, it is hard to discern their presence
in all but the most sensitive experimental tasks.
They attribute the trichromatic behavior in the color-matching
experiment to a neural bottleneck rather than 
a limit on the number of photopigment types.
Since the differences between the normal and anomalous photopigments
are very small, however, this hypothesis will be difficult
to prove or disprove and it will have very little impact
on color technologies.

The relationship between anomalous observers and
normal observers parallels the relationship
between color cameras and normal observers.
The spectral responsivities of the
color sensors in most color cameras
differ from the spectral responsivity of the human
cone photoreceptors.
Worse yet, the camera sensors are not within a
linear transformation of the cone photopigments.
As a result, lights that cause the same effect on the
camera, that is lights that are visual matches
when measured at the camera sensors,
may be discriminable to the human observer.
Conversely, there will be pairs of lights that are
visual matches but that cause different responses in
the camera sensors.
I will return to discuss this topic when we return
to discuss color appearance in Chapter~\ref{chapter:Color}.

\section{Color Appearance}

Color-matching provides a standard of precision to strive
for when we analyze the relationship
between behavior and physiology.
The work in color-matching is also important because
it has had an impact well beyond basic science, into
engineering and technology that touch our lives.

The success of color-matching and its
explanation is so impressive,
that there is a tendency to
believe that color-matching explains more than it does.
The theory and data of photopic
color-matching provide a remarkably
complete explanation of when two lights will
match.
But, the theory is silent about what the lights look like.

Often, students who are introduced
to color-matching for the first time
are surprised that the words 
brightness, saturation and hue never enter the discussion.
The logic of the color-matching
experiment, and what the color-matching experiment tells
us about human vision, does not speak to
color appearance.
What we learn from color-matching is fundamental,
but it is not everything we want to know.
For many purposes we want to know 
An understanding of color-matching is necessary
for an understanding color appearance;
but, it is not a solution to the problem.

\begin{figure}
\centerline {
\psfig{figure=../04col/fig/albers.ps,clip=,height=3.0in}
}
\caption[Albers Cross]{
{\em Color-matching does not predict color appearance.}
The X's are physically the same (notice where they join)
and thus have the same effect on the photopigments;
but, their appearance differs.
The photopigment responses at a point do not
determine the color appearance at that point.
Appearance instead depends on the spatial structure in the image.
(Source: Albers, 1975).
}
\label{f3:albers}
\end{figure}
To emphasize the separation between color-matching
and color appearance, consider the following experiment.
Suppose that we form a color-match between two lights
that are presented as a pair of crossing
lines against one background.
Such a pair is illustrated on the left hand side
of Figure \ref{f3:albers}.
On the left, the lines both appear gray.
Now, move this pair of lines
to a new background.
Color-matching assures us that the two lights will
continue to match one another as we move them about.

But we should not be assured that their appearance
remains the same.
For example, on the right of the figure we find that the
pair of lights
now have quite a different color appearance.
By examining the point where the
lines come together at the top
of Figure \ref{f3:albers},
which is a painting by the artist Joseph Albers,
you can see that the lines are physically identical
on both sides of the painting.

Color-matching is different from color appearance.
To build theories of color appearance
we will need to
incorporate experimental factors -- such as the viewing context --
that are not included in either the theory or experimental
manipulations of the color-matching experiment.
It is precisely because the important discoveries
recounted in this chapter do not solve the problem of color appearance
that the chapter is so oddly titled.
We will review the topic of color in Chapter \ref{chapter:Color}.
