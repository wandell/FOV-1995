\chapter{Wavelength Encoding}
\label{chapter:wavelength}
\section{Introduction}

\begin{figure}
\centerline {
\psfig{figure=../04col/fig/newton.ps,clip=,height=3.50in}
}
\caption[Newton's Experiments]{
{\em Newton's summary drawing} of his experiments with light.
Using a point source of light and a prism, Newton separated sunlight
into its fundamental components.
By reconverging the rays, he 
also showed that the decomposition is reversible.
% RL Gregory's book Eye and Brain, Psychology of Seeing and Princeton
% University press, 1990, Fig. 2.4 p. 29
}
\label{f3:newton}
\end{figure}
Sir Isaac Newton's sketch in Figure~\ref{f3:newton}
summarizes his investigations into the
properties of light.
In these experiments, Newton separated daylight into
its fundamental components
by passing it through a prism and creating a rainbow.
Newton's demonstration that
light can be decomposed into rays
of different wavelength is at the foundation
of our understanding of light and color.

To perform these experiments, Newton placed a shutter 
containing a small hole in the window 
in his room at Cambridge.
The light emerging from the hole in the window
shutter served as a point source to illuminate his apparatus.
The key elements of the
apparatus are featured prominently in the center
of the figure: the lens and prism.
Newton's drawing shows that when
the daylight passed through the prism,
it formed an image of a rainbow on his wall.
With two experimental manipulations,
he showed that the components of the rainbow
were fundamental constituents of light.
In the upper left of the sketch,
we see a series of holes that Newton drilled in the wall
permitting part of the rainbow to continue through
to a second prism.
This ray of light was cast upon a second surface,
but the new image did not produce a second rainbow;
rather, as Newton wrote:
\begin{quote}
the color of the light was never changed in the least.
If any part of the red light was refracted, it remained totally
of the same red color as before.
No orange, no yellow, no green or blue, nor other new color was
produced by that refraction.
(Newton, Opticks)
\end{quote}
From this experiment, Newton concluded that
the pass through the first prism had
separated the daylight into its fundamental components.
No further change was observed when the ray passed
through a second prism.

At the bottom of the sketch
Newton illustrated that the decomposition is reversible:
passing light through the prism does not destroy the
character of the light.
To show this Newton converged the rays
following their passage through the prism to form a new image;
he found that the color of the image same is the same as that of the source.
Newton concluded that
\begin{quote}
light being transmitted through the parallel surfaces of two prisms ...
if it suffered any change by the refraction by one surface, it lost
that impression by the contrary refraction of the other surface.
(Newton, Opticks)
\end{quote}
From the second experiment, he concluded that passage through
the prism had not destroyed, but merely revealed, the character of the light.

We now know that Newton succeeded in
decomposing the sunlight into its {\em spectral} components.
each with its own characteristic wavelength.
The prism separates the rays because the prism bends
each wavelength of light by a different amount.
(See the section on Snell's law in Chapter~\ref{chapter:mosaic}).
When we see the spectral components separately, they each
have a different color appearance.
Light with relatively long wavelengths
appears red when viewed against a dark background.
Light with relatively short wavelengths
appears blue when viewed against a dark background.
Shorter wavelengths of light are
refracted more strongly than longer wavelengths.
A spectral light, with energy only at a single wavelength
is also called a {\em monochromatic light}.

Newton's apparatus suggests a simple device we might
build to measure the amount of power a light
has in each of the different wavelength bands.
As illustrated on the top of Figure \ref{f3:spectroradiometer},
by proper use of lenses and prisms,
we can form a focused image of the
spectral components in an image plane with a movable slit
placed in front of a photodetecting sensor.
To measure the energy at different wavelengths,
we move the slit passing only
some of the spectral components at each position,
and thus we measure the energy of the source at
different wavelengths of light.
In the visible region,
the wavelength of light is on the order of a few
hundred billionths of a meter, or {\em nanometers} (nm).
%height = 7.0in width = 7.5in
\begin{figure}
\centerline {
\psfig{figure=../04col/fig/spectroradiometer.ps,clip=,height=2.0in}
}
\caption[Spectro-radiometer]{
{\em A spectroradiometer}
is used to measure the spectral power distribution of light.
(a) A schematic design of a spectroradiometer
includes a means for separating the input light into
its different wavelengths and a detector for measuring
the energy at each of the separate wavelengths.
(b) The color names associated with the
appearance of lights at
a variety of wavelengths are shown (After Wyszecki and Stiles, 1982).
% Figure 1(1.5.1) p. 64
}
\label{f3:spectroradiometer}
\end{figure}

The {\em spectral power distribution} of a light
is the function that defines the power
in the light at each wavelength.
In the modern theory of physics, the wavelength of light
can be thought of in two different ways.
We describe the light as if it were
a continuous wave as it passes through a medium.
When the light exchanges energy with some material,
say by giving up its energy to be absorbed,
we describe the light as if it were a discrete object
called a {\em photon} or {\em quantum} of light.
The amount of energy given up by the photon is predicted
by the wavelength of the light.

The experimental aspect of light measurement
that makes it useful and predictable
is that the measurement
satisfies the principle of superposition.
We can demonstrate the superposition of light
measurement as follows.
First, measure the spectral power distributions
of two lights separately.
Then, mix the two lights together and measure again.
The spectral power distribution of the mixture
will be the sum of the first two spectral power distributions.
This property of light mixture
is illustrated in Figure~\ref{f3:ill.superposition}.
Superposition is a crucial property of light measurement
because it implies that we can
measure the energy of a light at each wavelength separately,
and then combine the individual measurements
to predict spectral power distribution
when the spectral components are mixed together.
\begin{figure}
\centerline {
\psfig{figure=../04col/fig/ill.superposition.ps,clip=,height=3.0in}
}
\caption[Light Superposition]{
{\em The measurement of light spectral power
distributions} satisfies the principle of superposition.
The spectral power distributions of two lights measured
separately are shown in (a) and (b) and together in (c).
The spectral power distribution of the mixture is the
sum of the individual measurements, thus demonstrating that
superposition holds true.
}
\label{f3:ill.superposition}
\end{figure}

Suppose we wish to measure the spectral power
distribution of a light source.
How many wavelengths should we measure?  Or, equivalently,
how finely do we have to sample
along the wavelength dimension?
The answer to this question is important for both practical
and theoretical reasons because the number
of samples can be quite large.
For example, to sample the visible spectrum
from 400 nm to 700 nm in 1 nm steps,
we need about 300 measurements.
To sample in 10 nm steps,
we need about 30 measurements.

\begin{figure}
\centerline {
 \psfig{figure=../04col/fig/spectra.ps,clip=,height=2.0in}
}
\caption[Spectral Power Distributions]{
{\em The spectral power distribution}
of two important light sources are shown:
blue skylight (a) and the yellow disk of the sun (b).
}
\label{f3:spectra}
\end{figure}
The answer to this sampling question depends on the same
set of issues as the sampling questions
we addressed in Chapter~\ref{chapter:mosaic} on the
spatial sampling of the retinal image by the photoreceptor mosaics.
If the energy in the light varies rapidly as a function
of wavelength, then 
we may have to sample quite finely to measure accurately;
if the functions vary slowly, then only a few
measurements are necessary.
Also, the precision of the representation
requires that we know how sensitive the photopigments
in the are to rapid changes in the energy as a function
of wavelength.
It is difficult to make accurate generalizations about
how spectral power distributions vary as a function of wavelength,
but it is believed widely that for practical purposes
we can approximate
spectral power distributions
using smooth, regular functions as shown
in Figure~\ref{f3:spectra}.
Also, it is known that the photopigments integrate broadly
across the wavelength spectrum.
Consequently, international standards organizations suggest making
measurements every 5 nm to achieve an excellent representation
of the signal.
Practical measurements often rely on measurements spaced
every 10 or 20 nm.
We will consider
this issue much more completely
when we review color appearance, in Chapter~\ref{chapter:Color}.

\section{Scotopic Wavelength Encoding}

What information do we encode about
the spectral power distribution
when rods initiate vision, under {\em scotopic} conditions?
We can answer this question by an experiment designed
to measure how well people
can discriminate different spectral power distributions.
In the {\em scotopic matching} experiment, we present an
observer with two lights, side by side in a {\em bipartite} field.
One side of the field contains the {\em test} light;
it may have any spectral power distribution whatsoever.
The second side of the field contains the {\em primary} light;
it has a fixed relative spectral power distribution
and can vary only by an overall intensity factor.
The observer's task in the {\em scotopic matching experiment} is
to adjust the primary light intensity
so that the primary light appears indistinguishable from the test light.
The observer can adjust only the intensity of the primary light,
so when the match is achieved the spectral power distributions of the
test and primary lights that match are still different.

Under scotopic conditions, observers can adjust
the primary intensity so that
the primary matches any test light.
Since subjects can always make this match,
we have a simple answer to our question:
The rods encode nothing about
the relative spectral density of a light.
An observer can adjust the intensity of
a primary light to match the appearance of
a test light with any spectral power distribution.
The relative spectral power distribution is immaterial,
all that matters is the relative intensities of the two lights.

\subsection*{Matching: Homogeneity and superposition}

We can learn more about scotopic wavelength
encoding by studying the quantitative properties
of the matching experiment.
To characterize the matching experiment completely,
we must be able to predict how a subject will adjust
the primary intensity to match any test light.
We treat the experiment as a transformation by identifying
the spectral power distribution of the test light as the input
and the intensity of the primary light as the output.
A quantitative description of the experiment tells us
how to map the input to the output.

Naturally, we first ask whether
we can characterize the matching experiment
transformation using linear systems methods.
Denote the spectral power distribution
of the test and primary lights using the vectors
$\test$ and $\primary$ respectively.
The $\nl$ entries of these vectors describe the power
at each of the $\nl$ sample wavelengths.
To test linearity,
we evaluate whether the scotopic matching experiment
satisfies the linear systems properties of
homogeneity and superposition.
We can evaluate these properties from the following experimental tests:

\begin{itemize}
\item  (Homogeneity) If $\test$ matches $\primaryinti{} \primary$, 
will $a \test$ match $ a (\primaryinti{} \primary)$?
\item (Superposition)
If $\test$ matches $\primaryinti{} \primary$, and
${\test '}$ matches ${\primaryinti{} '} \primary$, will 
$\test + \test '$ match $( \primaryinti{} \primary ) + ( \primaryinti{} ' \primary )$?
\end{itemize}

\begin{figure}
\centerline {
\psfig{figure=../04col/fig/scotopic.ps,clip=,height=2.0in}
}
\caption[Scotopic Color Matching Linearity]{
{\em Hypothetical scotopic matching experiment.}
The horizontal scale measures the intensity of a monochromatic
test light and the vertical scale measures the
intensity a matching 510 nm primary light.
Since the scotopic matching experiment satisfies homogeneity, the
data will fall along a straight line.
The slope of the line defines the
relative scotopic sensitivity to each test wavelength.
}
\label{f3:scotopic}
\end{figure}
An hypothetical test of homogeneity is shown
in Figure~\ref{f3:scotopic}.
The separate panels show the intensity of the test light
on the horizontal axis and the intensity
of the matching primary light on the vertical axis.
Each panel plots the results using
spectral test lights at a series of
wavelengths and  a 510 nm primary light.
In the scotopic matching experiment
the data will fall on a straight line, consistent with the
prediction from homogeneity.
The slope of the line defines the relative scotopic
sensitivity to the test and the primary lights.
For example, in panel (c) the hypothetical results from an experiment
with a 580 nm test light are shown.
The slope of the line shows
that we need $8.3$ units of energy
at 580 nm to have the same effect as one unit of
energy at 510 nm.
Hence, the light at 510 nm is $8.3$ times
more effective, per unit energy,
than the light at 580 nm.

%height = 5.5in width = 7.0in
\begin{figure}
\centerline {
\psfig{figure=../04col/fig/scotopic.mat.ps,clip=,height=2.0in}
}
\caption[Scotopic System Matrix]{
{\em Matrix tableau of the scotopic matching experiment.}
The primary light intensity, $e$,
equals the product of the $1 \times \nl$
scotopic matching system matrix,
$\Scotopic$, and the $\nl \times 1$ test light spectral
power distribution, $\test$.
}
\label{f3:scotopic.mat}
\end{figure}
Because the scotopic matching experiment is linear,
there must be a system matrix, $\Scotopic$ that maps the
input ($\test$, the test spectral power distribution),
to the output ($e$, the primary light intensity).
The system matrix, call it $\Scotopic$, must have one row and $\nl$ columns.
The test light, system matrix, and primary intensity
are related by the
product, $e = \Scotopic \test$.

We can relate the measurements in the scotopic matching
experiment to the entries of the system matrix as follows.
Write the matrix product $\Scotopic \test$
as a summation over the sample wavelengths,
\begin{equation}
\label{e2:sm}
e = \sum_{i=1}^{i=\nl} \Scotopici{i} \testi{i} 
\end{equation}
Suppose we use a monochromatic test light of unit intensity, 
that is, an input $\test$ that has only
a single non-zero wavelength, $(0,0,...,0,1,0,...0)^{T}$.
Then Equation~\ref{e2:sm} becomes simply $e = \Scotopici{i} \testi{i}$.
This shows that the slope of the
line relating the monochromatic test intensity, $\testi{i}$,
to the primary intensity, $e$,
is the system matrix entry, $\Scotopici{i}$.
Hence, we can estimate
the system matrix from the slopes of
the experimental lines we measure in the test of homogeneity
shown in Figure~\ref{f3:scotopic}.

Figure~\ref{f3:scot.sens} is a graphical method of representing
the system matrix of the scotopic matching experiment.
The curve shows the entries of $\Scotopic$ as a function of wavelength,
interpolated from experimental measurements at many sample wavelengths.
The curve is called the {\em scotopic sensitivity function}.
\begin{figure}
\centerline {
\psfig{figure=../04col/fig/scot.sens.ps,clip=,height=3.0in}
}
\caption[Scotopic Sensitivity]{
{\em The scotopic spectral sensitivity function} 
defines the human wavelength sensitivity under
scotopic viewing conditions.
The curve is a plot of the entries of the scotopic system matrix.
}
\label{f3:scot.sens}
\end{figure}

Once we measure the system matrix, $\Scotopic$,
we can predict whether any pair of lights will match under
scotopic conditions.
Figure~\ref{f3:scotopic.mat} shows how we use
the system matrix to predict the intensity of a primary light needed
to match a test light.
Suppose we have two test lights, $\test$ and $\test '$.
Two lights will match when they are matched by
the same intensity of the primary light.
So, these two lights will match when
$\Scotopic \test = \Scotopic \test '$.

\subsection*{Uniqueness}
The hypothetical experiment illustrated in
Figure~\ref{f3:scotopic} assumed a 510 nm primary light.
Suppose that we perform the scotopic matching experiment
using a different primary light.
How will this effect the system matrix, $\Scotopic$?

We can answer this question by a thought experiment.
Call the second primary light $\primary '$.
We can set a match 
between the new primary light,
${\primary '}$ and the first primary light $\primary$.
We will find that there is some scalar,
$k$, such that $k {\primary '}$ matches $\primary$,
and we expect that
whenever $a \primary$ matches a test light, $\test$, 
then $ a [ k \primary ']$  will match $\test$.
In particular, since $\Scotopici{i} \primary$ matches
the $i^{th}$ monochromatic test light,
we expect that $\Scotopici{i} k  { \primary '}$ will
match the $i^{th}$ monochromatic test light as well.
It follows that the entries of the new system matrix
will be $k \Scotopici{i}$,
equal to the original except for a constant scale factor, $k$.
Hence, the new system matrix will be
$k \Scotopic$, and
we say that the estimate of $\Scotopic$ is unique
up to an unknown scale factor.

\subsection*{The Biological Basis of Scotopic Matching}

\begin{figure}
\centerline {
 \psfig{figure=../04col/fig/alligator.ps,clip=,height=4.5in}
}
\caption[Rhodopsin in the Alligator]{
{\em Rhodopsin in the rod photoreceptors} of an alligator eye
in the dark-adapted and light-adapted states.
(a) After being in the dark,
the unbleached rhodopsin is visible as a purple substance.
(b) After exposure to light, the rhodopsin is bleached
and we see only the white, reflective tapetum.
(From Rushton, 1962).
% Figure in Sci. Am. which I got from Held and Richards
% volume of reprints Mechanisms and Models.
}
\label{f3:alligator}
\end{figure}
The scotopic matching experiment is remarkable in its simplicity.
We can understand the biological basis of the
experimental matches by studying
the properties of the rod photopigment, {\em rhodopsin}.
Figure \ref{f3:alligator}
is a photograph of
the photopigment contained in the rod outer segments.
In part (a) of the figure
the photopigment is photographed
in the eye of an alligator.
Because the back of the alligator's eye contains a white 
reflective surface, called the {\em tapetum}, it is possible
to see the color of the rod photopigment.
Cats too have a white tapetum,
which is why cats eyes appear to glow so brightly when they
catch the beam of a car's headlights.
The alligator shown in the
picture had been kept in a dark closet for 24 hours so
that the photopigment would be fully regenerated and
easy to photograph.
The closet was opened briefly, a flash picture taken,
and then I suppose the door was shut again. Whew.
\nocite{RushtonAlligatorSciAm}

Rod photopigment is present in much higher density
than any of the cone photopigments.
Thus, researchers have been able to
isolate and extract the rod photopigment
for fifty years,
whereas the cone photopigments have only become 
available recently through the methods of genetic engineering
(Merbs and Nathans, 1992).
Characteristically, when the rod photopigment
is exposed to light, it undergoes a series of
rapid changes in chemical state (Hubbard and Wald, 1951;
Wald and Brown, 1956; Wald, 1968).
Whenever a quantum of light is absorbed by the rhodopsin
photopigment, it undergoes a specific sequence of events
resulting in the decomposition of the rhodopsin
molecule into opsin and vitamin A.
Figure~\ref{f3:alligator}b
shows the same alligator after
its eye  has been exposed to light.
The rhodopsin is been broken into two parts
and the resulting products are clear, rather
than purple.
In this state, the white tapetum of the eye is evident.
It is the wavelength selectivity of the rhodopsin photopigment
that provides the biological basis of scotopic matching.
The relationship between the behavioral experiment and
the properties of the rod photopigment is based on
an important property called {\em univariance}.

W. Rushton emphasized that
when a photopigment molecule absorbs 
light, the effect upon the photopigment is the same
no matter what the wavelength
of the absorbed light might be.
Thus, even though quanta at 400 nm possess more energy
than quanta at 700 nm, the sequence of rhodopsin reactions
to absorption of a 400 nm quantum is the same as
the sequence of reactions to a 700 nm quantum.
Rushton used the word {\em univariance} for this principle
to remind us
that a single photopigment makes only a single-variable response
to the incoming light.
The photopigment maps all
spectral lights into a single-variable output, the 
{\em rate of absorptions}.
The response of a single photopigment
does not encode any information
about the relative spectral composition of the light.
This explains why we cannot discriminate between lights with
different spectral power distributions under scotopic viewing
conditions (Rushton, 1965; Naka and Rushton, 1966).

Univariance does not mean, however,
that the photopigment responds equally well to all spectral lights.
The photopigment is much more likely to
absorb some wavelengths of light than others.
Univariance asserts that once absorbed, however,
all quanta have same visual effect.

%height = 4.5 width = 6.5in
\begin{figure}
\centerline {
\psfig{figure=../04col/fig/self.screening.ps,clip=,height=2.0in}
}
\caption[Photopigment Measurement]{
{\em An apparatus to measure the spectral absorption
of a photopigment.}
Using the monochromator, one can select
light at one wavelength from the light source.
To estimate the fraction of photons absorbed by the photopigment
at that wavelength,
we divide the number of photons detected through the
glass and photopigment by
the number detected after passing through the glass alone.
}
\label{f3:self.screening}
\end{figure}
We can measure the probability of absorption
using the experimental apparatus
shown in Figure~\ref{f3:self.screening}.
We place a thin layer of photopigment on a clear plate of glass.
We create a monochromatic light
by passing the light from an ordinary source
through a {\em monochromator}.
The monochromator can be constructed using prisms or
diffraction gratings to separate the incident light
into its separate wavelengths, much as in Newton's original experiments.
We measure the amount of monochromatic light
passed through the photopigment and the glass plate
by means of a photodetector at the rear of the apparatus.
We then move the glass plate upwards,
to remove the photopigment from the light path,
and measure again.
The difference in the photodetector signal measured
in these two conditions
is proportional to the amount of light absorbed by the photopigment.

\begin{figure}
\centerline {
\psfig{figure=../04col/fig/scot.homogeneity.ps,clip=,height=1.7in}
}
\caption[Photopigment Homogeneity]{
{\em Rhodopsin absorptions at different wavelengths.}
The number of absorptions in a thin layer of
photopigment are proportional to the intensity of the input light
and thus satisfy the principle of homogeneity.
The slope of the linear relationship between the
light intensity and the number of absorptions describes
the fraction of photon absorptions.
The slope varies with the wavelength of the test light,
thus defining the photopigment wavelength sensitivity.
}
\label{f3:scot.homogeneity}
\end{figure}
If only a thin layer of photopigment is present,
the experimental measurements of the
absorptions will satisfy homogeneity and superposition.
To test homogeneity, we increase the intensity of the test light.
We will find that the number of absorptions will increase proportionately
over a significant range.
To test superposition, we measure the
photopigment absorptions to a test light
$\test$ to be $a$, 
and the number of absorptions to a second
light $\test '$  to be $a'$.
When we superimpose the two input lights,
we will measure $ a + a'$ absorptions.
Since the measurement process is linear,
we can estimate the system matrix of
this absorption process, $\Rhodopsin$,
just as we measured the system matrix,
of the scotopic matching experiment, $\Scotopic$.

\begin{figure}
\centerline {
\psfig{figure=../04col/fig/rhodopsin.sens.ps,clip=,height=3.0in}
}
\caption[Rhodopsin in the Eye]{
{\em Comparisons of scotopic matching and rhodopsin wavelength sensitivity.}
The filled circles show human rhodopsin absorption
measured as in Figure~\ref{f3:self.screening}.
The open circles show human scotopic sensitivity,
corrected for light loss at the lens and optical media.
(Source: Wald and Brown, 1958) 
%Original is Science, 1958 v. 127, p. 222
%
%	Remove the aphakic observer measurements?
%
%I took this from a review article in Assessment of Visual Function,
%Potts. article by Alpern, Figure 3-5.
%Also reprinted in Wald and Brown, Cold Spring Harbor Symposia on
%Quantitative Biology, 1965 v. 30 p. 345-361, 
}
\label{f3:rhodopsin.sens}
\end{figure}
We can predict the matches
in the scotopic matching experiment
from the absorptions of the rhodopsin photopigment.
A test and primary light match in the scotopic matching
experiment when the two lights
create the same number of absorptions in the rhodopsin photopigment.
We can demonstrate this by comparing the system matrices
of the scotopic matching experiment and the rhodopsin absorption
experiment.
After we correct for the effects of the wavelength sensitive elements 
of the eye, mainly the lens,
we can plot the system matrices of the scotopic matching experiment
$\Scotopic$, and the rhodopsin absorption experiment, $\Rhodopsin$,
on the same graph.
Wald and Brown (1958) made this comparison
in the graph in shown Figure \ref{f3:rhodopsin.sens}.
The filled circles in the graph
plot the the measurements of the
system matrix from the rhodopsin absorption experiment, $\Rhodopsin$.
The completely open circles plot estimates of the entries of
$\Scotopic$ after correcting
for the fact that the lens absorbs a significant amount of light
in the short-wavelength part of the spectrum.

The agreement between the measurements of the
rhodopsin photopigment and the scotopic matching
experiment confirm a simple model of
the observer's behavior.
Under scotopic viewing conditions the observer's perception
of the two halves of the bipartite field depends
on a signal initiated by the rod photopigment absorptions.
The two sides of the field appear identical when the 
rhodopsin absorption rates on the
two sides of the bipartite field are equal.
During the experiment, then,
the observer adjusts the intensity of the matching light
to equalize the rod absorption rates on the two sides of the bipartite field.
Since the absorption of the light 
is a linear process,
the observer's behavior is linear, too.

The precise quantitative match between the scotopic matches
and the rod photopigment make
a very strong connection between performance and biological
encoding of the light.
This type of precise
quantitative relationship between behavior and
the biological encoding of light serves
as a good standard to use when we consider the relationship
between behavior and biology in other conditions.
