
%
%	Delete,change,below here
%


For each of these observers,
the $\Red$ receptor responsivity is the weighted sum
of these columns.
It follows that for each observer there is
a pair of weights, which we place in a two-dimensional
column vector, such that multiplying the vector and
the matrix equals
the responsivity of the $\Red$ photoreceptor
they share in common.
We express these relationships as

\begin{equation}
\label{e3:dichromacy1}
\Red =
\left (  \Photopic_{\Red,\Green } \right )^{t} 
\left ( \begin{array}{c} u \\ v  \end{array} \right )
  = 
\left (  \Photopic_{\Red,\Blue} \right )^{t}
\left (  \begin{array}{c}  w \\ x \end{array}\right )
\end{equation}

where $(u,v)$ and $(w,x)$ contain the weights for 
the tritanopic and deteranopic observers, respectively.

We can regroup the terms in this matrix equation
into one equation
\begin{equation}
\label{e3:dichromacy2}
{\bf 0} =
\left (  \Photopic_{\Red,\Green} \Photopic_{\Red,\Blue} \right )^{t}
\left ( \begin{array}{c} u \\ v \\ w \\ x  \end{array} \right )
\end{equation}
that contains the four unknown weights
in one vector on the right,
The four columns of color-matching functions
fill the columns of the matrix.
Since the matrix product is equal to zero, if $(u,v,w,x)$ is
a solution so is $k (u,v,w,x)$.
Equations of this form are called homogeneous linear equations.

We can pick a solution to Equation~\ref{e3:dichromacy2}
whose largest entry is 1.0.
Having all four weights, we can use
Equation~\ref{e3:dichromacy1}
to estimate the sensitivity of the $\Red$ receptors.
We can repeat this process two more times,
using observers that share in common each of the photoreceptor types,
and estimate each of the photoreceptor responsivities.
This is the basic procedure Smith and Pokorny and many others
have used to estimate the photoreceptor responsivities
in the living eye.
The estimates obtained this way,
from behavioral experiments,
are very close to the measurements by Baylor, Nunn and Schnapf.
