\comment
Orthogonality as a point about sinusoids
and cosinusoids.
Common vector length and orthogonality
makes the linear regression problem very easy
for solving the Fourier series.
Build the matrix.

Calculation of transfer when a phase
shift is possible.
}
\item Suppose the response of
a linear, shift-invariant system to
a cosinusoid is $A \sin(i) +  B \cos(i) $.
Using the fact that a sinusoid is a shifted copy of
a cosinusoid, express the response to a sinusoid in terms of $A$ and $B$.
(Use the fact that $\cos( - \pi /2 ) = 0$ and $\sin ( - \pi / 2 )  = -1$).
\comment{
sin(f) = cos(f - { \pi / 2} ) 
	-> A sin(f -  {\pi / 2}) + B cos(f -  {\pi / 2})  
	= A{ cos(f)sin(- {\pi / 2}) + cos(- {\pi / 2})sin(f)}
	  + B{ cos(f)cos(- {\pi / 2}) - sin(f) sin(- {\pi / 2})}
	= A{ cos(f) (-1) + 0 sin(f) }
	  + B{ cos(f) (0) - sin(f) (-1) }
 	= -A cos(f) + B sin(f)
	=  B sin(f) - A cos(f)
}

\begin{enumerate}

\item A $2N$ by $2N$ circulant matrix for
a shift-invariant linear system only has
$2N$ independent numbers in it.
These values can be found from the central column.
When we use a line as an input to a shift-invariant
linear system, we measure all $2N$ values in the response to 
a single stimulus.

\begin{enumerate}
\item How many independent measurements do we obtain when we
use a sinusoidal input?
\comment{ two}.
\item How many different sinusoidal frequencies will we have
to measure before we obtain the same
number of measurements as from a line?
\comment{N}
\item How many measured values do we obtain when we use
an eigenfunction as an input stimulus?
\comment{one}
\item How many eigenfunctions must we measure before
we have $2N$ independent values?
\comment{2N}
\end{enumerate}

