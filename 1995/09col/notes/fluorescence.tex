\paragraph{Fluorescence}
There are a number of other physical processes,
other than passive reflection,
that cause light to be radiated from materials.
For industrial materials, especially in the publishing industry,
fluorescence is the most important process.

Fluorescence refers to the phenomenon that when light
is absorbed at one wavelength, typical a high energy
short-wavelength light such as 370nm, the energy
in the quantum of light may cause changes in the
atomic orbital energy patterns that give rise to
the emission of a distribution of quanta at lower
energy levels, say around 600nm.
This is unlike the usual reflection process in which
we conceive of the light ray as bouncing around amongst
the elements of the material without being absorbed.
In fluorescence, a quantum of light from the
incident signal is absorbed.
Some of the energy continues as heat and some is
re-radiated at a lower energy level light signal.

Fluorescence is often a linear process.
If we measure the fluorescence due to absorption of
a 380nm light, and the fluorescence due to absorption
of a 390nm light, then we can predict the fluorescence
we will observe to the superposition of the two wavelengths.
As a result, we can describe a fluorescing surface by
a surface transfer matrix $F( \lambda_{out}, \lambda_{in})$
that describes how incident
radiation, $E( \lambda_{in} )$ creates a color signal
output, $C( \lambda_{out})$.

% 8.5 inches wide by 6 inches high
\begin{figure}
\centerline{
  \psfig{figure=../08col/fig/fluorescence.ps ,clip= ,width=4.25in,height=3in}
}
\caption[Fluorescence Matrix Tableau]{
(a)  The energy re-radiated from a red ink when
light at 400nm is incident.
Some light is reflected a 400nm, but a great deal
more of the light is re-radiated at the longer wavelengths.
(b) We can describe the transformation of an input
signal (E) to an output signal (C)
on a fluorescent surface by defining a transfer matrix, $F$.
The transfer matrix of a fluorescing surface describes
the light re-radiated as a function of each input wavelength.
Thus, the data in part (a) of this figure define one
column of the transfer matrix.
The off-diagonal terms describe the fluorescence component.
The diagonal terms describe the usual reflection components.
For certain types of materials the fluorescence distribution
remains invariant for different input signals, giving
the matrix a simple structure.
}
\label{f8:fluorescence}
\end{figure}

For many common inks, the fluorescence spectral power
distribution remains is the same
for various illuminant wavelengths.
In this case,
the fluorescence transfer matrix has a simple
structure that can be described more efficiently
than the $N \times N$ transfer matrix.

