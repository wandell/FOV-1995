%
%	References go here
%
\newpage
\section*{Exercises}

\be % All Questions

\item Apart from surface reflectance, what other aspects of the
material world might be represented by color vision?

\item Color vision has evolved separately in birds, fish and primates.  What
does this imply about the significance of wavelength information in the
environment?

\item We have used finite dimensional linear models of surfaces and
illuminants to understand how the visual system might infer surface
reflectance from the responses of only three cone types.
Answer the following questions about linear models.
 \be
 \item Offer a definition of a linear model that relates them to
the Discrete Fourier Series.

 \item Describe the relationship between linear models used in this
chapter and linear models used in spatial pattern sensitivity and
multiresolution image representations.

 \item List some reasons why light bulb manufacturers would be
interested in linear models for illuminants.

 \item Using a linear model for daylights, we can design a device
with two color sensors to estimate the relative spectral power
distribution of daylight.  In order to see how we could build such a
device, answer the following questions.

 \be
  \item Suppose a device has sensor sensitivities $\sensorMat$.
    Write an equation showing how the sensor responses depend on
    the illuminant model basis functions and the illuminant coefficients.
\comment{
\[
\recResp = \sensorMat^t \illBasis \illCoef
\]
}
 \item Describe why you can solve this equation for the linear model weights.

 \item Once you have calculated the linear model weights,
how do you calculate the estimated illuminant the spectral power distribution?

 \ee

 \ee

\item Answer these questions about the asymmetric color-matching
experiments. 

 \be

 \item What was J. Von Kries' hypothesis concerning the
biological mechanisms of color constancy?

 \item How did E. Wassef test this theory experimentally?

 \item What was E. Land's view of color science and its role in
 understanding color appearance?

 \item In his recent book, S. Zeki writes

\begin{quote}
Why did colour constancy play such a subsidiary role in enquiries on
colour vision?  Almost certainly because, until only very recently it
has been treated as a departure from a general rule, although it is in
fact the central problem of colour vision.  That general rule supposes
that there is a precise and simple relationship between the wavelength
composition of the light reaching the eye from every point on a
surface and the colour of that point. [Zeki, 1993, p.12]
\end{quote}

Zeki also quotes Newton's famous passage in support of his view.

\begin{quote}
Every Body reflects the rays of its own colour more copiously than the
rest, and from their excess and predominance in the reflect light has
its colour.
\end{quote}

Do you think Newton's passage supports Zeki's assertion that most
authors thought it was the light at the eye governs color appearance?
Or, does Newton's passage support Helmholtz' view that surface
reflectance determines appearance?  What do you think based on your
reading of this chapter?  Go to the historical literature and see what
others have written on the topic.

 \ee

\item Answer the following questions about color appearance and 
opponent-colors organization.

 \be

 \item Design a neural wiring diagram such that the system
will see yellow and blue at the same time, but not yellow and red.

 \item Given the strong chromatic aberration of the eye, what spatial
sensitivity properties would you assign to the blue-yellow neural pathway?

 \ee

\item W.S. Stiles developed a sophisticated quantitative
behavioral analysis of light adaptation to uniform fields.  His papers
were collected in a 1979 volume edited by J. Mollon entitled {\em
Mechanisms of Colour Vision}.  Stiles' work makes substantial use of
linear methods coupled with static nonlinearities.  Read the
experimental literature in this area and consider your answers to the
following questions.

 \be

 \item What role did homogeneity play in Stiles' analysis of the color
mechanisms?

  \item Which aspect of his measurements led him to believe that the
data might be revealing the action of independent cone pathways?

 \item As you read about Stiles' experiments, consider how you would
perform experimental tests of superposition.  Then, read the papers by
on the Stiles' mechanisms by Boynton, Mollon, Pugh, and others.  How
did they test superposition?

 \ee

\item Speculate about why prosopagnosia and cerebral dyschromatopsia
are frequently associated in patients.  Make sure that you list some
reasons that have to do with functional issues concerning visual
computation, and some that have to do with the organization of the
neural and vasculature properties related to vision.  Design
experimental tests of your hypotheses.

\ee % All Questions

