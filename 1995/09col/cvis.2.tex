\newcommand{\match}{\mbox{${\bf m}$}}
\newcommand{\matchi}[1]{\mbox{${\bf m}_{#1}$}}

\section{Color Constancy: Experiments}

\begin{quote}
In woven and embroidered stuffs the appearance of colors is profoundly
affected by their juxtaposition with one another (purple, for
instance, appears different on white than on black wool), and also by
differences of illumination.  Thus embroiderers say that they often
make mistakes in their colors when they work by lamplight, and use the
wrong ones.  [Aristotle, c. 350 B.C.E. Meteorologica].
\end{quote}

\subsection*{Asymmetric Color-matching Experiments}

To formulate some ideas about how the visual pathways compute color
appearance, we adopted the view that color appearance is a
psychological estimate of the surface reflectance function (body
reflectance).  By thinking about computational methods of estimating
body reflectance, we have discovered how the cone absorptions from an
object vary with illuminant changes.  Finally, we have seen that it is
possible compensate approximately for these changes by a applying
linear transformation to the cone absorptions.

From the computational analysis, we have discovered a good principle
to examine in experimental studies of color appearance: What is the
relationship between the cone absorptions of objects that appear the
same under different illuminants?  The computational analysis suggests
that the cone absorptions of lights with the same color appearance,
but seen under different illuminants, are related by a linear
transformation.

\begin{figure}
\centerline{
  \psfig{figure=../09col/fig/matchingMethods.ps,clip= ,width=5.5in}
}
\caption[Asymmetric Color-matching methods]{
{\em Two experimental methods for measuring asymmetric color-matches.}
(a)  In a memory matching method, the observer sees a target under
one illumination, remembers it, and then identifies 
a matching target after adapting to a second illumination.
(b)  In a haploscopic experiment the observer adapts 
the two eyes separately and makes a simultaneous appearance match.
The basic findings from these two types of experiments are the same.
}
\label{f8:matchingMethods}
\end{figure}
It is up to the experimentalist, then, to find an experimental method
to use for measuring the cone absorptions that correspond to the same
color appearance when seen in different viewing contexts.
Figure~\ref{f8:matchingMethods} illustrates two methods of making
such measurements.  Panel (a) shows a {\em memory matching} method.  In
this method, the subject studies the color of a target that is
presented under one illumination source and then must select a target
that looks the same under a second illumination source.  These
measurements identify the stimuli, and thus the cone absorptions, of
targets that appear the same under the two illuminants.  The drawback
of the method is that making such matches is very time-consuming
because the subject must adapt to the two illumination sources
completely, a process which can take two minutes or more.

Figure~\ref{f8:matchingMethods}b shows a second method called {\em
dichoptic matching}.  In this experiment the observer views the two
scenes simultaneously in different eyes.  One eye is exposed to a
large neutral surface illuminated by, say, a daylight lamp.  The other
eye is exposed to an equivalent large neutral surface illuminated by,
say, a tungsten lamp.  These surfaces define a background stimulus
that is different to each eye.  The two backgrounds fuse in
appearance, and appear as a single large background.  To establish the
asymmetric color-matches, the experimenter places a test object on top
of the standard background seen by one eye.  The observer selects a
matching object from an array of choices seen by the second eye.  The
color appearance mapping is defined by measuring the cone absorptions
of the {\em test} and {\em matching} objects, usually small colored
papers, seen under their respective illuminants.

The dichoptic method has the advantage that the matches may be set
quickly, avoiding the tedious delays required for visual adaptation in
memory matches.  The method has the disadvantage of making the
assumption that adaptation occurs independently, prior to binocular
combination ~\footnote{ This binocular method makes sense if one
accepts the view that the adjustment for the illumination is mediated
primarily before the signals from the two eyes are combined, in the
superficial layers of area V1.  The coherence of the experimental
method can be tested psychophysically by examining transitivity.  The
observer matches a test on backgrounds L and $R_1$, and then on
backgrounds L and $R_2$.  The experimenter then places $R_1$ in the
left eye and $R_2$ in the right eye and verifies that the matching
lights match one another.  There is no guarantee, of course, that
these measurements are governed by precisely the same visual
mechanisms that govern adaptation under normal viewing conditions.  }.

The experimental methods illustrated in Figure~\ref{f8:matchingMethods}
generalize conventional usual color-matching experiment.  These
methods are called {\em asymmetric} color-matching because, unlike
conventional color-matching, in these experiments the matches are set
between stimuli presented in different contexts.  As we have already
seen, because color appearance discounts estimated changes of the
illumination, matches set in the asymmetric color-matching experiment
are {\bf not} cone absorption matches.  Rather, the observer is
establishing a match at a more central site following the correction
for the properties of the scene.

The asymmetric color-matching experiment is directly relevant to the
questions raised by our computational analysis of color appearance.
Moreover, the experiment has a central place in the study of color
appearance simply for practical experimental reasons.  There are many
general questions we might ask about color appearance.  For example,
we would like to be able to measure which colors are similar to one
another; which colors have a common hue, saturation or brightness, and
so forth.  If we had to study these questions separately under each
illuminant, the task would be overwhelming.  By beginning with the
asymmetric color-matching experiment, we can divide color appearance
measurements into two parts and reduce our experimental burden.  The
asymmetric matches define a mapping between the cone absorptions of
objects with the same color appearance seen under different
illuminants.  From these experiments, we learn how to convert a target
seen under one illuminant into an equivalent target under a standard
illuminant.  This transformations saves a great deal of experimental
effort since we can focus most of our questions about color appearance
on studies using just one standard illuminant.

\subsection*{The linearity of asymmetric color-matches}
We have seen measurements of superposition to test linearity
throughout this volume.  Tests of linearity in asymmetric color
matching appear very early in the color appearance literature.  When
Von Kries (1902, 1905) introduced the coefficient law, he listed several
testable empirical results.  Among the predictions of the basic law
he listed the basic test of linearity, namely
\begin{quote}
 ... there exist several very simple laws, which also appear to be
specially adapted for experimental test.  Namely, it must be that if
$L_1$ on one retinal region causes the same result as $L_2$ on
another, and similarly $M_1$, working on the first, causes the same
effect as $M_2$ on the other, in every case also $L_1 + M_1$ must have
here the same effect as $L_2 + M_2$ there.

 ...  The extended studies of Wirth (1900-1903) show that the law can
be considered as nearly valid for reacting lights that are not too
weak.
\end{quote}
Evidently, von Kries not only raised the question of linearity of
asymmetric color-matching, but by 1905 he considered it answered
affirmatively.

While von Kries considered the question settled, not everyone was
persuaded.  Over the years, there have been many separate experimental
tests of linearity in the asymmetric color-matching experiment.  I am
particularly impressed by a series of papers by Elaine Wassef, working
first in London and then at the University College for Girls in Cairo.
Wassef wrote at roughly the same time E. H. Land was working at
Polaroid.  In her papers, she reports on new studies and a review of
the experimental test of asymmetric color-matching linearity\footnote{
Interestingly, one of the largest sets of data she reviewed was a
series of experiments performed at the Kodak research laboratories,
Polaroid's competitor.  }  (Wassef, 1952, 1958, 1959) Like von Kries,
Wassef asked whether one could predict asymmetric color-matches using
the principle of superposition.  And, like von Kries, she concluded
that the weight of the experimental evidence supported the linearity
hypothesis: When the illumination changes, the cone absorptions of the
test and matching lights are related by a linear transformation.

I have replotted some of Wassef's data to illustrate the nature of the
measurements and the size of the effect (Figure~\ref{f8:wassefShift}).
The illuminant spectral power distributions she used in her dichoptic
matching experiment are plotted in Figure~\ref{f8:wassefShift}a.  To
plot her results, I have converted Wassef's reported measurements into
one absorptions.  I have plotted the cone absorptions of the surfaces
that matched in color appearance when seen under the two illuminants.
shows the $\Red$ and $\Green$ cone absorptions of the surfaces under
the two illuminants, and Figure~\ref{f8:wassefShift}c shows the $\Red$
and $\Blue$ cone absorptions.  The cone absorptions for objects seen
under a tungsten illuminant are plotted as open
circles; the cone absorptions for objects seen under a blue sky
illumination are plotted as filled squares.  The size of the effect is
quite substantial.  Two sets of points show the cone absorptions of
targets that look identical in their respective contexts.  Yet, the
cone absorption values from the surfaces under these illuminants don't
even overlap in their values.
\begin{figure}
\centerline{
 \psfig{figure=../09col/fig/wassefShift.ps ,clip= ,width=5.5in}
}
\caption[Color constancy Effect Size]{
{\em Data from an asymmetric color-matching experiment using the
dichoptic method.}  The test and matching lights are viewed in
different contexts and appear identical.  But, the two lights have
very different cone absorption rates.  Hence, appearance matches made
across an illuminant change are not cone absorption matches.  The
spectral power distributions of two illuminants, one approximating
mean daylight and the other a tungsten illuminant are shown in panel a.
Cone absorptions for targets that appear identical to one another in
these two contexts are shown as scatterplots in (b) for the
($\Red$,$\Green$) cones, and (c) for the ($\Red$,$\Blue$) cones.  The
points plotted as open circles are cone absorptions for tests seen
under the first illuminant; matches seen under the second illuminant
are plotted as filled squares.  The stimuli represented by the
absorptions have the same color appearance, but they correspond to
very different cone absorptions (Source: Wassef, 1959).  }
\label{f8:wassefShift}
\end{figure}

Taken together, the color matching and asymmetric color-matching show
the following.  When the objects are in the same context, equating the
cone absorptions equate appearance.  But, when the two objects are
seen under different illuminants, equating cone absorptions does not
equate for appearance.  Within each context the observer uses the
pattern of cone absorptions to infer color appearance, probably by
comparing the relative cone absorption rates.  Color appearance is an
interpretation of the physical properties of the objects in the image.

\subsection*{Von Kries Coefficient Law:  Experiments}

Through his Coefficient Law, J. Von Kries sought to explain these
asymmetric color matches by a simple physiological mechanism.  He
suggested that the visual pathways adjust to the illumination by
scaling the signals from the individual cone classes.  This
hypothesis has a simple experimental prediction: If we plot, say, the
$\Blue$ cone absorptions of the test and match surfaces on a single
graph, the data should fall along a straight line through the origin.
The slope of the predicted line is the scale factor for the
illuminant change.

Neither von Kries or Wassef knew the photopigment spectral curves;
hence, they could not create the graph they needed to test the von
Kries Coefficient Law directly.  But, using an indirect measurement
based on estimation of the eigenvectors of the measured linear
transformations, Burnham et al. (1957) and Wassef (1959) rejected von
Kries scaling.  Despite this rejection, von Kries' hypothesis
continued to be used widely to explain how color appearance varies
with illumination.  Among theorists, for example, E. H. Land relied
entirely on von Kries scaling as the foundation of his retinex theory
(Brewer, 1954; Brainard and Wandell, 1986; Land, 1986).

\begin{figure}
\centerline{
  \psfig{figure=../09col/fig/wassefVK.ps,clip= ,width=5.5in}
}
\caption[Wassef-von Kries]{
{\em The cone absorptions of the test and match surfaces} fall close
to a straight line.  These appearance matches were made by presenting
the test and match objects to different eyes.  The illuminant for one
eye was similar to a tungsten bulb and the other eye was blue
skylight.  {\em The Von Kries coefficient law} predicts that the line
should pass through the origin of the graph; while not precisely
correct, the rule is a helpful starting point (Source: Wassef, 1959).
}
\label{f8:wassefVK}
\end{figure}
Today, we have good estimates of the spectral sensitivities of the
cone photopigments and it is possible convert Wassef's data into cone
absorptions and analyze von Kries coefficient law directly.
Figure~\ref{f8:wassefVK} shows a graphical evaluation of von Kries
hypothesis for the data in Figure~\ref{f8:wassefShift}.  Each panel
plots the cone absorptions of corresponding test and match targets for
one of the three cone types.  As predicted by Von Kries, the cone
absorptions of the test and match targets fall along a line.
Moreover, the slope of the lines relating the cone absorptions also
make sense.  The slope is largest for the $\Blue$ cones where
illuminant change has its largest effect.  The data are not perfectly
consistent with von Kries scaling, however, because the lines through
the data do not pass through the origin, as required by the
theory~\footnote{Indeed, this is equally a failure of the simple
linearity that Wassef uses to summarize the data, and more in line
with some of the conclusions that Burnham et al. (1957) drew about their
data.}.

There is an emerging consensus in many branches of color science that the
von Kries coefficient law explains much about how color appearance
depends on the illumination.  J. von Kries simple hypothesis is
important partly because of its practical utility, and partly because
of its implications for the representation of color appearance within
the brain.  The hypothesis explains the major adjustments for color
constancy to in terms of the photoreceptor signal, and the
photoreceptor signals combine within the retina
(Chapter~\ref{chapter:wavelength}).  Hence, von Kries hypothesis
implies that either (a) the main adjustment takes place very early, or
(b) the photoreceptor signals can be separated in the central
representation.  This topic will come up again later, when we review
some of the phenomena concerning color appearance in the central
nervous system.

\subsection*{How Color Constant Are We?}
Finally, let's consider how well the visual pathways correct for
illumination change.  On this point there is some consensus: The
asymmetric color-matches do not compensate completely for the
illumination change.  The visual pathways compensate for only part of
the illuminant change (Helson, 1938; Judd, 1940).

Brainard and Wandell (1991,1992) described this phenomenon using results
from a recent experiment.  We used an experimental apparatus
consisting of simulated surfaces and illuminants and an asymmetric
color-matching experiment based on memory-matches.  We presented
subjects with images of simulated colored papers, rendered under a
diffuse daylight illuminant, on a CRT display.  The subjects memorized
the color appearance of one of the surfaces.  Next, we changed the the
simulated illuminant, slowly over a period of two minutes, giving
subjects a chance to adapt to the new illuminant.  Then, the subject
adjusted the appearance of a simulated surface to match the color
appearance of the surface they had memorized.

\begin{figure}
\centerline{
  \psfig{figure=../09col/fig/equivIll.ps ,clip= ,width=5.5in}
}
\caption[Equivalent Illuminant]{
{\em A comparison of the illuminant change and the subjective
illuminant change}, as inferred from an asymmetric matching
experiment.  The simulated illuminant change and subjective illuminant
changes are shown by the filled squares and open squares respectively.
Subjects behave as if the illuminant change is about half of the true
illuminant change (Source: Brainard and Wandell, 1991).  }
\label{f8:equivIll}
\end{figure}
We can represent the difference between the two simulated illuminants
by plotting the illuminant change.  The filled symbols in
Figure~\ref{f8:equivIll} show the illuminant changes in two
experimental conditions.  The top panel shows an illuminant change
that increased the short-wavelength light and decreased the
long-wavelength light.  The bottom panel shows an illuminant change
that increased the energy at all wavelengths.

Suppose that subjects equated the perceived surface reflectance, but
that the illuminant change they estimated was different from the true
illuminant change.  In that case, we can use the observed matches to
infer the the subjects' illuminant estimates, which are plotted as the
open symbols in two panels of Figure~\ref{f8:equivIll}.  Subjects are
acting as if the illuminant change they are correcting for is similar
to the simulated illuminant change but, smaller.  Subjects'
performance is conservative, correcting for about half the true
illuminant change.

Brainard and Wandell's (1991,1992) experiments were conducted on display
monitors, and the images were far less interesting than full natural
scenes.  It is possible that given additional clues, subjects may come
closer to true illuminant estimation.  But, in most laboratory
experiments to date, subjects do not compensate fully for changes in
the illumination.  When the illumination changes color appearance
changes less than it might if color was defined by the cone
absorptions; but, it changes more than it would if the nervous system
used the best possible computational algorithms.  The performance of
biological systems often seems to fall in this regime.  Very poor
behavior is forced to change towards a better solution.  But, the
evolutionary pressure does not force our nervous system to solve
estimation problems perfectly.  When the marginal return for
additional improvements is not great, pretty well seems to do.
