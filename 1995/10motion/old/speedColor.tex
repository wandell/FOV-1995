
\begin{figure}
\centerline{
 \psfig{figure=../09motion/fig/speedColor.ps,clip= ,height=2.5in}
}
\caption[Speed depends on color]{
Cavanagh, Tyler and Favreau, fig. 2, middle column
at 1.2 deg/sec 0.8 cpd grating.
}
\label{f9:speedColor}
\end{figure}
\paragraph{Perceived velocity depends on color.}
Moreland (1980) and Cavanagh, Tyler and Favreau (1984)
report that the perceived speed of a colored grating
changes as we reduce the luminance component of the
grating.
Subjects adjusted the velocity of a luminance grating
presented in the upper field to match the velocity
of a colored grating in the lower field.
Figure \ref{f9:speedColor} shows the perceived speed of
a colored red-green grating as its luminance component 
was varied.
For all three subjects, the colored grating slowed 
considerably as the luminance component of the target
was eliminated.
The size of these effects is consistent with the luminance
measurements in the Stone and Thompson study described above.

When the colored grating has zero luminance contrast,
it remains very visible.
This is quite counter-intuitive, though not too dissimilar
from the luminance effects.
When the speed of gratings in contrast ratios
that differ by a factor 10 are compared 
by Stone and Thompson (1991) the perceived velocities
often differ by as much as a factor of 2.
For example, a 70 percent contrast grating and a 7 percent
contrast grating are perceived at quite different
speeds.
The 7 percent contrast grating is very visible.

