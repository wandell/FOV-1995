\begin{figure}
\centerline{
 \psfig{figure=../09motion/fig/mgc.ps,clip= ,height=3.5in}
}
\caption[Gradient Constraint]{
One-dimensional motion gradient constraint.
The curve shows the translated intensity separated
by one unit of time.
The change in intensity at a point in time
($f_t$) is predicted by the velocity
and the slope of the intensity ($v f_x$).
}
\label{f9:gradConstraint}
\end{figure}


-------------

Spare text.


For example, we might not
weight the points in the small region equally.
Points closer to the center may be more important than
those further away, so we might weight
the intensities into a ``pseudo-point''
before calculating the derivative.
Or, we might create a weighted sum
that emphasizes information in different frequency bands,
say by using the Discrete Cosine Transform,
before we take derivatives.
As you can imagine, there is tremendous room for
selecting different grouping methods.

Despite the many variants,
Equation \ref{f9:motionFlow} does give us a few clear messages.
First, the variations I have described begin with a weighted
sum of the local intensities followed by a derivative.
Both of these steps are linear; we can always combine these
two steps into a single linear filtering operation.
The linear filtering operation will provide us information
about the space-time orientation of the local intensity pattern.
This linear filter information 
is coded into the entries of the matrix $\bf M$.

Second, we estimate the velocity vector
by selecting a velocity vector
that minimizes the error in the linear Equation \ref{e9:motionFlow}.
Usually, we will find a least-squares estimate,
so that the velocity estimate will
have a non-linear dependence on the filter outputs,
involving squaring the filtered responses.

There are many competing methods of computing motion flow fields.
So, pick one and let's move on to the next problem.
See the appendix for one specific method.

The question we treat next is about the structure
of the information in the motion flow field.

