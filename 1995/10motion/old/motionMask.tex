\subsection{Psychophysical Estimates of Neural Direction Selectivity}
A number of psychophysical investigators are exploring
the hypothesis that 
direction-selective neurons in primary visual cortex (area V1)
are the physiological elements that
define the space-time oriented filters needed in motion
estimation algorithms.
We will consider some measurements of the space-time
receptive fields of these units in Section \ref{sec9:physio}.

If the responses of a single group of neurons
with space-time oriented receptive fields
are essential to perceiving a moving stimulus, then perhaps
when we study the limits of our ability to perceive
motion we will be studying limitations on the signals
in the direction-selective neurons of V1.
Anderson, Burr and Morrone (1991, 1990;
Daugman, 1984; Harvey and Doan, 198?) have measured
subjects' ability to discriminate motion in an attempt
to correlate these behavioral measures with
neurophysiological measures of direction-selective
receptive fields.
Perhaps, the behavioral performance will serve as an indicator
of the space-time receptive field of the 
motion-directive neurons.

In Anderson et al.'s experiments observers
adjusted the contrast of
a vertical sign wave until it was possible
to discern whether the stimulus was drifting to the right or left.
Observers made their setting in the presence of a
masking stimulus.
The masking stimuli were also sinusoids, but 
they had varying spatial frequencies and orientations.
Also, the masking stimulus did not have a steady motion,
but rather it jumped randomly back and forth across the screen.

\begin{figure}
\centerline{
 \psfig{figure=../09motion/fig/motionMask.ps,clip= ,height=3.5in}
}
\caption[Direction of Motion in the Presence of Masking Stimuli]{
Masking data as a surface plot.
Redrawn from Anderson et al. 1991. JOSA paper.
}
\label{f9:motionMask}
\end{figure}
The data in Figure \ref{f9:motionMask} show that the
mask was most effective when it was most similar to the
target.
The panel on the left contains the data for 
a targets at 1 cpd.
The bottom axes of the plots show the spatial frequency
and orientation of the masking stimulus.
The height of the curve shows the threshold elevation,
compared to no mask,
required before the observer felt he could see the
direction of motion.
The panel on the right shows the estimated
spatial receptive field of this neuron;
the experiment yields no estimate of its temporal
properties so we cannot verify the space-time orientation.

The measurements in these masking experiments
vary as with the parameters of the mask and test stimuli.
As far as I know, there have been no mixture experiments
performed to verify that one can generalize from
measurements with these masks to measurements with
other masks.
You may wish to return to
Chapter \ref{chapter:space} to remind yourself of
some of the challenges in interpreting
masking experiments.


