\section{The Optical Quality of the Eye}
We are now ready to correct the measurements
for the effects
of double passage through the optics of the eye.
To make the method easy to understand,
we will analyze how to do the correction 
by first making the assumption that the optics introduce
no phase shift into the retinal image;
this means, for example, that
a cosinusoidal stimulus
creates a cosinusoidal retinal image, scaled in amplitude.
It is not necessary to assume that there is no phase shift but
the assumption is reasonable
and the main principles of the analysis are easier to see
if we assume there is no phase shift.

\begin{figure}
\centerline{
\psfig{figure=../02imgfor/fig/doublepass.ps,clip=,height=3.0in}
}
\caption[Sinusoids and Double Passage]{ 
{\em Double passage.}
(a) The amplitude, $A$, of an input cosinusoid stimulus
is scaled by a factor, $s$, after passing
through even-symmetric shift-invariant symmetric optics
as shown in part (b).
(c) Passage through the optics a second
time scales the amplitude again, resulting
in a signal with amplitude $s^2 A$.
}
\label{f1:doublepass}
\end{figure}
To understand how to correct for double passage, consider
a hypothetical alternative experiment Campbell and Gubisch might have done
(Figure \ref{f1:doublepass}).
Suppose Campbell and Gubisch had used input stimuli equal
to cosinusoids at various spatial frequencies, $f$.
Because the optics are shift-invariant and there is no frequency-dependent phase shift,
the retinal image of a cosinusoid at frequency $f$ 
is a cosinusoid scaled by a factor $s_f$.
The retinal image passes back through the optics
and is scaled again, so that the measurement
would be a cosinusoid scaled by the factor ${s_f}^2$.
Hence, had Campbell and Gubisch used a cosinusoidal input stimulus, 
we could deduce the retinal image from the measured image easily:
The retinal image would be a cosinusoid with an amplitude
equal to the square root of the amplitude of the measurement.

Campbell and Gubisch used a single
line, not a set of cosinusoidal stimuli.
But, we can still apply the basic idea of
the hypothetical experiment to their measurements.
Their input stimulus, defined over $N$ locations, is
\begin{equation}
\pixveci{i} =
 \left\{ 
  \begin{array}{ll}
   N & \mbox{if $i=0$} \\
   0 & \mbox{if $1 \leq i < N$}
  \end{array}
 \right.
\end{equation}

As I describe in the appendix,
we can express the stimulus as the weighted sum
of harmonic functions by using the {\em discrete Fourier series}.
The representation of a single line
is equal to the sum of cosinusoidal functions
\begin{equation}
\label{e1:lineDFT}
\pixvechati{i} = 0.5 + \sum_{f=1}^{f=N-1} \cos( 2 \pi f {i/N})~~.
\end{equation}
Because the system is shift-invariant, 
the retinal image of each cosinusoid
was a scaled cosinusoid, say with scale factor $s_f$.
The retinal image was
scaled again during the second pass through the optics,
to form the cosinusoidal term they measured.
\footnote{
Be bothered by
the fact that the discrete Fourier series approximation is an infinite
set of pulses, rather than a single line.
To understand why, consult the Appendix.
}

Using the discrete Fourier series, we also 
can express the measurement
as the sum of cosinusoidal functions,
\begin{equation}
\mbox{Measurement} = 0.5 + \sum_{f=1}^{f=N-1} (s_f)^2 \cos( 2 \pi f {i/N}).
\end{equation}
We know the values of ${s_f}^2$, since this was Campbell and Gubisch's
measurement.
The image of the line at the retina, then, must have been
\begin{equation}
\lspreadi{i} = 0.5 + \sum_{f=1}^{f=N-1} s_f \cos( 2 \pi f {i/N}).
\end{equation}
The values $\lspreadi{i}$ define the linespread function of the eye's optics.
We can correct for the double passage
and estimate the linespread
because the system is linear and shift-invariant.

As you read further about experimental and computational methods in vision science,
remember that there is nothing inherently important
about sinusoids as visual stimuli;
we must not confuse the stimulus with
the system or with the theory we use to analyze the system.
When the system is a shift-invariant linear system,
sinusoids can be helpful in simplifying our calculations
and reasoning, as we have just seen.
The sinusoidal stimuli are important
only insofar as they help us to measure or clarify
the properties of the system.
And if the system is not shift-invariant,
the sinusoids may not be important at all.

\subsection*{The Linespread Function}
\begin{figure}
\centerline{
\psfig{figure=../02imgfor/fig/cg.linespread.ps,clip=,height=3.0in}
}
\caption[The Estimated Linespread Function]{ 
{\em The linespread function of the human eye.}
The solid line in each panel
is a measurement of the  linespread.
The dotted lines are the
diffraction-limited linespread for a pupil
of that diameter.
(Diffraction is explained later in the text).
The different panels show
measurements for a variety of pupil diameters
(From Campbell and Gubisch, 1967).
%Fig. 10 Optical quality of the human eye, J. of Physiology.
}
\label{f1:cg.linespread}
\end{figure}
Figure \ref{f1:cg.linespread} contains Campbell and Gubisch's
estimates of the linespread functions of the eye.
Notice that as the pupil size increases, the width of
the linespread function increases which indicates that
the focus is worse for larger pupil sizes.
As the pupil size increases, light reaches the retina
through larger and larger sections of the lens.
As the area of the lens affecting the passage of light
increases, the amount of blurring increases.

The measured linespread functions, $\lspreadi{i}$, along with
our belief that we are studying a shift-invariant linear system,
permit us to predict the retinal image
for any one-dimensional input image.
To calculate these predictions, it is convenient to
have a function that describes the linespread of the human eye.
G. Westheimer (1986) suggested the following formula
to describe the measured linespread function of the human eye, when
in good focus, and when the pupil diameter is near 3mm.
\begin{equation}
\lspreadi{i} = 0.47 e^{ - 3.3 i ^ {2} } + 0.53 e ^ { -0.93 | i | }
\end{equation}
where the variable $i$ refers to position on the
retina specified in terms of minutes of visual angle.
A graph of this
linespread function is shown in Figure \ref{f1:westheimer.ls}
\begin{figure}
\centerline{
 \psfig{figure=../02imgfor/fig/westheimer.ls.ps,clip=,height=3.0in}
}
\caption[Westheimer's Linespread Function]{ 
{\em An analytic approximation of the human linespread function}
for an eye with a 3.0mm diameter pupil (Westheimer, 1986).
}
\label{f1:westheimer.ls}
\end{figure}
\nocite{Westheimer1964}

We can use Westheimer's linespread function
to predict the retinal image of
any one-dimensional input stimulus\footnote{
Westheimer's linespread function is for an
average observer under one set of viewing conditions.
As the pupil changes size and as observer's age,
the linespread function can vary.
Consult IJspeert et al. (1993) and Williams et al. (1995)
for alternatives to Westheimer's formula.}.
Some examples of the predicted
retinal image are shown in Figure \ref{f1:blurring}.
Because the optics blurs the image,
even the light from a very fine line is spread across
several photoreceptors.
We will discuss the relationship between the optical defocus and
the positions of the photoreceptors in Chapter~\ref{chapter:mosaic}.
\begin{figure}
\centerline{
\psfig{figure=../02imgfor/fig/blurring.ps,clip=,height=3.0in}
}
\caption[Example Retinal Images]{ 
{\em Examples of the effect of optical blurring.}
(a) Images of a line, edge and a bar pattern.
(b) The estimated retinal image
of the images after blurring by Westheimer's linespread function.
The spacing of the photoreceptors in the retina
is shown by the stylized arrows.
}
\label{f1:blurring}
\end{figure}

\subsection*{The Modulation Transfer Function}

%7 high by 6.5 wide
\begin{figure}
\centerline{
\psfig{figure=../02imgfor/fig/otfDataTheory.ps,clip=,height=3.0in}
}
\caption[The MTF for Westheimer's Linespread]{ 
{\em Modulation transfer function}
measurements of the optical quality of the lens
made using visual interferometry 
(Williams et al., 1995; described in Chapter \ref{chapter:mosaic}).
The data are compared with the predictions
from the linespread suggested by Westheimer (1984)
and a curve fit through the data by Williams et al. (1995).
}
\label{f1:otfTheoryData}
\end{figure}
In correcting for double passage, we thought about
the measurements in two separate ways.
Since our main objective was to derive the linespread function,
a function of spatial position,
we spent most of our time thinking of the measurements
in terms of light intensity as a function of spatial position.
When we corrected for double passage through the optics, however,
we also considered a hypothetical experiment in which
the stimuli were harmonic functions (cosinusoids).
To perform this calculation,
we found that it was easier to correct for double passage
by thinking of the stimuli as the sum of harmonic functions,
rather than as a function of spatial position.

These two ways of looking at the system,
in terms of spatial functions or sums of harmonic functions,
are equivalent to one another.
To see this,
notice that we can use the linespread function to derive the retinal
image to any input image.
Hence, we can use the
linespread to compute the scale factors of the harmonic functions.
Conversely, we already saw that by measuring how the system responds
to the harmonic functions, we can derive the linespread function.
It is convenient to be able to
reason about system performance in both ways.

The {\em optical transfer function} defines the system's
complete response to harmonic functions.
The optical transfer function is a complex-valued
function of spatial frequency.
The complex values code both the scale
factor and the phase shift the system
induces in each harmonic function.

When the linespread function of the eye is an even-symmetric
function, there is no phase shift of the harmonic functions.
In this case, we can describe the system
completely using a real valued function,
the {\em modulation transfer function}.
This function defines the scale factors applied to
each spatial frequency.
The data points in Figure \ref{f1:otfTheoryData}
show measurements of the modulation transfer function of the human eye.
These data points were measured using a method called
{\em visual interferometry} that is described in Chapter~\ref{chapter:mosaic}.
Along with the data points in Figure~\ref{f1:otfTheoryData},
I have plotted the predicted modulation transfer
function using Westheimer's
linespread function and a curve
fit to the data by Williams et al. (1995).
The curve derived by Westheimer (1986) using completely
different data sets differs from the measurements by Williams et al.
(1995) by no more than about twenty percent.
This should tell you something about the relative
precision of these descriptions of the optical quality of the lens.

The linespread function and the modulation transfer function
offer us two different ways to think about the optical quality
of the lines.
The linespread function in Figure \ref{f1:westheimer.ls},
describes defocus as the spread
of light from a fine slit across the photoreceptors:
the light is spread across three to five photoreceptors.
The modulation transfer function in Figure \ref{f1:otfTheoryData}
describes defocus
as an amplitude reduction of harmonic stimuli:
beyond 12 cycles per degree the amplitude
is reduced by more than a factor of two.
