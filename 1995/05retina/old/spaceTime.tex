From additivity we know that when we present
the neuron with a pair of these point stimuli at two different
locations the observed response will simply be the sum
of the individual responses.
This is what
Enroth-Cugell and Pinto observed in Figure \ref{f4:ec.pinto}.
If we present a tiny flickering spot at location $(x,y)$
and a second flickering spot at location $(x' , y')$ we expect that
the sum to both together will be

\begin{eqnarray}
\responset  & = & 
   \sigma_{0} + A(x,y) \cos ( 2 \pi f_t t ) + B(x,y) \sin ( 2 \pi f_t t ) \nonumber \\
& + & A(x',y') \cos ( 2 \pi f_t t ) + B(x',y') \sin ( 2 \pi f_t t) \nonumber \\
& = &  \sigma_{0} + [ A(x,y) + A(x',y') ] \cos ( 2 \pi f_t t)  \nonumber \\
& + & [ B(x,y) + B(x',y') ] \sin ( 2 \pi f_t t)
\end{eqnarray}

From the linear systems principle of homogeneity,
we know that
if we halve the intensity of the flickering point of light,
then the response will halve as well.
If the input stimulus is a point of light
with a contrast of one at location $(x,y)$ and a contrast
of one-half at location $(x',y')$, then we expect the output
to be
\begin{eqnarray}
\label{e3:g1}
\responset
  = \sigma_{0} & + & [ A(x,y) + { 1 / 2 } A(x',y') ] \cos ( 2 \pi f_t t) \nonumber \\
  & + & [ B(x,y) + { 1 / 2 } B(x',y') ] \sin ( 2 \pi f_t t)
\end{eqnarray}
By adding more and more points together, each with its own contrast
value, we can build any contrast pattern.

From Equation \ref{e3:g1} we see that
the predicted output
is the weighted sum of the measurements made to
the points of unit contrast at position $(x,y)$,
$A(x,y)$ and $B(x,y)$.
For an arbitrary image, whose contrast differs at each point,
the weights are the pattern contrasts at each point.

The general formula for predicting the response to a
contrast pattern $\contrastxy$ is just slightly
different from what we have been using.
The temporal response can be written in two parts,
as shown in this equation.
\begin{eqnarray}
\label{e3:g2}
\responset 
   = \sigma_{0} & + & [\sum_{x,y} \contrastxy A (x,y) ] \cos ( 2 \pi f_t t ) \nonumber \\
   & + & [\sum_{x,y} \contrastxy B (x,y) ] \sin ( 2 \pi f_t t ) .
\end{eqnarray}
Compare equation \ref{e3:g2} with equation \ref{e3:g0} to see that
we can derive the two weights $A_{c}$ and $B_{c}$ from
the contrast pattern as follows.
\begin{eqnarray}
\label{e3:g3}
A_{c} & = &\sum_{x,y} \contrastxy A (x,y) \nonumber \\
B_{c} & = &\sum_{x,y} \contrastxy B (x,y)
\end{eqnarray}

------

To see why,
use Equation \ref{e3:g4} to re-write Equation \ref{e3:g2} as
\begin{eqnarray}
\label{e3:g5}
\responset
   & = & \sigma_{0} + (\sum_{x,y} \contrastxy A (x,y) )
   ( \cos ( 2 \pi f_t t ) + \alpha \sin ( 2 \pi f_t t ) ) \nonumber \\
   & = & \sigma_{0}  + A_{c} ( \cos ( 2 \pi f_t t ) + \alpha \sin ( 2 \pi f_t t ) ) .
\end{eqnarray}
In this case only one number,
$A_{c}$, depends upon the contrast pattern.
This number can be computed using the point measurements $A(x,y)$,
so in this special case the function $A(x,y)$ represents
the linear spatial receptive field of the neuron.

Equation \ref{e3:g5} defines the special condition that must
hold if there is but one linear receptive field.
When this equation holds,
the cosinusoidal temporal output
will always have the same temporal phase,
$\tan^{-1} ( \alpha )$, no matter what spatial
contrast pattern we use.
We can check this fact experimentally
to see whether we are justified
in summarizing the neuron's input-output
relationship using only a single 
spatial receptive.
