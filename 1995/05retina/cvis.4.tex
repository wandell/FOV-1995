%
% References go here
%
\newpage
\section*{Exercises}

\be %All Questions

\item Suppose that a monkey retinal ganglion cell of the X type has
a center that is fed by $\Red$ cones and an
inhibitory surround that is fed by $\Green$ cones.

 \be

 \item  Suppose we measure the the retinal
ganglion cell's receptive field using a 670nm light.
Draw the one-dimensional you will measure.

 \item Draw the one-dimensional receptive field you will measure
using a 560nm light.

 \item Draw the contrast sensitivity function
you will measure using contrast patterns
at each of these two different wavelengths presented on a
neutral background.

 \ee

\item  Robson (1980) writes about physiological measurements
of neural response:

\begin{quote}
if a cell's responses to a stimulus are so unreliable
that we have to average responses to dozens or even hundreds of
stimulus presentations in order to see that the cell {\em is}
responding to the stimulus, then what do our
measurements have to do with vision?
\end{quote}

Write an answer to his question.

\item  In the central foveal representation,
the ratio of photoreceptors
to ganglion cells and ganglion cells to cortical
cells has preoccupied anatomists.
What are the current best estimates?
What features of the fovea make these numbers
hard to estimate?

\item Answer these questons about the 
parvocellular and magnocellular pathways.

\be
\item Based on the properties of the
neurons described in this chapter,
what hypothesis do you have about the functions
of these two pathways?

\item Design an experimental test of your hypothesis.

\item Do you think that the signals from the parvo- and magno-cellular
pathways will remain segregated further along in the
visual pathways?  If so, to what end?

 \item Livingstone et al. (1991) have argued that developmental
dyslexia is due to a dysfunction in the magnocellular pathway.
Read their paper and read about dyslexia.
What more do you need to learn to decide whether they are right?
If they are, what types of cures do you think might be possible?

\ee

\item Answer the following questions about receptive fields.

\be

 \item The outputs of individual retinal ganglion cells
synapse on neurons in the lateral geniculate pathway.
The receptive fields of neurons in the lateral geniculate,
however, are quite similar to the receptive fields
of neurons retinal ganglion cells.
What implication does this have for the connectivity
pattern between retinal ganglion cells and lateral geniculate neurons?

 \item The receptive fields of neurons in the superficial
layers of the visual cortex
do not have the same center-surround organization
as lateral geniculate neurons.
Suppose that the response of a cortical neuron is the weighted
sum of the responses of two nearby retinal ganglion cells.
What will the cortical receptive field shape be?


 \item What advantage is there to having
a center-surround organization?

 \item In a classic study of the frog retina, Lettvin,
Maturana, McCulloch and Pitts describe how they analyze
the receptive fields of retinal ganglion cells in the frog.
\begin{quote}
We decided then how we ought to work.
First, we should find a way of recording from single
myelinated and unmyelinated fibers in the intact optic nerve.
Second, we should present the frog with as wide a range of
visible stimuli as we could, not only spots of light
but things he would be disposed to eat, other things from which
he would flee, sundry geometrical figures, stationary and moving
about, etc.
From the variety of stimuli we should then try to discover what common
features were abstracted by whatever groups of fibers we could find in
the
optic nerve.
Third, we should seek an anatomical basis for the grouping.
[page 1942, Lettvin et al., 1959]
\end{quote}

In the primate retina,
their approach is not particularly useful since
linearity holds so well.
Beyond the retina, however, there are many neurons whose
responses are nonlinear and we do not undersand.
Explain whether you think
Lettvin et al.'s approach is attractive for the study of
those areas.

 \item Later in their discussion section Lettvin et al. write

\begin{quote}
One might attempt to measure numerically how the response of each kind
of fiber varies with various
properties of the successions of patterns of light which evoke them.
But to characterize a succesion of patterns in space requires
knowledge of so
many independent variables that this is hardly possible by
experimental enumeration of cases. ...
We would prefer to state the operations of ganglion cells as simply as
possible in
terms of whatever {\em quality} they seem to detect ...
\end{quote}

Comment on how they might have obtained added efficiencies in their
measurement procedures.

 \ee

\item Answer these questions about visual adaptation.

\be

 \item Weber's Law does not hold equally well when we measure using
contrast patterns with different spatial frequencies.  What type of
neural adaptation mechanism could achieve this effect?

 \item Explain whether the spatial and wavelength sensitivities of
visual adaptation have to be the same as the spatial and
wavelength properties of a system that communicates information
about the pattern on the retina.

\ee


\ee %All Questions
